\documentclass[12pt]{article}
\usepackage{hhline}
\usepackage{graphicx}
\graphicspath{{pictures/}}
\DeclareGraphicsExtensions{.png}
\usepackage{multirow}
\usepackage{amsmath}
\usepackage{mathtext}
\usepackage[T2A]{fontenc}
\usepackage[utf8]{inputenc}
\usepackage{pscyr} 
\usepackage[left=2.5cm,right=2.5cm, top=1.5cm,bottom=1cm,bindingoffset=0cm]{geometry}
\begin{document}
\pagestyle{empty}
\begin{center}
\large{\textbf{Университет ИТМО}}
\end{center}
\rule{500pt}{1pt}
\par\bigskip\par\bigskip\par\bigskip\par\bigskip\par\bigskip\par\bigskip\par\bigskip\par\bigskip
\begin{center}
\Large
\textbf{Лабораторная работа №1}

\textbf{\textit{«Основные понятия линейного программирования»}}


\end{center}
\par\bigskip\par\bigskip\par\bigskip\par\bigskip\par\bigskip\par\bigskip\par\bigskip\par\bigskip\par\bigskip\par\bigskip\par\bigskip\par\bigskip\par\bigskip\par\bigskip      
\begin{flushright}
\large
Выполнил: Федюкович С. А.
\par\bigskip
Факультет: МТУ “Академия ЛИМТУ”
\par\bigskip
Группа: S3100                       
\par\bigskip\par\bigskip\par\bigskip

\rule{150pt}{0.5pt}
\par\bigskip\par\bigskip\par\bigskip\par\bigskip                                                            
 Проверила: Авксентьева Е. Ю.
\par\bigskip \par\bigskip

\rule{150pt}{0.5pt}
\end{flushright}
\par\bigskip\par\bigskip\par\bigskip\par\bigskip\par\bigskip\par\bigskip\par\bigskip\par\bigskip\par\bigskip\par\bigskip     
\begin{center}
\large
Санкт-Петербург
\par\bigskip
2018
\end{center}
\newpage
\section*{Теоретические основы лабораторной работы}

Линейное программирование – это направление математического
программирования, изучающее методы решения экстремальных задач, которые
характеризуются линейной зависимостью между переменными и линейным критерием.

Необходимым условием постановки задачи линейного программирования являются
ограничения на наличие ресурсов, величину спроса, производственную мощность
предприятия и другие производственные факторы.

Сущность линейного программирования состоит в нахождении точек наибольшего
или наименьшего значения некоторой функции при определенном наборе ограничений,
налагаемых на аргументы и образующих систему ограничений, которая имеет, как
правило, бесконечное множество решений. Каждая совокупность значений переменных
(аргументов функции $F$), которые удовлетворяют системе ограничений, называется
допустимым планом задачи линейного программирования. Функция $F$, максимум или
минимум которой определяется, называется целевой функцией задачи. Допустимый план,
на котором достигается максимум или минимум функции $F$, называется оптимальным
планом задачи.

Система ограничений, определяющая множество планов, диктуется условиями
производства. Задачей линейного программирования (ЗЛП) является выбор из множества
допустимых планов наиболее выгодного (оптимального).

В общей постановке задача линейного программирования выглядит следующим
образом:

Имеются какие-то переменные $х = (х_1,  х_2 , … х_n )$ и функция этих переменных $f(x) =
f (х_1, х_2, … х_n )$, которая носит название целевой функции. Ставится задача: найти
экстремум (максимум или минимум) целевой функции $f(x)$ при условии, что переменные x
принадлежат некоторой области $G$:

\begin{center}
$\begin{cases}
   f(x) \Rightarrow extr \\
  x \in G 
 \end{cases}$
 \end{center}
 
Линейное программирование характеризуется:
\begin{itemize}
\itemфункция $f(x)$ является линейной функцией переменных $х_1 , х_2 , … х_n$
\itemобласть $G$ определяется системой линейных равенств или неравенств.
\end{itemize}

Математическая модель любой задачи линейного программирования включает в
себя:
\begin{itemize}
\itemмаксимум или минимум целевой функции (критерий оптимальности);
\itemсистему ограничений в форме линейных уравнений и неравенств;
\itemтребование неотрицательности переменных.
\end{itemize}
\newpage
Наиболее общую форму задачи линейного программирования формулируют следующим образом:
\begin{equation}
 \begin{cases}
  a_{11}x_1 +a_{12}x_2+...+a_{1n}x_n \{\le,\ge,=\}b_1,\\ 
  a_{21}x_1 +a_{22}x_2+...+a_{2n}x_n \{\le,\ge,=\}b_2,\\
  ...\\
  a_{m1}x_1 +a_{m2}x_2+...+a_{mn}x_n \{\le,\ge,=\}b_m.\\
 \end{cases}
\end{equation}
\begin{equation}
x_1\ge0, x_2\ge0, ..., x_n\ge0
\end{equation}
\begin{equation}
F=c_1x_1+c_2x_2 + ... + c_nx_n \rightarrow max(min)
\end{equation}

Коэффициенты $a_{i,j} , b_i , c_j , j = 1, 2, ... , n, i =1, 2, ... , m$ – любые действительные числа
(возможно 0).

Решения, удовлетворяющие системе ограничений $(1)$ условий задачи и
требованиям неотрицательности $(2)$, называются допустимыми, а решения,
удовлетворяющие одновременно и требованиям минимизации (максимализации) $(3)$
целевой функции, - оптимальными.

Выше описанная задача линейного программирования (ЗЛП) представлена в общей
форме, но одна и та же (ЗЛП) может быть сформулирована в различных эквивалентных
формах. Наиболее важными формами задачи линейного программирования являются
каноническая и стандартная.

В канонической форме задача является задачей на максимум (минимум)
некоторой линейной функции $F$, ее система ограничений состоит только из равенств
(уравнений). При этом переменные задачи $х_1, х_2, ..., х_n$ являются неотрицательными:
\begin{equation}
 \begin{cases}
  a_{11}x_1 +a_{12}x_2+...+a_{1n}x_n =b_1,\\ 
  a_{21}x_1 +a_{22}x_2+...+a_{2n}x_n =b_2,\\
  ...\\
  a_{m1}x_1 +a_{m2}x_2+...+a_{mn}x_n=b_m.\\
 \end{cases}
\end{equation}
\begin{equation}
x_1\ge0, x_2\ge0, ..., x_n\ge0
\end{equation}
\begin{equation}
F=c_1x_1+c_2x_2 + ... + c_nx_n \rightarrow max(min)
\end{equation}

К канонической форме можно преобразовать любую задачу линейного
программирования.

В стандартной форме задача линейного программирования является задачей на
максимум (минимум) линейной целевой функции. Система ограничений ее состоит из
одних линейных неравенств типа $«\ge»$ или $«\le»$. Все переменные задачи
неотрицательны.
\begin{equation}
 \begin{cases}
  a_{11}x_1 +a_{12}x_2+...+a_{1n}x_n \ge b_1,\\ 
  a_{21}x_1 +a_{22}x_2+...+a_{2n}x_n \ge b_2,\\
  ...\\
  a_{m1}x_1 +a_{m2}x_2+...+a_{mn}x_n\ge b_m.\\
 \end{cases}
\end{equation}
\begin{equation}
x_1\ge0, x_2\ge0, ..., x_n\ge0
\end{equation}
\begin{equation}
F=c_1x_1+c_2x_2 + ... + c_nx_n \rightarrow max(min)
\end{equation}
Всякую задачу линейного программирования можно сформулировать в
стандартной форме.
\newpage
\section*{Решение заданий}
Привести к канонической форме следующие задачи линейного
программирования:
\begin{enumerate}
\item
\begin{center}
$\begin{cases}
  2x_1 -x_2+3x_3 \le 5,\\ 
  x_1 +2x_3 =8,\\
  -x_1 -2x_2\ge 1.\\
\end{cases}$

$x_1\ge0, x_2\ge0, x_3\ge0$

$F=x_1-x_2 +3x_3 \rightarrow min$

Также привести к стандартному виду.
\end{center}

\textbf{Решение}

\textit{Приведение к каноническому виду:}

Введем дополнительные переменные $x_4$, $ x_5$. Причем в первое неравенство
введем неотрицательную переменную $x_4$ со знаком плюс, а в третье – со
знаком минус переменную $x_5$, поменяв знак целевой функции,  запишем задачу в виде:
\begin{center}
$\begin{cases}
  2x_1 -x_2+3x_3  + x_4= 5,\\ 
  x_1 +2x_3 =8,\\
  -x_1 -2x_2-x_5= 1.\\
\end{cases}$

$x_j\ge0,j =1...5$

$F=-x_1+x_2 -3x_3 \rightarrow max$
\end{center}
что и дает эквивалентную задачу в канонической форме.

\textit{Приведение к стандартному виду:}

Второе уравнение заменим на два равносильных противоположных неравенства, а третье неравенство умножим на минус единицу. Поменяв знак целевой функции, запишем задачу в виде:
\begin{center}
$\begin{cases}
  2x_1 -x_2+3x_3 \le 5,\\ 
  x_1 +2x_3 \le8,\\
  -x_1 -2x_3 \le-8,\\
  x_1 +2x_2\le -1.\\
\end{cases}$

$x_1\ge0, x_2\ge0, x_3\ge0$

$F=-x_1+x_2 -3x_3 \rightarrow max$
\end{center}
что и дает эквивалентную задачу в стандартной форме.
\newpage
\item
\begin{center}
$\begin{cases}
  x_1 -2x_2+x_3 \ge 4,\\ 
  x_1 +x_2 - 3x_3 \le9,\\
  x_1 +3x_2 +2x_3= 10.\\
\end{cases}$

$x_1\ge0, x_2\ge0, x_3\ge0$

$F=2x_1+x_2 -x_3 \rightarrow max$

Также привести к стандартному виду.
\end{center}

\textbf{Решение}

\textit{Приведение к каноническому виду:}

Введем дополнительные переменные $x_4$, $ x_5$. Причем в первое неравенство
введем неотрицательную переменную $x_4$ со знаком минус, а во второе – со
знаком плюс переменную $x_5$ запишем задачу в виде:
\begin{center}
$\begin{cases}
  x_1 -2x_2+x_3 -x_4 = 4,\\ 
  x_1 +x_2 - 3x_3 +x_5 =9,\\
  x_1 +3x_2 +2x_3= 10.\\
\end{cases}$

$x_1\ge0, x_2\ge0, x_3\ge0$

$F=2x_1+x_2 -x_3 \rightarrow max$
\end{center}
что и дает эквивалентную задачу в канонической форме.

\textit{Приведение к стандартному виду:}

Третье уравнение заменим на два равносильных противоположных неравенства, а первое неравенство умножим на минус единицу и запишем задачу в виде:
\begin{center}
$\begin{cases}
  -x_1 +2x_2-x_3 \le -4,\\ 
  x_1 +x_2 - 3x_3 \le9,\\
  x_1 +3x_2 +2x_3\le 10,\\
  -x_1 -3x_2 -2x_3\le -10.\\
\end{cases}$

$x_1\ge0, x_2\ge0, x_3\ge0$

$F=2x_1+x_2 -x_3 \rightarrow max$
\end{center}
что и дает эквивалентную задачу в стандартной форме.
\newpage
\item

\begin{center}
$\begin{cases}
  x_1 +2x_2-x_3-2x_4 + x_5 = 5,\\ 
  -2x_2 +4x_3 + 4x_4 \le4.\\
\end{cases}$

$x_2\ge0, x_3\ge0, x_5\ge0$

$F=2x_1-x_2+ 3x_3 + x_4-2x_5\rightarrow min$
\end{center}

\textbf{Решение}

Введем дополнительную переменную $x_6$ во второе неравенство со знаком плюс, поменяем знак целевой функции и переменные $x_1$ и $x_4$ распишем через разность двух соответствующих неотрицельных переменных. Запишем задачу в виде:
\begin{center}
$\begin{cases}
  x_1\rq{} - x_1\rq{} \rq{} +2x_2-x_3-2x_4\rq{} + 2x_4\rq{} \rq{} + x_5 = 5,\\ 
  -2x_2 +4x_3 + 4x_4\rq{} - 4x_4\rq{}\rq{}  + x_6=4.\\
\end{cases}$

$x_1\rq{}\ge0,x_1\rq{}\rq{}\ge0,x_2\ge0, x_3\ge0,x_4\rq{}\ge0,x_4\rq{}\rq{}, x_5\ge0, x_6\ge0$

$F=-2x_1\rq{} + 2x_1\rq{} \rq{}+x_2- 3x_3 - x_4\rq{} +x_4\rq{}\rq{}+2x_5\rightarrow max$
\end{center}
что и дает эквивалентную задачу в канонической форме.


\item

\begin{center}
$\begin{cases}
  -x_1 +x_2+4x_3-2x_4\ge 6,\\ 
  x_1-2x_2+ 3x_3 + x_4+x_5 =2.\\
\end{cases}$

$x_1\ge0, x_3\ge0, x_4\ge0, x_5\ge0$

$F=x_1+2x_2+ 3x_3 + 2x_4+x_5\rightarrow max$
\end{center}

\textbf{Решение}

Введем дополнительную неотрицательную переменную $x_6$ в первое неравенство со знаком минус и переменную $x_2$ распишем через разность двух соответствующих неотрицельных переменных. Запишем задачу в виде:
\begin{center}
$\begin{cases}
  -x_1 +x_2\rq{} - x_2\rq{}\rq{}+4x_3-2x_4 - x_6 =  6,\\ 
  x_1-2x_2\rq{} +2 x_2\rq{}\rq{}+3x_3 + x_4+x_5 =2.\\
\end{cases}$

$x_1\ge0,x_2\rq{}\ge0, x_2\rq{}\rq{}\ge0, x_3\ge0, x_4\ge0, x_5\ge0, x_6\ge0$

$F=x_1+2x_2\rq{} - 2x_2\rq{}\rq{}+ 3x_3 + 2x_4+x_5\rightarrow max$\end{center}
что и дает эквивалентную задачу в канонической форме.
\newpage
\item
\begin{center}
$\begin{cases}
  2x_1 -x_2+6x_3 \le 12,\\ 
 3x_1 +5x_2 - 12x_3 = 14,\\
 -3x_1 +6x_2 +4x_3\le18.\\
\end{cases}$

$x_1\ge0, x_2\ge0, x_3\ge0$

$F=-2x_1-x_2 +x_3 \rightarrow min$

Также привести к стандартному виду.
\end{center}

\textbf{Решение}

\textit{Приведение к каноническому виду:}

Введем дополнительные неотрицательные переменные $x_4$ и $ x_5$ в первое и третье неравенство соответсвенно со знаком плюс, поменяем знак целевой функции и запишем задачу в виде:
\begin{center}
$\begin{cases}
  2x_1 -x_2+6x_3 + x_4 = 12,\\ 
 3x_1 +5x_2 - 12x_3 = 14,\\
 -3x_1 +6x_2 +4x_3 + x_5=18.\\
\end{cases}$

$x_1\ge0, x_2\ge0, x_3\ge0, x_4\ge0, x_5\ge0$

$F=2x_1+x_2 -x_3 \rightarrow max$
\end{center}
что и дает эквивалентную задачу в канонической форме.

\textit{Приведение к стандартному виду:}

Второе уравнение заменим на два равносильных противоположных неравенства, поменяем знак целевой функции и запишем задачу в виде:
\begin{center}

$\begin{cases}
  2x_1 -x_2+6x_3 \le 12,\\ 
 3x_1 +5x_2 - 12x_3 \le 14,\\
 -3x_1 -5x_2 + 12x_3 \le - 14,\\
 -3x_1 +6x_2 +4x_3 \le18.\\
\end{cases}$

$x_1\ge0, x_2\ge0, x_3\ge0$

$F=2x_1+x_2 -x_3 \rightarrow max$\end{center}
что и дает эквивалентную задачу в стандартной форме.

\newpage
\item
\begin{center}
$\begin{cases}
  4x_1 +2x_2+5x_3 \le 12,\\ 
 6x_1 -3x_2 +4x_3 = 18,\\
 3x_1 +3x_2 -2x_3\ge16.\\
\end{cases}$

$x_1\ge0, x_2\ge0, x_3\ge0$

$F=-2x_1+x_2 +5x_3 \rightarrow max$

Также привести к стандартному виду.
\end{center}

\textbf{Решение}

\textit{Приведение к каноническому виду:}

Введем дополнительные неотрицательные переменные $x_4$ и $ x_5$ в первое неравенство со знаком плюс и в третье со знаком минус соответственно. Запишем задачу в виде:
\begin{center}
$\begin{cases}
  4x_1 +2x_2+5x_3 +x_4 = 12,\\ 
 6x_1 -3x_2 +4x_3 = 18,\\
 3x_1 +3x_2 -2x_3-x_5=16.\\
\end{cases}$

$x_1\ge0, x_2\ge0, x_3\ge0, x_4\ge0, x_5\ge0$

$F=-2x_1+x_2 +5x_3 \rightarrow max$
\end{center}
что и дает эквивалентную задачу в канонической форме.

\textit{Приведение к стандартному виду:}

Второе уравнение заменим на два равносильных противоположных неравенства, третье неравенство умножим на минус единицу и запишем задачу в виде:
\begin{center}

$\begin{cases}
  4x_1 +2x_2+5x_3 \le 12,\\ 
 6x_1 -3x_2 +4x_3 \le 18,\\
 -6x_1 +3x_2 -4x_3 \le -18,\\
 -3x_1 -3x_2 +2x_3\le-16.\\
\end{cases}$

$x_1\ge0, x_2\ge0, x_3\ge0$

$F=-2x_1+x_2 +5x_3 \rightarrow max$
\end{center}
что и дает эквивалентную задачу в стандартной форме.
\newpage
\item
\begin{center}
$\begin{cases}
  -x_1 +x_2+x_3 \ge 4,\\ 
 2x_1 -x_2 +x_3 \le 16,\\
 3x_1 +x_2 +x_3\ge18.\\
\end{cases}$

$x_1\ge0, x_2\ge0, x_3\ge0$

$F=2x_1-5x_2- 3x_3 \rightarrow min$

Также привести к стандартному виду.
\end{center}

\textbf{Решение}

\textit{Приведение к каноническому виду:}

Введем дополнительные неотрицательные переменные $x_4$, $x_5$ и $ x_6$ в первое неравенство со знаком минус, во второе со знаком плюс  и в третье со знаком минус соответственно. Поменяем знак целевой функции и запишем задачу в виде:
\begin{center}
$\begin{cases}
  -x_1 +x_2+x_3 - x_4 = 4,\\ 
 2x_1 -x_2 +x_3 + x_5 = 16,\\
 3x_1 +x_2 +x_3 - x_6=18.\\
\end{cases}$

$x_1\ge0, x_2\ge0, x_3\ge0, x_4\ge0, x_5\ge0, x_6\ge0$

$F=-2x_1+5x_2+ 3x_3 \rightarrow max$
\end{center}
что и дает эквивалентную задачу в канонической форме.

\textit{Приведение к стандартному виду:}

Первое и третье неравенство умножим на минус единицу, поменяем знак целевой функции и запишем задачу в виде: и запишем задачу в виде:
\begin{center}

$\begin{cases}
  x_1 -x_2-x_3 \le  -4,\\ 
 2x_1 -x_2 +x_3 \le 16,\\
 -x_1 -x_2 -x_3\le-18.\\
\end{cases}$

$x_1\ge0, x_2\ge0, x_3\ge0$

$F=-2x_1+5x_2+ 3x_3 \rightarrow max$
\end{center}
что и дает эквивалентную задачу в стандартной форме.\newpage
\item
\begin{center}
$\begin{cases}
  -4x_1 +3x_2+8x_3 \ge 15,\\ 
 2x_1 +5x_2 -7x_3 \le 12,\\
 3x_1 -x_2 +10x_3\le17.\\
\end{cases}$

$x_1\ge0, x_2\ge0, x_3\ge0$

$F=-3x_1-5x_2- 6x_3 \rightarrow min$

Также привести к стандартному виду.
\end{center}

\textbf{Решение}

\textit{Приведение к каноническому виду:}

Введем дополнительные неотрицательные переменные $x_4$, $x_5$ и $ x_6$ в первое неравенство со знаком минус, во второе и третье со знаком плюс. Поменяем знак целевой функции и запишем задачу в виде:
\begin{center}
$\begin{cases}
  -4x_1 +3x_2+8x_3 - x_4 =15,\\ 
 2x_1 +5x_2 -7x_3 + x_5 = 12,\\
 3x_1 -x_2 +10x_3 +x_6=17.\\
\end{cases}$

$x_1\ge0, x_2\ge0, x_3\ge0, x_4\ge0, x_5\ge0, x_6\ge0$

$F=3x_1+5x_2+ 6x_3 \rightarrow max$
\end{center}
что и дает эквивалентную задачу в канонической форме.

\textit{Приведение к стандартному виду:}

Первое неравенство умножим на минус единицу, поменяем знак целевой функции и запишем задачу в виде: 
\begin{center}

$\begin{cases}
  4x_1 -3x_2-8x_3 \le -15,\\ 
 2x_1 +5x_2 -7x_3 \le 12,\\
 3x_1 -x_2 +10x_3\le17.\\
\end{cases}$

$x_1\ge0, x_2\ge0, x_3\ge0$

$F=3x_1+5x_2+ 6x_3 \rightarrow max$\end{center}
что и дает эквивалентную задачу в стандартной форме.
\newpage
\item
\begin{center}
$\begin{cases}
  2x_1 -x_2-x_3+x_4 \le 6,\\ 
 x_1 +2x_2+x_3-x_4 \ge 8,\\
 3x_1 -x_2 +2x_3+2x_4\le10,\\
  -x_1 +3x_2 +5x_3-3x_4=15.\\
\end{cases}$

$x_1\ge0, x_2\ge0, x_3\ge0, x_4\ge0$

$F=-x_1+2x_2- x_3+x_4  \rightarrow min$

Также привести к стандартному виду.
\end{center}

\textbf{Решение}

\textit{Приведение к каноническому виду:}

Введем дополнительные неотрицательные переменные $x_5$, $x_6$ и $ x_7$ в первое неравенство со знаком плюс, во второесл знаком минус и в третье со знаком плюс. Поменяем знак целевой функции и запишем задачу в виде:
\begin{center}
$\begin{cases}
  2x_1 -x_2-x_3+x_4+x_5 = 6,\\ 
 x_1 +2x_2+x_3-x_4 -x_6= 8,\\
 3x_1 -x_2 +2x_3+2x_4+x_7=10,\\
  -x_1 +3x_2 +5x_3-3x_4=15.\\
\end{cases}$

$x_1\ge0, x_2\ge0, x_3\ge0, x_4\ge0, x_5\ge0, x_6\ge0, x_7\ge0$

$F=x_1-2x_2+ x_3-x_4  \rightarrow max$
\end{center}
что и дает эквивалентную задачу в канонической форме.

\textit{Приведение к стандартному виду:}

Второе неравенство умножим на минус единицу, четвертое уравнение заменим на два равносильных противоположных неравенства, поменяем знак целевой функции и запишем задачу в виде: 
\begin{center}

$\begin{cases}
  
  2x_1 -x_2-x_3+x_4 \le 6,\\ 
 -x_1 -2x_2-x_3+x_4\le -8,\\
 3x_1 -x_2 +2x_3+2x_4\le10,\\
  -x_1 +3x_2 +5x_3-3x_4\le15.\\
x_1 -3x_2 -5x_3+3x_4\le-15.\\

\end{cases}$

$x_1\ge0, x_2\ge0, x_3\ge0, x_4\ge0$

$F=x_1-2x_2+ x_3-x_4  \rightarrow max$
\end{center}
что и дает эквивалентную задачу в стандартной форме.
\newpage
\item
\begin{center}
$\begin{cases}
  2x_1 +x_3+x_4+x_5 \le 2,\\ 
 x_1 -x_3+2x_4+x_5 \le 3\\
 x_3 -x_4 +2x_3\le6,\\
  x_1 -x_2 +x_4-5x_5\ge8.\\
\end{cases}$

$x_1\ge0, x_2\ge0, x_3\ge0, x_4\ge0, x_5\ge0$

$F=3x_1-2x_2- 5x_4+x_5  \rightarrow max$


\end{center}

\textbf{Решение}


Введем дополнительные неотрицательные переменные $x_6$, $x_7$, $x_8$ и $ x_9$ в первое, второе и третье неравенство со знаком плюс и в четвертое со знаком минус. Запишем задачу в виде:
\begin{center}
$\begin{cases}
  2x_1 +x_3+x_4+x_5 + x_6= 2,\\ 
 x_1 -x_3+2x_4+x_5 + x_7=3\\
 x_3 -x_4 +2x_3+ x_8=6,\\
  x_1 -x_2 +x_4-5x_5- x_9=8.\\
\end{cases}$

$x_1\ge0, x_2\ge0, x_3\ge0, x_4\ge0, x_5\ge0, x_6\ge0, x_7\ge0, x_8\ge0, x_9\ge0$

$F=3x_1-2x_2- 5x_4+x_5  \rightarrow max$

\end{center}
что и дает эквивалентную задачу в канонической форме.

\end{enumerate}
\end{document}