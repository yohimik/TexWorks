\documentclass[12pt]{article}
\usepackage[russian]{babel}
\usepackage{hhline}
\usepackage{graphicx}
\graphicspath{{pictures/}}
\DeclareGraphicsExtensions{.png}
\usepackage{multirow}
\usepackage{amsmath}
\usepackage{mathtext}
\usepackage[T2A]{fontenc}
\usepackage[utf8]{inputenc}
\usepackage{pscyr} 
\usepackage[left=1.5cm,right=1.5cm, top=1cm,bottom=1cm,bindingoffset=0cm]{geometry}
\begin{document}

\pagestyle{empty}
\begin{center}
\large{\textbf{Университет ИТМО}}
\end{center}
\rule{525pt}{1pt}
\par\bigskip\par\bigskip\par\bigskip\par\bigskip\par\bigskip\par\bigskip\par\bigskip\par\bigskip
\begin{center}
\Large
\textbf{Лабораторная работа №4.1}

\textbf{\textit{«Оптимальность плана транспортной задачи.»}}


\end{center}
\par\bigskip\par\bigskip\par\bigskip\par\bigskip\par\bigskip\par\bigskip\par\bigskip\par\bigskip\par\bigskip\par\bigskip\par\bigskip\par\bigskip\par\bigskip\par\bigskip      
\begin{flushright}
\large
Выполнил: Федюкович С. А.
\par\bigskip
Факультет: МТУ “Академия ЛИМТУ”
\par\bigskip
Группа: S3100                       
\par\bigskip\par\bigskip\par\bigskip

\rule{150pt}{0.5pt}
\par\bigskip\par\bigskip\par\bigskip\par\bigskip                                                            
 Проверила: Авксентьева Е. Ю.
\par\bigskip \par\bigskip

\rule{150pt}{0.5pt}
\end{flushright}
\par\bigskip\par\bigskip\par\bigskip\par\bigskip\par\bigskip\par\bigskip\par\bigskip\par\bigskip\par\bigskip\par\bigskip     
\begin{center}
\large
Санкт-Петербург
\par\bigskip
2018
\end{center}
\newpage
\section*{Теоретические основы лабораторной работы}

\newpage
\section*{Решение заданий}

Составить опорный план (любым из методов опорного плана), проверить его на
оптимальность и множественность: 

\subsection*{Задача 1}

\begin{table}[h!]
\begin{center}
\begin{tabular}{|c|c|c|c|c|c|}
\hline
           & $B_1$ & $B_2$ & $B_3$ & $B_4$ & $A_i$	\\
\hline
 $A_1$ & 2 & 3 & 2 & 4 & 30	\\
\hline
 $A_2$ & 3 & 2 & 5 & 1 & 40	\\
\hline
 $A_3$ & 4 & 3 & 2 & 6 & 20	\\ 
\hline
 $B_j$ & 20 & 30 & 30 & 10 & 90	\\
\hline
\end{tabular}
\end{center}
\end{table} 
\subsection*{Решение}

\subsection*{Задача 2}

\begin{table}[h!]
\begin{center}
\begin{tabular}{|c|c|c|c|c|c|c|}
\hline
           & $B_1$ & $B_2$ & $B_3$ & $B_4$ & $B_5$ & $A_i$	\\
\hline
 $A_1$ & 2 & 7 & 3 & 6 & 2 & 30	\\
\hline
 $A_2$ & 9 & 4 & 5 & 7 & 3 & 70	\\
\hline
 $A_3$ & 5 & 7 & 6 & 2 & 4 & 50	\\ 
\hline
 $B_j$ & 10 & 40 & 20 & 60 & 20 & 150	\\
\hline
\end{tabular}
\end{center}
\end{table} 
\subsection*{Решение}

\subsection*{Задача 3}

\begin{table}[h!]
\begin{center}
\begin{tabular}{|c|c|c|c|c|c|c|}
\hline
           & $B_1$ & $B_2$ & $B_3$ & $B_4$ & $B_5$ & $A_i$	\\
\hline
 $A_1$ & 2 & 7 & 3 & 6 & 2 & 30	\\
\hline
 $A_2$ & 9 & 4 & 5 & 7 & 3 & 70	\\
\hline
 $A_3$ & 5 & 7 & 6 & 2 & 4 & 50	\\ 
\hline
 $B_j$ & 10 & 40 & 20 & 60 & 20 & 150	\\
\hline
\end{tabular}
\end{center}
\end{table} 
\subsection*{Решение}
\end{document}