\documentclass[12pt]{article}
\usepackage{hhline}
\usepackage{graphicx}
\usepackage{pgfplots}
\usepgfplotslibrary{fillbetween}
\usetikzlibrary{patterns}
\pgfplotsset{width=7cm,compat=1.15}
\graphicspath{{pictures/}}
\DeclareGraphicsExtensions{.png}
\usepackage{multirow}
\usepackage{amsmath}
\usepackage{mathtext}
\usepackage[T2A]{fontenc}
\usepackage[utf8]{inputenc}
\usepackage{pscyr} 
\usepackage[left=2.5cm,right=2.5cm, top=1.5cm,bottom=1cm,bindingoffset=0cm]{geometry}
\begin{document}

\pagestyle{empty}
\begin{center}
\large{\textbf{Университет ИТМО}}
\end{center}
\rule{500pt}{1pt}
\par\bigskip\par\bigskip\par\bigskip\par\bigskip\par\bigskip\par\bigskip\par\bigskip\par\bigskip
\begin{center}
\Large
\textbf{Лабораторная работа №2}

\textbf{\textit{«Геометрическое истолкование задачи линейного программирования в стандартной форме в случае двух переменных»}}


\end{center}
\par\bigskip\par\bigskip\par\bigskip\par\bigskip\par\bigskip\par\bigskip\par\bigskip\par\bigskip\par\bigskip\par\bigskip\par\bigskip\par\bigskip\par\bigskip\par\bigskip      
\begin{flushright}
\large
Выполнил: Федюкович С. А.
\par\bigskip
Факультет: МТУ “Академия ЛИМТУ”
\par\bigskip
Группа: S3100                       
\par\bigskip\par\bigskip\par\bigskip

\rule{150pt}{0.5pt}
\par\bigskip\par\bigskip\par\bigskip\par\bigskip                                                            
 Проверила: Авксентьева Е. Ю.
\par\bigskip \par\bigskip

\rule{150pt}{0.5pt}
\end{flushright}
\par\bigskip\par\bigskip\par\bigskip\par\bigskip\par\bigskip\par\bigskip\par\bigskip\par\bigskip\par\bigskip\par\bigskip     
\begin{center}
\large
Санкт-Петербург
\par\bigskip
2018
\end{center}
\newpage
\section*{Теоретические основы лабораторной работы}

Задача линейного программирования в стандартной форме с двумя переменными имеет вид:
\begin{center}
$ \begin{cases}$
 $ a_{11}x_1 +a_{12}x_2 \ge b_1,\\ $
$  a_{21}x_1 +a_{22}x_2 \ge b_2,\\$
$  ...\\$
 $ a_{m1}x_1 +a_{m2}x_2 \ge b_m.$
$ \end{cases}$

$x_1\ge0, x_2\ge0$

$F=c_1x_1+c_2x_2 \rightarrow max(min)$
\end{center}

Введя систему координат,  изобразим каждую совокупность значений переменных $x_1$ и $x_2$ точкой на плоскости, откладывая по одной оси $x_1$, а по другой - $x_2$. Тогда совокупность решений  решений одного отдельно взятого неравенства будет являться полуплоскостью, а системы нескольких неравенств --- выпуклой многоугольной областью. Условия неотрицательности переменных $x_1 \ge 0$ и $x_2 \ge 0$ приводят к тому,  что эта область находится в первой координатной области.

\section*{Решение заданий}
\begin{enumerate}

\item На звероферме могут выращиваться лисицы и песцы. Для обеспечения
нормальных условий их выращивания используется три вида кормов. Количество кормов
каждого вида, которое должны получать животные, приведено в таблице 1. В ней также
указаны общее количество корма каждого вида, которое может быть использовано
зверофермой ежедневно, и прибыль от реализации одной шкурки лисицы и песца.
Определить, сколько лисиц и песцов можно вырастить при имеющихся запасах корма. Решить графически.
\begin{table}[h!]
\begin{center}
Таблица 1 

\begin{tabular}{|c|c|c|c|}
\hline
\multirow{2}{70pt}{Вид корма} 	&	\multicolumn{2}{|c|}{Количество единиц корма в день}	&	\multirow{2}{70pt}{Запас корма}	\\
\hhline{~--~}
			&	Лисица		&		Песец				&				\\
\hline
А			&	2			&		2					&	180		   	\\
\hline
Б			&	4			&		1					&	240		   	\\
\hline
В			&	6			&		7					&	426		   	\\
\hline
Цена одной шкурки, руб	&	1600		&	1200		&\\
\hline
\end{tabular}
\end{center}
\end{table} 

\textbf{Решение}

Пусть $x_1$ --- количество лисиц, а $x_2$ --- количество песцов. Составим модель задачи:
\begin{center}
$\begin{cases}$
$  2x_1+2x_2 \le180,\\ $
$  4x_1 +x_2 \le 240,\\$
$  6x_1 +7x_2 \le 426.$
 $\end{cases}$

$x_1\ge0, x_2\ge0$\\
$F=1600x_1+1200x_2 \rightarrow max$\\
\end{center}
\newpage
Поделим обе части первого неравенства на два:
\begin{center}
$\begin{cases}$
$  x_1+x_2 \le90,\\ $
$  4x_1 +x_2 \le 240,\\$
$  6x_1 +7x_2 \le 426.$
 $\end{cases}$

$x_1\ge0, x_2\ge0$\\
$F=1600x_1+1200x_2 \rightarrow max$\\
\end{center}
Выразим $x_2$ из каждого неравенства системы:
\begin{center}
$\begin{cases}$
$  x_2 \le90-x_1,\\ $
$   x_2 \le 240- 4x_1,\\$
$  x_2 \le \frac{426-6x_1}{7}.$
 $\end{cases}$
 
 $x_1\ge0, x_2\ge0$\\
$F=1600x_1+1200x_2 \rightarrow max$\\
 \end{center}

Найдём решение системы, выделив область решений каждого неравенства:
\begin{center}
\begin{tikzpicture}
\begin{axis}[		
	xlabel = {$x_1$},
	ylabel = {$x_2$},	
	xmin = 0,
	ymin = 0,
	xmax = 90,
	ymax = 100,	
	axis x line=center,
	axis y line=center,
	width = 400,
	height = 200,
	domain = 0:1000	
]
\legend{ 
	Ограничение корма А, 
	Ограничение корма Б, 
	Ограничение корма В,
	Оптимальное решение
};
\addplot [green,fill=green,fill opacity=0.4,]  {90-x}
|- (0,0) -- cycle;

\addplot [brown,fill=brown,fill opacity=0.4,]  {240-4*x}
|- (0,0) -- cycle;

\addplot [blue,fill=blue,fill opacity=0.4,]  {426/7-6/7*x}
|- (0,0) -- cycle;

\addplot+ [red] coordinates{(57,12)};

\end{axis}
\end{tikzpicture}
\end{center}
Решением всей системы будет тёмно-синия область на графике, а оптимальное решение достигается в точке пересечения ограничений Б и В (57,12).

\textbf{Ответ:} звероферме нужно содержать 57 лисиц и 12 песцов
\newpage
\item При подкормке посевов необходимо внести на $0,01га.$ почвы не менее $8ед.$ азота, не менее $24ед.$ фосфора и не менее 16 единиц калия. Фермер закупает комбинированные удобрения двух видов \lq{}Азофоска\rq{} и \lq{}Комплекс\rq{}. В таблице 2 указано содержание количества единиц химического вещества в $1кг.$ каждого вида удобрений и цена $1кг.$ удобрений. Определить графически потребность фермера в удобрениях того и другого вида на $0,01га.$ посевной площади при минимальных затратах на потребление.
\begin{table}[h!]
\begin{center}
Таблица 2 

\begin{tabular}{|c|c|c|}
\hline
\multirow{2}{70pt}{Химическое вещество} 	&	\multicolumn{2}{|c|}{Содержание химических веществ в $1кг.$ удобрения}	\\
\hhline{~--}
			&	Азофоска		&		Комплекс	\\
\hline
Азот			&	1			&		2		\\
\hline
Фосфор		&	12			&		3		\\
\hline
Калий 		&	4			&		4		\\
\hline
Цена $1кг.$ удобрения, руб.	&	50		&	20	\\
\hline
\end{tabular}
\end{center}
\end{table} 

\textbf{Решение}

Пусть $x_1$ --- количество килограмм \lq{}Азофоски\rq{}, а $x_2$ --- \lq{}Комплекса\rq{}. Составим модель задачи:
\begin{center}
$\begin{cases}$
$  x_1+2x_2 \ge8,\\ $
$  12x_1 +3x_2 \ge 24,\\$
$  4x_1 +4x_2 \ge 16.$
 $\end{cases}$

$x_1\ge0, x_2\ge0$\\
$F=50x_1+20x_2 \rightarrow min$\\
\end{center}
Поделим обе части второго неравенства на три, а третьего на четыре:
\begin{center}
$\begin{cases}$
$  x_1+2x_2 \ge8,\\ $
$  4x_1 +x_2 \ge 8,\\$
$  x_1 +x_2 \ge 4.$
 $\end{cases}$

$x_1\ge0, x_2\ge0$\\
$F=50x_1+20x_2 \rightarrow min$\\
\end{center}
Выразим $x_2$ из каждого неравенства системы:
\begin{center}
$\begin{cases}$
$  x_2 \ge \frac{8-x_1}{2},\\ $
$  x_2 \ge 8-4x_1,\\$
$  x_2 \ge 4 -x_1.$
 $\end{cases}$

$x_1\ge0, x_2\ge0$\\
$F=50x_1+20x_2 \rightarrow min$\\
\end{center}
\newpage
Найдём решение системы, выделив область решений каждого неравенства:
\begin{center}
\begin{tikzpicture}
\begin{axis}[		
	xlabel = {$x_1$},
	ylabel = {$x_2$},	
	xmin = 0,
	ymin = 0,
	xmax = 8,
	ymax = 8,	
	axis x line=center,
	axis y line=center,
	width = 400,
	height = 200,
	domain = 0:100	
]
\legend{ 
	Ограничение азота, 
	Ограничение фосфора, 
	Ограничение калия,
	Оптимальное решение
};
\addplot [green, name path=B]  {(8-x)/2};


\addplot [brown, name path=C]  {8-4*x};

\addplot [blue, name path=D]  {4-x};


\addplot+ [red] coordinates{(8/7,24/7)};
\addplot [white,name path = A] coordinates{(0,8) (8,8)};
\addplot [green,fill opacity=0.4] fill between [of=A and B,soft clip={domain=0:8}];
\addplot [brown,fill opacity=0.4] fill between [of=A and C,soft clip={domain=0:8}];
\addplot [blue,fill opacity=0.4] fill between [of=A and D,soft clip={domain=0:8}];
\end{axis}
\end{tikzpicture}
\end{center}
Решением всей системы будет тёмно-синия область на графике, а оптимальное решение достигается в точке пересечения ограничений азота и фосфора ($\frac{8}{7},\frac{24}{7}$).

\textbf{Ответ:} фермеру потребуется $\frac{8}{7}кг.$ \lq{}Азофоски\rq{}  и $\frac{24}{7}кг.$ \lq{}Комплекса\rq{}.
\newpage
\item Полной даме необходимо похудеть, а за помощью она обратилась к подруге. Подруга посоветовала перейти на рациональное питание, состоящее из двух продуктов $P$ и $Q$. 

Суточное питание этими продуктами должно давать менее $14ед.$ жира (чтобы похудеть), но не менее $300$ килокалорий. На упаковке продукта $Р$ написано, что в одном килограмме этого продукта содержится $15ед.$ жира и $150$ килокалорий, а на упаковке с продуктом $Q$ --- $4$ единицы жира и $200$ килокалорий соответственно. При этом цена продукта $Р$ равна $250 руб./кг$, а цена продукта $Q$ равна $210 руб./кг$.

Так как дама была стеснена в средствах, то ее интересовал вопрос: в какой пропорции нужно брать эти продукты для того, чтобы выдержать условия диеты и истратить как можно меньше денег? Решите задачу графически. 

\textbf{Решение}

Пусть $x_1$ --- количество продукта $P$, а $x_2$ --- $Q$. Составим модель задачи:
\begin{center}
$\begin{cases}$
$  15x_1+4x_2 < 14,\\ $
$  150x_1 +200x_2 \ge 300.$
$\end{cases}$

$x_1\ge0, x_2\ge0$\\
$F=250x_1+210x_2 \rightarrow min$\\
\end{center}
Обе части второго неравенства поделим на 50:
\begin{center}
$\begin{cases}$
$  15x_1+4x_2 < 14,\\ $
$  3x_1 +4x_2 \ge 6.$
$\end{cases}$

$x_1\ge0, x_2\ge0$\\
$F=250x_1+210x_2 \rightarrow min$\\
\end{center}
Выразим $x_2$ из каждого неравенства системы:
\begin{center}
$\begin{cases}$
$  x_2 < \frac{14-15x_1}{4},\\ $
$  x_2 \ge \frac{6-3x_1}{4}.$
$\end{cases}$

$x_1\ge0, x_2\ge0$\\
$F=250x_1+210x_2 \rightarrow min$\\
\end{center}
\newpage
Найдём решение системы, выделив область решений каждого неравенства:
\begin{center}
\begin{tikzpicture}
\begin{axis}[		
	xlabel = {$x_1$},
	ylabel = {$x_2$},	
	xmin = 0,
	ymin = 0,
	xmax = 1,
	ymax = 4,	
	axis x line=center,
	axis y line=center,
	width = 400,
	height = 200,
	domain = 0:10	
]
\legend{ 
	Ограничение жира, 
	Ограничение килокалорий, 	
	Оптимальное решение
};
\addplot [yellow,fill=yellow,fill opacity=0.4,]  {(14-15*x)/4}
|- (0,0) -- cycle;

\addplot [blue, name path=C]  {(6-3*x)/4};

\addplot+ [red] coordinates{(0,3/2)};
\addplot [white,name path = A] coordinates{(0,4) (1,4)};

\addplot [blue,fill opacity=0.4] fill between [of=A and C,soft clip={domain=0:8}];

\end{axis}
\end{tikzpicture}
\end{center}
Решением всей системы будет жёлто-синия область на графике, а оптимальное решение достигается в точке пересечения ограничений килокалорий и оси $x_2$ ($0,\frac{3}{2}$).

\textbf{Ответ:} даме потребуется $0кг.$ продукта $P$ и $\frac{3}{2}кг.$ продукта $Q$.
\end{enumerate}
\end{document}