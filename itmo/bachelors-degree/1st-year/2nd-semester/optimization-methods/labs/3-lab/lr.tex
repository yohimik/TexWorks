\documentclass[12pt]{article}
\usepackage[russian]{babel}
\usepackage{hhline}
\usepackage{graphicx}
\graphicspath{{pictures/}}
\DeclareGraphicsExtensions{.png}
\usepackage{multirow}
\usepackage{amsmath}
\usepackage{mathtext}
\usepackage[T2A]{fontenc}
\usepackage[utf8]{inputenc}
\usepackage{pscyr} 
\usepackage[left=1.5cm,right=1.5cm, top=1cm,bottom=1cm,bindingoffset=0cm]{geometry}
\begin{document}

\pagestyle{empty}
\begin{center}
\large{\textbf{Университет ИТМО}}
\end{center}
\rule{525pt}{1pt}
\par\bigskip\par\bigskip\par\bigskip\par\bigskip\par\bigskip\par\bigskip\par\bigskip\par\bigskip
\begin{center}
\Large
\textbf{Лабораторная работа №3}

\textbf{\textit{«Симплекс-метод линейного программирования»}}


\end{center}
\par\bigskip\par\bigskip\par\bigskip\par\bigskip\par\bigskip\par\bigskip\par\bigskip\par\bigskip\par\bigskip\par\bigskip\par\bigskip\par\bigskip\par\bigskip\par\bigskip      
\begin{flushright}
\large
Выполнил: Федюкович С. А.
\par\bigskip
Факультет: МТУ “Академия ЛИМТУ”
\par\bigskip
Группа: S3100                       
\par\bigskip\par\bigskip\par\bigskip

\rule{150pt}{0.5pt}
\par\bigskip\par\bigskip\par\bigskip\par\bigskip                                                            
 Проверила: Авксентьева Е. Ю.
\par\bigskip \par\bigskip

\rule{150pt}{0.5pt}
\end{flushright}
\par\bigskip\par\bigskip\par\bigskip\par\bigskip\par\bigskip\par\bigskip\par\bigskip\par\bigskip\par\bigskip\par\bigskip     
\begin{center}
\large
Санкт-Петербург
\par\bigskip
2018
\end{center}
\newpage
\section*{Теоретические основы лабораторной работы}
Симплекс-метод является основным в линейном программировании. Решение
задачи начинается с рассмотрений одной из вершин многогранника условий (задаваемого системой условий задачи). Если
исследуемая вершина не соответствует максимуму (минимуму), то переходят к соседней,
увеличивая значение функции цели при решении задачи на максимум и уменьшая при
решении задачи на минимум. Таким образом, переход от одной вершины к другой
улучшает значение функции цели. Так как число вершин многогранника ограничено, то за
конечное число шагов гарантируется нахождение оптимального значения или
установление того факта, что задача неразрешима.

Симплекс-метод основан на теореме, которая называется фундаментальной
теоремой симплекс-метода: если среди оптимальных планов задачи линейного
программирования в канонической форме есть хотя бы одно решение системы ограничений, то хотя бы одно из них является базисным, а их количество ограничено.

Симплекс-метод вносит определенный порядок как при нахождении
базисного решения, так и при переходе к другим базисным
решениям. Его идея состоит в следующем: находится любое базисное решение, если оно является допустимым, то оно же проверяется на оптимальности. Если оно не оптимально, то осуществляется переход
к другому, обязательно допустимому базисному решению.

Симплекс-метод гарантирует, что при этом новом решении линейная форма,
если и не достигнет оптимума, то приблизится к нему. С новым допустимым базисным
решением поступают так же, пока не находят решение, которое является оптимальным.

Если первое найденное базисное решение окажется недопустимым, то с помощью
симплекс-метода осуществляется переход к другим базисным решениям, которые
приближают нас к области допустимых решений, пока на каком-то шаге решения либо
базисное решение окажется допустимым и к нему применяют алгоритм симплекс-метода, либо мы убеждаемся в противоречивости системы ограничений.

Таким образом, применение симплекс-метода распадается на два этапа:
нахождение допустимого базисного решения системы ограничений или установление
факта ее несовместности и нахождение оптимального решения.

\newpage
\section*{Решение задания}
\subsection*{Задача}
Для изготовления $n$ видов изделий $И_1, И_2, ..., И_n$ необходимы ресурсы $m$ видов:
трудовые, материальные, финансовые и др. Известно необходимое количество отдельного
$i-гo$ ресурса для изготовления каждого $j-гo$ изделия. Назовем эту величину нормой
расхода. Пусть определено количество каждого вида ресурса, которым предприятие
располагает в данный момент. Известна прибыль $П_j$, получаемая предприятием от
изготовления каждого $j-гo$ изделия. Требуется определить, какие изделия и в каком
количестве должно изготавливать предприятие, чтобы обеспечить получение
максимальной прибыли. Необходимая исходная информация представлена в таблице:
\begin{table}[h!]
\begin{center}
\begin{tabular}{|c|c|c|c|c|c|}
\hline
\multirow{2}{70pt}{Используемые ресурсы} 	&	\multicolumn{4}{|c|}{\makebox{Изготавливаемые изделия}}	&	\multirow{2}{70pt}{Наличие ресурсов}	\\
\hhline{~----~}
			& \makebox{	$И_1$}		&	\makebox{	$И_2$	}	& 	\makebox{$И_3$}		&\makebox{$И_4$	}	& \\
\hline
Трудовые			&	3			&		5					&	2&7&15   	\\
\hline
Материальные			&	4			&		3					&	3&5&9   	\\
\hline
Финансовые			&	5			&		6					&	4&8&30	\\
\hline
Прибыль $П_j$	&	40			&		50					&30&20&		   \\
\hline
\end{tabular}
\end{center}
\end{table} 
\subsection*{Решение}
Составим математическую модель задачи: 
Через $x_1, x_2, x_3, x_4$ обозначим соответствующее  количество изделий $И_1, И_2, И_3, И_4$. Тогда задача будет заключаться в поиске максимума функции: 
\begin{center}
$F=40x_1+50x_2 +30x_3 +20x_4\rightarrow max$
\end{center}
При выполении следующих ограничений:
\begin{center}
$\begin{cases}
  3x_1 +5x_2+2x_3  +7x_4\le 15,\\ 
  4x_1 +3x_2+3x_3  +5x_4\le 9,\\ 
  5x_1 +6x_2+4x_3  +8x_4\le 30.\\ 
\end{cases}$\\
$x_j\ge0,j =1...4$
\end{center}
Обратим систему неравенств в систему уравнений, прибавив к каждой левой части неравенств добавочные неотрицательные переменные: $x_5, x_6, x_7$. В условиях данной задачи переменные будут содержать остаток сырья каждого вида после выполнения плана:
\begin{center}
$\begin{cases}
  3x_1 +5x_2+2x_3  +7x_4 + x_5 = 15,\\ 
  4x_1 +3x_2+3x_3  +5x_4+ x_6 = 9,\\ 
  5x_1 +6x_2+4x_3  +8x_4+ x_7 =30.\\ 
\end{cases}$\\
$x_j\ge0,j =1...7$
\end{center}
\newpage
Найдём любое базисное решение, приняв основными переменные $x_5, x_6, x_7$, и приравняв переменные $x_1, x_2, x_3, x_4$ к нулю. Тогда получим решение $(0; 0; 0; 0; 15; 9; 30)$, являющееся допустимым. Составим первую симплекс таблицу:
\begin{table}[h!]
\begin{center}
\begin{tabular}{|c|c|c|c|c|c|c|c|c|}
\hline
Базисные переменные & Свободные члены & $x_5$ & $x_6$ & $x_7$ & $x_1$ & $x_2$ & $x_3$  & $x_4$ \\
\hline
		$x_5$	  & 		15 		 & 	1    &     0    &     0    &     3    &     5    &     2    &     7     \\
\hline
		$x_6$	  & 		9 		 & 	0    &     1    &     0    &     4    &     3    &     3    &     5     \\
\hline
		$x_7$	  & 		30		 & 	0    &     0    &     1    &     5    &     6    &     4    &     8    \\
\hline
	         $F$		  & 		0		 & 	0    &     0    &     0    &   -40   &   -50   &   -30   &   -20    \\
\hline
\end{tabular}
\end{center}
\end{table} 

$X_1 = min\left\{ \frac{30}{6};\frac{9}{3};\frac{15}{3} \right\} = min\left\{ 5; 3; 5 \right\} = 3$
\begin{table}[h!]
\begin{center}
\begin{tabular}{|c|c|c|c|c|c|c|c|c|}
\hline
Базисные переменные & Свободные члены & $x_5$ & $x_6$ & $x_7$ & $x_1$ & $x_2$ & $x_3$  & $x_4$ \\
\hline
		$x_5$	  & 		0 		 & 	1    &  -5/3   &     0    & -11/3  &     0    &    -3    &  -4/3     \\
\hline
		$x_2$	  & 		3 		 & 	0    &   1/3   &     0    &   4/3   &     1    &     1    &   5/3     \\
\hline
		$x_7$	  & 		12		 & 	0    &    -2    &    1    &     -3    &     0    &    -2    &    -2    \\
\hline
	         $F$		  & 		150		 & 	0    &    50/3    &    0    &   80/3   &     0    &   20   &  190/3    \\
\hline
\end{tabular}
\end{center}
\end{table} 

Последняя строка таблицы не содержит отрицательных элементов, а значит, что мы нашли оптимальное решение $(0; 3; 0; 0; 0; 0; 12)$.

Ответ: требуется произвести $И_3$ в количестве трёх штук, заработав на продаже которых мы получим 150.
\end{document}