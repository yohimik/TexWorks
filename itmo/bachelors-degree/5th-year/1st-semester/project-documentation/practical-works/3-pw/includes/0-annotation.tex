\section*{\centering{АННОТАЦИЯ}}
\addcontentsline{toc}{section}{Аннотация}

Техническое задание -- это основной документ, оговаривающий набор требований и порядок создания программного продукта,
в соответствии с которым производится разработка программы, её тестирование и приёмка.

Настоящее Техническое задание на разработку по <<Автоматизации управления жизненным циклом веб-сервиса>> содержит следующие разделы:
<<Введение>>, <<Основания для разработки>>, <<Назначение разработки>>, <<Требования к программе>>, <<Требования к программным документам>>,
<<Технико-экономические показатели>>, <<Стадии и этапы разработки>>, <<Порядок контроля и приёмки>> и приложения.

В разделе <<Введение>> указано наименование и краткая характеристика области применения программы.

В разделе <<Основания для разработки>> указан документ, на основании которого ведётся разработка и наименование темы разработки.

В разделе <<Назначение разработки>> указано функциональное и эксплуатационное назначение программного продукта.

Раздел <<Требования к программе>> содержит основные требования к функциональным характеристикам, к надёжности, к условиям
эксплуатации, к составу и параметрам технических средств, к информационной и программной совместимости, к маркировке и
упаковке, к транспортировке и хранению, а также специальные требования.

Раздел <<Требования к программным документам>> содержит предварительный состав программной документации и специальные требования к ней.

Раздел <<Технико-экономические показатели>> содержит ориентировочную экономическую эффективность,
предполагаемую годовую потребность, экономические преимущества разработки программы.

Раздел <<Стадии и этапы разработки>> содержит стадии разработки, этапы и содержание работ.

В разделе <<Порядок контроля и приёмки>> указаны общие требования к приёмке работы.

Настоящий документ разработан в соответствии с требованиями:
\begin{enumerate}
    \item ГОСТ 19.103-77 Обозначения программ и программных документов;
    \item ГОСТ 19.104-78 Основные надписи;
    \item ГОСТ 19.105-78 Общие требования к программным документам;
    \item ГОСТ 19.106-78 Требования к программным документам, выполненным печатным способом;
    \item ГОСТ 19.201-78 Техническое задание.
    Требования к содержанию и оформлению.
\end{enumerate}

Изменения к Техническому заданию оформляются согласно ГОСТ 19.603-78, ГОСТ 19.604-78.

Перед прочтением данного документа рекомендуется ознакомиться с терминологией, приведённой в Приложении 1 настоящего
Технического задания.