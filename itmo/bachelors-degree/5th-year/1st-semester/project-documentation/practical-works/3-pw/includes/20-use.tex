\section{Назначение разработки}

\subsection{Функциональное назначение}

Функциональным назначением комплекса является предоставление пользователям возможности управлять жизненным циклом веб-сервиса.
Пользователи могут запускать новые веб-сервисы на сервере, управлять версиями и масштабированием.

Запуск сервиса пользователем производится путём создания удалённого репозитория, описания исходного кода и загрузки Dockerfile.
Для последующей развёртки будут использоваться технологии непрерывной интеграции (Continuous Integration), непрерывного развертывания (Continuous Delivery) и инфраструктуры, как кода (Infrastructure-as-Code).

Основная идея заключается в хранении всех настроек окружения в удалённых репозиториях GitLab.
Особенность данного git сервиса заключается в наличии всех необходимых функций и документаций бесплатно: CI/CD, Docker Container Registry, Package registry и т. д.
Скорость работы комплекса обуславливается грамотной установкой и настройкой GitLab Runner.
Несмотря на то, что потребуется наличие собственного выделенного сервера, с текущей реализацией это не является проблемой.
Таким образом пользователи могут быстро проводить тестирование, управлять версиями и загружать обновления сервиса непрерывно без последствий для конечных пользователей продуктов.

Такой подход позволяет сильно упростить работу с ВС на всех этапах разработки.

\subsection{Эксплуатационное назначение}

Комплекс может быть использован пользователями для совместного управления жизненным циклом веб-сервиса.
Для этого пользователю требуется загрузить Dockerfile в удалённый репозиторий.
После этого можно сразу же приступить к развёртке, нажав одну кнопку в веб-интерфейсе GitLab.

В комплексе реализован широкий функционал для общения и взаимодействия пользователей друг с другом.
Можно предлагать и обсуждать Merge Request, вести доски Kanban, а так же вести документации -- всё для того, чтобы добиться максимальной эффективности в процессе совместной разработки ВС.

Данный комплекс становится особенно актуальным на начальных этапах разработки, когда нет лишнего бюджета на альтернативные средства: Jira, Confluence, ClickUp и т. д.
Комплекс не ограничивает пользователя в выборе конкретных ЯП, архитектур или шаблонов проектирования при реализации ВС.