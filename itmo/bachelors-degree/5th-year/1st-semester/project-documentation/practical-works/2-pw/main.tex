\documentclass{../common/TechDoc}
\usepackage[T1]{fontenc}
\usepackage[utf8]{inputenc}
%\usepackage[pdftex]{graphicx}
\DeclareGraphicsRule{*}{mps}{*}{}
\documentclass[12pt]{article}
\usepackage{tabularx}
\usepackage{listings}
\usepackage[russian,british]{babel}
\usepackage[utf8]{inputenc}
\setcounter{tocdepth}{3}

\renewcommand{\cftsecleader}{\cftdotfill{\cftdotsep}}

\newcommand{\intro}[1]{
    \stepcounter{section}
    \section*{\hfillПРИЛОЖЕНИЕ \arabic{section}}
    \begin{center}
        \Large\bf{#1}
    \end{center}
    \markboth{\MakeUppercase{#1}}{}
    \addcontentsline{toc}{section}{Приложение \arabic{section}. #1}
}

\lstset{basicstyle=\ttfamily,
    showstringspaces=false,
    commentstyle=\color{red},
    keywordstyle=\color{blue}
}

\title{Автоматизации управления жизненным циклом веб-сервиса}
\author{Студент группы P3555}{С. А. Федюкович}
\academicTeacher{Собственник ИП Федюкович Семен Андреевич}{С. А. Федюкович}

\documentTitle{Пояснительная записка}
\documentCode{RU.17701729.02.06-01 81 01-1}

\begin{document}
    \begin{abstract}
        В данном программном документе приведена пояснительная записка на разработку по <<Автоматизации управления жизненным циклом веб-сервиса>>.

        В данный документ внесены разделы «Введение», «Назначение и область применения программы», «Технические характеристики», «Ожидаемые технико-экономические показатели», «Источники, использованные при разработке».

        В разделе «Введение» указано наименование программы и документы, на основании которых ведется разработка.

        В разделе «Назначение и область применения» указано функциональное назначение программы и эксплуатационное назначение программы.

        В разделе «Технические характеристики» указаны постановка задачи на разработку программы, описание и обоснование метода организации входных и выходных данных, описание и обоснование выбора состава технических и программных средств.

        В разделе «Ожидаемые технико-экономические показатели» указана предполагаемая потребность и экономические преимущества разработки по сравнению с отечественными и зарубежными образцами или аналогами.

        Настоящий документ разработан в соответствии с требованиями:
        \begin{enumerate}
            \item ГОСТ 19.101-77 Виды программ и программных документов [3];
            \item ГОСТ 19.102-77 Стадии разработки [4];
            \item ГОСТ 19.103-77 Обозначения программ и программных документов [5];
            \item ГОСТ 19.104-78 Основные надписи [6];
            \item ГОСТ 19.105-78 Общие требования к программным документам [7];
            \item ГОСТ 19.106-78 Требования к программным документам, выполненным печатным способом [8];
            \item ГОСТ 19.404-79 Пояснительная записка.
            Требования к содержанию и оформлению [9].
        \end{enumerate}
        Изменения к Пояснительной записке оформляются согласно ГОСТ 19.603-78 [10], ГОСТ 19.604-78 [11].
    \end{abstract}

    \newpage

    \tableofcontents

    \newpage
    \section{Введение}

    \newpage
    \section{Назначение и область применения}

    \newpage
    \section{Технические характеристики}

    \newpage
    \section{Ожидаемые технико-экономические показатели}

\end{document}