\documentclass[12pt]{article}

\usepackage[english, russian]{babel}
\usepackage[T2A]{fontenc}
\usepackage[utf8]{inputenc}
\usepackage[left=2cm,right=2cm, top=1cm,bottom=1.5cm,bindingoffset=0cm]{geometry}
\usepackage[explicit]{titlesec}
\usepackage{amsmath}
\usepackage{multirow}
\usepackage{bookmark}
\usepackage{minted}
\usepackage{hyperref}

\bookmarksetup{numbered,level=10}

\titleformat{\section}{\Large\bfseries}{}{0em}{#1\ \thesection}

\newenvironment{e}[1][dummy label]{
    \section{Задание}\label{#1}
    \subsection*{Условие}
    }{
    \subsection*{Решение}
}

\newcommand{\eref}[1]{\hyperref[{e:#1}]{\nameref*{e:#1} \ref*{e:#1}}}


\begin{document}
    \pagestyle{empty}
    \begin{center}
        \textbf{Федеральное государственное автономное образовательное учреждение высшего образования}

        \vspace{5pt}

        {\small
        \textbf{САНКТ-ПЕТЕРБУРГСКИЙ НАЦИОНАЛЬНЫЙ}

        \textbf{ИССЛЕДОВАТЕЛЬСКИЙ УНИВЕРСИТЕТ ИТМО}

        \textbf{ФАКУЛЬТЕТ ПРОГРАММНОЙ ИНЖЕНЕРИИ И КОМПЬЮТЕРНОЙ ТЕХНИКИ}%
        }

        \vspace{140pt}

        {\Large
        \textbf{ИНДИВИДУАЛЬНОЕ ЗАДАНИЕ}

        \vspace{7pt}

        \textbf{ПО ДИСЦИПЛИНЕ}%
        }

        \vspace{10pt}

        {\large
        \textbf{Правовые основы интеллектуальной собственности}

        \vspace{5pt}

        \textbf{}%
        }

        \vspace{170pt}

        \begin{tabular}{lll}
            Проверил:                                                                                   & \hspace{70pt} & Выполнил:                                             \\
            ........................                \rule[0.66\baselineskip]{2cm}{0.4pt}                &               & Студент группы P3555                                  \\
            «\rule[0.66\baselineskip]{1cm}{0.4pt}»  \rule[0.66\baselineskip]{2cm}{0.4pt} \the\year г.   &               & Федюкович С. А. \rule[0.66\baselineskip]{2cm}{0.4pt}  \\
            &               &                                                       \\
            Оценка          \hspace{12pt}           \rule[0.66\baselineskip]{2.7cm}{0.4pt}              &               &                                                       \\
        \end{tabular}

        \vspace*{\fill}

        Санкт-Петербург

        \the\year
    \end{center}
    \newpage
    \pagestyle{plain}
    \setcounter{page}{1}

    \begin{e}
        Компания Drako plc. активно рекламирует свой крем для бритья под брендом CoolDude. Кроме того, эта же компания производит крем для бритья, который продается без бренда. Доходы компании от продажи крема CoolDude составляют €200 млн, а ее операционная прибыль --- €30 млн. Компания ожидает, что маржа прибыли от продажи ее небрендового крема составит 10 \%.

        Какой должна быть оптимальная ставка роялти (в процентах от продаж) при лицензировании бренда CoolDude?
    \end{e}

    Представим доход франчайзера, получаемый им от предоставления прав интеллектуальной собственности третьему лицу $\text{Пр}_\text{лиц-та}$ , как долю $\alpha$ от дополнительной прибыли более высокой по сравнению с получаемой до приобретения предмета лицензии франчайзи $\Delta \text{Пр}_\text{лиц-та}$:

    \begin{equation}
        \text{Пр}_\text{лиц-та} = \Delta \text{Пр}_\text{лиц-та} \cdot \alpha.
    \end{equation}

    Следовательно, ставку роялти $R$ можно представить как отношение прибыли франчайзера $\text{Пр}_\text{лиц-та}$ к базе роялти $\text{БР}$:

    \begin{equation}
        \label{eq:r-base}
        R = \frac{\text{Пр}_\text{лиц-та}}{\text{БР}} = \frac{\text{Пр}_\text{лиц-та}}{C + \text{Пр}}.
    \end{equation}

    Для расчета дополнительной прибыли лицензиата необходимо сопоставить прибыль, получаемую им до приобретения лицензии с прибылью, получаемой после. Расчет дополнительной прибыли лицензиата  приведен в следующем выражении:

    \begin{equation}
        \Delta \text{Пр}_\text{лиц-та} = \text{Пр}_\text{факт} - \text{Пр}_\text{баз} = C \cdot \text{Рент}_\text{факт} - C \cdot \text{Рент}_\text{баз},
    \end{equation}
    где $\text{Рент}_\text{факт}$ --- фактическая рентабельность, достигаемая за счёт использования лицензии; $\text{Рент}_\text{баз}$ --- базовая рентабельность, получаемая без учета использования лицензии.

    Если представить знаменатель в виде $C + C \cdot \text{Рент}_\text{факт}$, получим выражение для расчёта ставки роялти:

    \begin{equation}
        \label{eq:royalty}
        R = \frac{\Delta \text{Пр}_\text{лиц-та} \cdot \alpha}{C + \text{Пр}_\text{факт}} = \frac{C \cdot ( \text{Рент}_\text{факт} - \text{Рент}_\text{баз}) \cdot \alpha}{C \cdot (1 +  \text{Рент}_\text{факт})} = \frac{(\text{Рент}_\text{факт} - \text{Рент}_\text{баз}) \cdot \alpha}{1 +  \text{Рент}_\text{факт}}
    \end{equation}

    Значение $\alpha$ возьмём равным $32$ из расчёта получения исключительной патентной лицензии средней ценности.

    Рассчитаем нужные значения рентабельности и роялти:
    \begin{equation}
        \label{eq:profitability}
        \text{Рент}_\text{баз} = 10\%;\
        \text{Рент}_\text{факт} = \frac{30}{200} \cdot 100 \%  = 15\%;\
        R = \frac{(15 - 10) \cdot 32}{1 +  15} = 10\%
    \end{equation}

    \textbf{Ответ:} оптимальная ставка роялти при лицензировании бренда CoolDude $10 \%$.

    \newpage
    \begin{e}
        INDICOOL --- небольшая, но успешная индийская компания --- производитель кондиционеров. Ее единственный продукт --- оконный кондиционер, который работает полностью на солнечной энергии. Компания вложила приблизительно 500 тыс. рупий в НИОКР с целью разработки солнечных панелей для кондиционеров. Хотя рынок кондиционеров отличается высоким уровнем конкуренции, а кондиционеры INDICOOL являются самой дорогой моделью на рынке, в Индии высокие цены на электричество, поэтому эти кондиционеры пользуются большой популярностью.

        За последние пять лет объем продаж INDICOOL в среднем составлял 10 млн рупий в год, а чистая прибыль составляла 2 млн рупий в год. INDICOOL обладает одним патентом, и это патент на кондиционер. Этот патент представляет собой усовершенствование существующих солнечных панелей, благодаря которому панель может быть гораздо меньше по сравнению с другими традиционными моделями. Стоимость получения и сохранения патента составила 40 тыс. рупий. Его срок действия истекает через три года.

        Учитывая скорое истечение срока действия патента, INDICOOL рассматривает два варианта. Во-первых, компания может представить собственную небрендовую модель кондиционера, чтобы увеличить свою долю на рынке небрендовых моделей до истечения срока действия патента. Цена небрендового изделия позволит получать прибыль в размере 5 \%. Такая маржа прибыли является средней на рынке.

        Во-вторых, компания может переуступить свой патент компании Tata, крупнейшему участнику рынка вентиляторов. Эта компания активно пытается расшириться и уже приобрела компанию-стартап, занимающуюся разработкой солнечных панелей.

        Компания INDICOOL обратилась к вам за консультацией.

        \begin{enumerate}
            \item Какой должна быть ставка роялти при выдаче лицензии на товарный знак INDICOOL исходя из процента продаж? Какие неопределенные факторы присутствуют при этом методе оценки?

            \newpage

            \item Какова стоимость патента INDICOOL в соответствии с подходом на основе дохода? Используйте следующие показатели:
            \begin{itemize}
                \item Год 1:

                чистые экономические преимущества от патента --- $1,0$ млн рупий;

                ставка дисконтирования --- 12 \%.
                \item Год 2:

                чистые экономические преимущества от патента — $1,5$ млн рупий;

                ставка дисконтирования --- 15 \%.
                \item Год 3:

                чистые экономические преимущества от патента — $2,0$ млн рупий;

                ставка дисконтирования --- 20 \%.
            \end{itemize}
        \end{enumerate}
    \end{e}

    \begin{enumerate}
        \item Рассчитаем ставку роялти аналогично предыдущему заданию, имеем:

        \begin{equation}
            \text{Рент}_\text{баз} = 5\%;\
            \text{Рент}_\text{факт} = \frac{2}{10}  \cdot 100 \% = 20\%;\
            R = \frac{(20 - 5) \cdot 35}{1 + 20} = 25\%
        \end{equation}

        Неопределенным фактором является доля $\alpha$ от дополнительной прибыли, равная $35$ из расчёта получения исключительной патентной лицензии средней ценности.

        \item Представим нужные расчёты в виде таблицы:

        \begin{table}[h!]
            \centering
            \begin{tabular}{| c | c | c | c |}
                \hline
                \multicolumn{4}{| c |}{Прогнозируемый доход} \\
                \hline
                & Год 1 & Год 2 & Год 3 \\
                \hline
                Прогнозируемый чистый доход, млн. рупий & $1,0$ & $1,5$ & $2,0$ \\
                \hline
                Ставка дисконтирования, $\%$ & $12$ & $15$ & $20$ \\
                \hline
                Текущая стоимость, млн. рупий & $0,9$ & $1,1$ & $1,6$ \\
                \hline
            \end{tabular}
        \end{table}

        Текущая стоимость:

        \begin{equation}
            0,9 + 1,1 + 1,6 = 3,6 \big[ \text{млн. рупий} \big].
        \end{equation}
    \end{enumerate}

    \textbf{Ответ:} \begin{enumerate}
                        \item $R = 25 \%$. Неопределенным фактором является доля $\alpha$.
                        \item Стоимость патента INDICOOL равняется 3,6 млн. рупий.
    \end{enumerate}

    \newpage
    \begin{e}
        Brussells plc. --- успешная машиностроительная компания, которая владеет множеством ценных патентов. Кроме того, у компании есть значительный репутационный капитал (в частности, хорошие отношения с покупателями и поставщиками), который она считает не менее ценным, чем свои патенты. Компания обладает материальными активами стоимостью €50 млн, а ее прибыль в текущем году составила €20 млн. Средний доход компании от капитальных инвестиций в материальные активы составляет 20 \% (в отношении нематериальных активов компания использует ставку 25 \%).

        Оцените стоимость патентов с помощью метода суперприбыли.
    \end{e}

    Рассчитаем действительную оценку чистых материальных активов фирмы:
    \begin{equation}
        20 \big[\text{млн €}\big] : 20 \% = 100 \big[\text{млн €}\big]
    \end{equation}

    Таким образом, неучтенные активы, участвовавшие в создании прибыли, составляют:
    \begin{equation}
        100 - 50 = 50 \big[\text{млн €}\big]
    \end{equation}

    Отсюда получаем действительную оценку чистых нематериальных активов фирмы:
    \begin{equation}
        50 \big[\text{млн €}\big] \cdot 25 \% = 12,5 \big[\text{млн €}\big]
    \end{equation}

    \textbf{Ответ:} Оценочная стоимость патента равняется 12,5 млн €.

\end{document}