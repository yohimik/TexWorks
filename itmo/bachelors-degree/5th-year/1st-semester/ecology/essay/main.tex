\documentclass[12pt]{article}

\usepackage[english, russian]{babel}
\usepackage[T2A]{fontenc}
\usepackage[utf8]{inputenc}
\usepackage[left=2cm,right=2cm, top=1cm,bottom=1.5cm,bindingoffset=0cm]{geometry}
\usepackage{hyperref}
% \usepackage{multirow}
% \usepackage{hhline}

\usepackage{indentfirst}

% \usepackage{enumitem,kantlipsum}

\usepackage{graphicx}
\graphicspath{{pictures/}}
\DeclareGraphicsExtensions{.pdf,.png,.jpg}

% \usepackage{tikz}
% \usetikzlibrary{patterns}
% \usepackage{pgfplots}
% \pgfplotsset{compat=1.9}
% \usepgfplotslibrary{fillbetween}

% \usepackage{ulem}

% \usepackage{hyperref}

% \usepackage{circuitikz}

% \usepackage{fp}
% \usepackage{xfp}

% \usepackage{siunitx}
% \sisetup{output-decimal-marker={,}}

% \usepackage{minted}

% \let\oldref\ref
% \renewcommand{\ref}[1]{(\oldref{#1})}

\begin{document}
    \pagestyle{empty}
    \begin{center}
        \textbf{Федеральное государственное автономное образовательное учреждение высшего образования}

        \vspace{5pt}

        {\small
        \textbf{САНКТ-ПЕТЕРБУРГСКИЙ НАЦИОНАЛЬНЫЙ}

        \textbf{ИССЛЕДОВАТЕЛЬСКИЙ УНИВЕРСИТЕТ ИТМО}

        \textbf{ФАКУЛЬТЕТ ПРОГРАММНОЙ ИНЖЕНЕРИИ И КОМПЬЮТЕРНОЙ ТЕХНИКИ}%
        }

        \vspace{140pt}

        {\Large
        \textbf{РЕФЕРАТ ПО ДИСЦИПЛИНЕ}

        \vspace{7pt}

        \textbf{ЭКОЛОГИЯ}%
        }

        \vspace{10pt}

        {\large
        \textbf{«Вымирающие виды»}

        \vspace{5pt}

        \textbf{}%
        }

        \vspace{170pt}

        \begin{tabular}{lll}
            Проверил:                                                                                   & \hspace{70pt} & Выполнил:                                             \\
            ........................                \rule[0.66\baselineskip]{2cm}{0.4pt}                &               & Студент группы P3555                                  \\
            «\rule[0.66\baselineskip]{1cm}{0.4pt}»  \rule[0.66\baselineskip]{2cm}{0.4pt} \the\year г.   &               & Федюкович С. А. \rule[0.66\baselineskip]{2cm}{0.4pt}  \\
            &               &                                                       \\
            Оценка          \hspace{12pt}           \rule[0.66\baselineskip]{2.7cm}{0.4pt}              &               &                                                       \\
        \end{tabular}

        \vspace*{\fill}

        Санкт-Петербург

        \the\year
    \end{center}

    \newpage

    \pagestyle{plain}
    \setcounter{page}{1}

    \tableofcontents

    \newpage

    \section*{Введение}

    Каждый день в мире происходит множество изменений: одни биологические виды находят преимущество над другими и запускается каскад изменений для всех. Так происходит благодаря эволюции и естественному отбору; дальше всё зависит от того, кто способен адаптироваться, а кто нет, иначе говоря быть эвритопом. Тогда основными вопросами с точки зрения экологии будут: как и кого могут задеть эти изменения? какие негативные последствия несут эти изменения? и как можно устранить или обратить негативные последствия? Ответить на эти вопросы можно только проанализировав исторические данные и сделав правильные выводы. В следующих главах этого реферата кратко будут приведены исторические примеры вымерших видов, разобрано почему так произошло и как и на что это повлияло.

    \newpage
    \section{Разбор исторических примеров}

    \subsection{Маврикийский дронт}

    Маврикийский дронт или додо (лат. Raphus) -- нелетающая птица семейства дронтов, эндемик острова Маврикий.

    О поведении додо известно не так много, а самые современные описания очень краткие. В них упоминается, что птица жила на фруктовых деревьях, гнездилась на земле и высиживала всего одно яйцо.

    Стандартная теория вымирания состоит в том, что голландские моряки съели большинство представителей этого вида. Додо была невероятно легко поймать в связи с тем, что она не испытывала никакого страха перед людьми, (почему он не опасалась существ много больше ее размера эта другая загадка). В этой теории есть рациональное зерно и доказательства. Моряки высадились и поселились на острове в 1598 году, в разных источниках есть подтверждение, что на Додо действительно охотились моряки из-за их неуклюжести.

    Отсюда следует вывод зачем в том числе и в нашей жизни нам нужно проявлять чувство страха.

    \subsection{Мамонт}

    Мамонт --- это вымерший род из группы хоботных с длинными загнутыми бивнями, северные виды были покрыты длинными волосами. Их останки находят в Африке, Европе, Азии и Северной Америки.

    Предположение о том, что за вымирание крупных животных ледникового периода, в том числе и мамонтов, ответственны люди, еще в XIX столетии озвучил британский натуралист, биолог и антрополог Альфред Уоллес (правда ранее он склонялся больше к версии о том, что причиной этому стали колоссальные физические изменения, сопровождавшие ледниковый период). Эта теория вскоре обрела большую популярность. В частности, предполагается, что человек вел активную охоту на мамонтов и других крупных представителей доисторической фауны, что и привело к стремительному уменьшению их популяции. Безусловно, влияние различных климатических изменений на сокращение ареала мамонтов сторонники антропогенной теории не отрицают, но при этом придерживаются мнения, что именно активная охота сыграла ключевую роль в исчезновении этих млекопитающих. В пользу этой точки зрения сотрудники Калифорнийского университета в Беркли и Университета штата Пенсильвания в 2009 году привели результаты проведенного ими масштабного исследования: ученым удалось выяснить, что вскоре после того, как люди заселили Северную Америку, видовое разнообразие млекопитающих упало как минимум на 15\%-42\% по сравнению с уровнем видового разнообразия, существовавшим в течение миллионов лет.

    Несмотря на популярность теории о том, что ответственность за вымирание мамонтов лежит в первую очередь на человечестве, есть множество тех, кто считает, что роль антропогенного фактора в данном вопросе слишком переоценена: в пользу этой точки зрения они приводят доводы о том, что вместе с мамонтами исчезло немало других видов, которые не представляли для людей ни интереса в качестве добычи, ни угрозы, а прямых доказательств того, что наши предки активно охотились на мамонтов, за все время исследований было найдено довольно мало.

    Как видно из перечисленного выше, всегда может существовать множество абсолютно различных факторов влияющих на вымирание конкретного вида и очень сложно назвать истинную причину.

    \subsection{Диплодок}

    Диплодок (лат. Diplodocus) — крупный завропод из Северной Америки, обитал в юрском периоде 154—152 млн лет назад.

    Первые окаменелые позвонки диплодока были найдены Бенджамином Моджом и Сэмюэлем Уилистоном в 1877 году у Кэнон-Сити, штат Колорадо, США. В следующем 1878 году профессор палеонтологии позвоночных Йельского университета Отниэл Чарльз Марш описал их под видом Diplodocus longus (голотип YPM 1920). Средняя часть хвоста содержала шевроны позвонков необычной формы, в форме «двойных балок», из-за которых диплодок и получил свое название. Экспонаты были размещены в Йельском музее естественной истории Пибоди, в Нью-Хейвене, штат Коннектикут. Череп D. longus был описан Маршем в 1884 году на основе найденного Маршалом Фелчем в сентябре 1883 года образца USNM 2672 (ранее YPM 1921) из Гарден Парка, штат Колорадо.

    Диплодок обладал по истине гигантскими размерами и известен как один из самых длинных динозавров. С ним мог соперничать сейсмозавр, который достигал в длину 50 метров. Кроме этого диплодок один из самых известных и самых изученных растительноядных динозавров.

    Диплодоки, как и многие другие крупные динозавры, вымерли в самом конце юрского периода - около 145 млн.л.н. Причины могут быть разными. Либо это изменения в той местности где обитали диплодоки. Сократилась кормовая база и динозаврам просто стало нечем питаться. Или пищи стало мало, чтобы прокормиться таким гигантам. Но возможно их исчезновение связано с появлением новых хищников, которые охотились на молодняк.

    \newpage
    \section{Разбор вымирающих видов}

    \subsection{Амурский тигр}

    Самая крупная хищная кошка: размером с малолитражным авто — три метра, весом как 55 домашних кошек — 220 кг. Живет в снегах ростом с человека, в температуре -40°С. Без проблем запрыгнет на двухэтажный дом, догонит машину со скоростью 80 км/ч по снегу, но предпочитает так не делать — гуляет в диком лесу.

    Обитает на юге Дальнего Востока: по берегам рек Амур и Уссури, в предгорьях Сихотэ-Алиня. Это Хабаровский и Приморский края. Еще часть животных теперь живет в Еврейской автономной области.

    \subsection{Дикий северный олень}

    Рогатый кочевник, которого весь мир знает как помощника Деда Мороза. По величине схож с лошадью или пони, но весит поменьше. Олени мигрируют до 3000 км за год — могут развести подарки от Таймыра до островов Северного-Ледовитого океана. Но они этого не делают, потому что 60\% времени добывают пищу и едят. Их скорость от 20 до 70 км/ч, причем реки — не помеха, переплывают в 9 раз быстрее человека.

    В России их места обитания — в Красноярском крае, Якутии, Карелии, на Сахалине, Кольском п-ве, Камчатке, горах Урала и Сибири, Чукотке, Якутии и Дальнем Востоке.

    \subsection{Сайгак}

    Древняя степная антилопа: пережила ледниковый период, пообщалась с мамонтами, а теперь переживает глобальное потепление. У этой антилопы самый странный нос — похож на хобот. Он фильтрует воздух от пыли и нагревает воздух в морозы. А еще самцы с его помощью хоркают — издают низкий звук, чтобы показать свое превосходство перед другими самцами. Это небольшое животное размером с овцу в день может пробежать 200 км со скоростью поезда — 60 км/ч.

    В России Сайгаки обитают в Северо-Западном Прикаспии — это Астраханская область и Республика Калмыкия.

    \subsection{Байкальская нерпа}

    Байкальская нерпа, или байкальский тюлень (лат. Pusa sibirica) --- один из трёх пресноводных видов тюленя в мире, эндемик озера Байкал, реликт третичной фауны.

    Средняя длина взрослой нерпы от носа до кончика задних ластов – 165 см. Вес может доходить до 130 кг, самки обычно крупнее самцов. На льду или твёрдом грунте байкальские тюлени практически беспомощны, передвигаются медленно, толкаясь передними и задними ластами или короткими скачками в случае опасности. Стихия нерпы – вода. Однако и там эти млекопитающие предпочитают плавать неспешно, редко развивая скорость более 10 км\\ч. Причина в том, что питается нерпа преимущественно голомянкой, бычками, моллюсками и ракообразными. Ей нет необходимости развивать большую скорость в погоне за быстрой рыбой и рыбакам, занимающимся промысловым ловом на Байкале тюлени конкуренции не создают.

    В конце октября 2017 года на южном побережье Байкала нашли тела 141 нерпы (38 - в Иркутской области, остальные - в Бурятии), точную причину их гибели установить не удалось. В июне 2018 года на восточном берегу озера было обнаружено 13 туш нерпы. Как выяснилось после проверки, животные погибли от парвовирусного энтерита. Эксперты высказывали мнение, что произошедшие инциденты - это сигналы, свидетельствующие о том, что популяция изменяется.

    Другими очевидными причинами могут быть выбросы мусора в озеро и как следствие уменьшение популяции основной пищи нерпы --- омуля.

    \newpage
    \section*{Заключение}

    В ходе работы над рефератом были разобраны причины вымирания видов из истории и на основании этого показано, какие виды вымирают во времени написания данной работы. Для того, чтобы понимать какие процессы и на что повлияли и нужна наука о взаимодействиях живых организмов между собой и с их средой обитания --- Экология.

    \newpage
    \section*{Список источников}

    \begin{enumerate}
        \item \href{https://extinct-animals.fandom.com/ru/wiki/\%D0\%9C\%D0\%B0\%D0\%B2\%D1\%80\%D0\%B8\%D0\%BA\%D0\%B8\%D0\%B9\%D1\%81\%D0\%BA\%D0\%B8\%D0\%B9_\%D0\%B4\%D1\%80\%D0\%BE\%D0\%BD\%D1\%82}{https://extinct-animals.fandom.com/ru/wiki/Маврикийский\_дронт} --- Маврикийский дронт
        \item
        \href{https://earthz.ru/why/Pochemu-vymerla-ptica-Dodo}{https://earthz.ru/why/Pochemu-vymerla-ptica-Dodo} --- Почему вымерла птица Додо
        \item
        \href{https://extinct-animals.fandom.com/ru/wiki/\%D0\%9C\%D0\%B0\%D0\%BC\%D0\%BE\%D0\%BD\%D1\%82\%D1\%8B}{https://extinct-animals.fandom.com/ru/wiki/Мамонты} --- Мамонты
        \item
        \href{https://extinct-animals.fandom.com/ru/wiki/\%D0\%94\%D0\%B8\%D0\%BF\%D0\%BB\%D0\%BE\%D0\%B4\%D0\%BE\%D0\%BA}{https://extinct-animals.fandom.com/ru/wiki/Диплодок} --- Диплодок
        \item
        \href{https://www.dinozavro.ru/juras/diplodoc.php}{https://www.dinozavro.ru/juras/diplodoc.php} --- Диплодок
        \item
        \href{https://kuzuk.ru/baikal/poi/nerpa}{https://kuzuk.ru/baikal/poi/nerpa} --- Байкальская нерпа
        \item
        \href{https://nauka.tass.ru/nauka/7513601}{https://nauka.tass.ru/nauka/7513601} --- Решение о регулировании численности нерпы на Байкале примут после исследований в 2020 году
        \item
        \href{https://wwf.ru/resources/blogs/plain-language-about-the-foundation-s-work/posts/endangered-animals-of-russia/}{https://wwf.ru/resources/blogs/plain-language-about-the-foundation-s-work/posts/endangered-animals-of-russia/} --- Исчезающие животные России
    \end{enumerate}
\end{document}