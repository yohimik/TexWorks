\documentclass[14pt]{article}

\usepackage{mathtext}
\usepackage{amsmath}
\usepackage[english, russian]{babel}
\usepackage[TS1]{fontenc}
\usepackage[utf8]{inputenc}
\usepackage[left=2cm,right=2cm, top=1cm,bottom=1.5cm,bindingoffset=0cm]{geometry}

\usepackage{float}
\usepackage{tabularx}
% \usepackage{multirow}
% \usepackage{hhline}

% \usepackage{indentfirst}

% \usepackage{enumitem,kantlipsum}

% \usepackage{graphicx}
% \graphicspath{{pictures/}}
% \DeclareGraphicsExtensions{.pdf,.png,.jpg}

% \usepackage{tikz}
% \usetikzlibrary{patterns}
% \usepackage{pgfplots}
% \pgfplotsset{compat=1.9}
% \usepgfplotslibrary{fillbetween}

% \usepackage{ulem}

% \usepackage{hyperref}

% \usepackage{circuitikz}

% \usepackage{fp}
% \usepackage{xfp}

% \usepackage{siunitx}
% \sisetup{output-decimal-marker={,}}

% \usepackage{minted}

% \let\oldref\ref
% \renewcommand{\ref}[1]{(\oldref{#1})}

\begin{document}
    \pagestyle{empty}
    \begin{center}
        \textbf{Федеральное государственное автономное образовательное учреждение высшего образования}

        \vspace{5pt}

        {\small
        \textbf{САНКТ-ПЕТЕРБУРГСКИЙ НАЦИОНАЛЬНЫЙ ИССЛЕДОВАТЕЛЬСКИЙ УНИВЕРСИТЕТ ИНФОРМАЦИОННЫХ ТЕХНОЛОГИЙ, МЕХАНИКИ И ОПТИКИ}

        \textbf{ФАКУЛЬТЕТ ПРОГРАММНОЙ ИНЖЕНЕРИИ И КОМПЬЮТЕРНОЙ ТЕХНИКИ}%
        }

        \vspace{140pt}

        {\Large
        \textbf{Реферат по дисциплине}

        \vspace{7pt}

        \textbf{операционные системы}%
        }

        \vspace{10pt}

        {\large
        \textbf{«Сравнение операционных систем Linux и Windows.}

        \vspace{5pt}

        }

        \vspace{170pt}

        \begin{tabular}{lll}
            Проверил: & \hspace{70pt} & Выполнил:                                             \\
            Сентерев Ю. А.                \rule[0.66\baselineskip]{2cm}{0.4pt}                & & Студент группы P3355                                  \\
            «\rule[0.66\baselineskip]{1cm}{0.4pt}»  \rule[0.66\baselineskip]{2cm}{0.4pt} \the\year г. & & Федюкович С. А. \rule[0.66\baselineskip]{2cm}{0.4pt}  \\
            & &                                                       \\
            & &                                                       \\
        \end{tabular}

        \vspace*{\fill}

        Санкт-Петербург

        \the\year
    \end{center}
    \newpage
    \pagestyle{plain}
    \setcounter{page}{1}

    \section{Введение}

    В настоящее время представить нашу жизнь без информационных технологий невозможно.
    Компьютеры получили огромное распространение как в домашнем пользовании, так и в производственных организациях.
    Особенно предприятия делают большой упор на создание автоматизированного рабочего места за счет компьютерных технологий.

    Информационные технологии развиваются огромными темпами.
    Какое-то оборудование либо программное обеспечение выпущенное сегодня, уже через год может быть вытеснено на задний план более усовершенствованным оборудованием или программным обеспечением.

    Операционные системы помогают пользователю взаимодействовать с компьютером для выполнения каких-то операций или задач.
    Когда человек совершает платежные операции в мультикассе для зачисления денежных средств на свой сотовый телефон, сам не подозревая того, он взаимодействует непосредственно с самой операционной системой в которой запущена специализированная программа для выполнения конкретной операции.

    Или же кассир отбивает на кассе купленные продукты покупателя, здесь кассир также непосредственно взаимодействует с операционной системой.
    Начисление заработной платы сотрудникам конкретной организации происходит на банковские карточки, где сотрудник идет к банкомату, чтобы снять свою честно заработанную денежную сумму и там непосредственно в банкомате он взаимодействует с операционной системой.

    Информационные технологии серьезно внедряются в нашу жизнь, без них мы много чего не сможем сделать.

    Наибольшей популярностью в мире пользуются операционные системы фирмы Microsoft.
    Их доля составляет 95\% среди всех операционных систем.
    Так же популярны операционные системы под названием Linux.
    Возникают дискуссии между пользователями по поводу какая система лучше, а какая хуже.

    Операционные системы Windows и Linux являются многопользовательскими многозадачными.
    В них широко развита поддержка сети, защита данных и множество других одинаковых функций.
    В результате этого у них совпадают области интересов относительно потребителя, что и послужило основой конфликта разгоревшегося как между поклонниками ОС, как и между их создателями.

    \newpage
    \section{Понятия операционных систем}

    Операционная система (ОС) связывает аппаратное обеспечение и прикладные программы.
    Многие свойства различных программ похожи, и операционная система обычно предоставляет этот общий сервис.
    Например, практически все программы считывают и записывают информацию на диск или отображают ее на дисплее.
    И хотя каждая программа в принципе может содержать инструкции, выполняющие эти повторяющиеся задачи, использование в этих целях операционной системы более практично.

    Операционная система может взаимодействовать с аппаратными средствами и пользователем или прикладными программами.
    Она также может переносить информацию между аппаратурой и прикладным программным обеспечением.

    Прикладной программист не должен беспокоиться о написании специального программного кода для записи данных на все множество дисков, которое может быть на ПК.
    Программист просто просит операционную систему записать данные на диск, а ОС занимается зависящей от аппаратуры информацией.
    Операционная система получает предоставляемые прикладными программами данные и записывает их на физический диск.

    Использование операционной системы делает программное обеспечение более общим: программы могут работать на любом компьютере, на котором можно запустить эту операционную систему, поскольку взаимодействуют с операционной системой, а не с аппаратурой.

    В каждой операционной системе существует несколько видов интерфейсов:
    \begin{enumerate}
        \item командный (текстовый) интерфейс;
        \item текстовый или графический полноэкранный интерфейс;
        \item графический многооконный пиктографический интерфейс;
    \end{enumerate}

    Рассмотрим каждый интерфейс более подробно:

    \textbf{Командный (текстовый) интерфейс.}
    Всякая операционная система имеет командный интерфейс (иногда в скрытой форме).
    В первой из ОС (OS/360) взаимодействие с пользователями было жестко поделено: диалоговый режим --- команда запуска --- остановки задач, привязки носителей информации к устройствам, получения информации о заданиях, ожидающих выполнения, вывода, наличия свободной памяти и свободных устройств и другие;
    и описание состава и структуры процесса обработки данных --- последовательность запуска программ, входные и выходные файлы, условия, при которых те или иные программы должны быть выполнены или пропущены и другие.

    В большинстве ОС в настоящее время сложился более или менее унифицированный формат командной строки.
    Командная строка включает в себя:
    \begin{enumerate}
        \item тип операции (имя команды или программы);
        \item рабочий вход (входные файлы или устройства);
        \item рабочий выход (выходные файлы или устройства);
        \item управляющий вход (управляющие параметры или ключи команды);
        \item управляющий выход (обычно – протокол, содержащий диагностику ошибок, код завершения или другую информацию).
    \end{enumerate}

    Командная строка --- приглашение оболочки, обозначающее готовность системы принимать команду пользователя, --- в наиболее явной форме демонстрирует идею диалога.
    На каждую введенную команду пользователь получает ответ от системы: либо очередное приглашение, обозначающее, что команда выполнена и можно вводить следующую, либо сообщение об ошибке, представляющее собой высказывание системы о произошедших в ней событиях, адресованное пользователю.
    При работе в операционной среде с графическим интерфейсом происходящий диалог пользователя с системой не столь очевиден, хотя с точки зрения системы клик мышью в определенной области на экране аналогичен команде, введенной с клавиатуры, а ответ системы пользователю может быть представлен в виде диалогового окна.

    \textbf{Текстовый или графический полноэкранный интерфейс.}
    Он имеет, как правило, в верхней части экрана систему меню с подсказками.
    Меню часто бывает выпадающим (ниспадающим – pull-down).

    Для управления компьютером курсор экрана или курсор мыши после поиска в древе каталогов устанавливается на командные файлы программ (*.exe, *.com, *.bat) и для запуска программы нажимается клавиша <Enter> или правая кнопка мыши.
    Различные файлы могут выделяться разным цветом или иметь разный рисунок.
    Каталоги (папки) отличаются от файлов размером или рисунком.
    Данный интерфейс является основным для всех видов программных оболочек.

    \textbf{Графический многооконный пиктографический интерфейс.}
    Представляет собой рабочий стол, на котором располагаются пиктограммы (значки или иконки программ).
    Все операции производятся, как правило, мышью.
    Для управления компьютером курсор мыши подводят к пиктограмме и запуск программы осуществляют щелчком левой кнопки мыши по пиктограмме.
    Это наиболее удобный и перспективный интерфейс, осебенно при работе с программами.
    Примеры: интерфейс с компьютеров AppleMacintosh, Windows 10, Ubuntu GNOME.

    Графический интерфейс пользователя (GUI --- GraphicsUserInterface).
    Появление ОС и оболочек с развитыми диалоговыми графическими средствами и средств программирования, позволяющих создавать графические интерфейсы, а особенно --- объектно–ориентированных систем программирования – привело к внедрению и широкому распространению элементов экранного интерфейса.

    Оболочка Microsoft Windows не была изначально операционной системой, так как она существует «поверх» операционной системы типа MS-DOS. Она возникла в виде стандартизатора графического интерфейса и прижилась исключительно потому, что пользователь хотел видеть программу, с которой ему часто приходится работать, красивой, практичной, удобной и легкой в освоении и использовании.

    \newpage
    \subsection{История создания Microsoft Windows Seven}
    Windows 7 --- это операционная система от компании Microsoft, пришедшая на смену Windows ХР и Windows Vista и выпущенная 22 октября 2009 года.
    Официально разработка Windows 7 началась сразу же после выпуска Windows Vista в конце 2006 года, но некоторые идеи были заложены еще в проекте Longhorn, работа над которым началась в 2001 году, после выхода операционной системы Windows ХР.
    Изначально в Longhorn планировалось реализовать целый комплекс фундаментальных новшеств, но за три года работы над этим проектом разработчикам так и не удалось создать полноценную рабочую систему.
    Сроки выхода Longhorn постоянно отодвигались, и для спасения проекта пришлось принимать радикальные меры.

    В середине 2004 года руководство Microsoft решило начать разработку операционной системы Longhorn заново, исключив из нее некоторые важные функции.
    Результатом этой работы стал выход в начале 2007 года операционной системы Windows Vista.
    Эта система получила неоднозначную оценку специалистов и пользователей.
    За два года после выхода Windows Vista на нее перешла лишь небольшая часть пользователей, а наиболее популярной оставалась проверенная временем Windows ХР.

    \subsection{История создания Linux}
    28 декабря 1968 г. в обычной финской семье Нильса и Анны Торвальдс родился сын Линус.
    Любимыми предметами Линуса всегда были математика и физика.
    Ему нравились точные науки, дающие возможность поломать голову над решением той или иной задачи.
    Ему было интересно пообщаться на математические темы, а также поиграть с калькулятором – одним из главных рабочих инструментов дедушки Лео – профессора статистики в Университете Хельсинки.
    Примитивный калькулятор – все, что Линусу тогда было нужно для счастья. В 1981 г. дедушка-профессор купил Commodore VIC-20.
    И в возрасте 10 лет он начал увлекаться программированием, активно работая на домашнем компьютере.

    В 1989 г., когда Линус готовился поступить в университет, на конференции ассоциации Usenix в Торонто представители корпорации AT\&T объявили о новой системе цен на UNIX System V: около 40 тысяч долларов в расчете на один процессор (7,5 тыс. долл. для учебных заведений).
    Это были очень большие деньги.
    Профессор Амстердамского университета Эндрю Таненбаум в ответ на это занялся написанием Minix-усеченной версии UNIX, способной работать на ПК.

    Чтобы досконально изучить Minix, Линусу Торвальдсу понадобилось не больше месяца.
    Он уже был постоянным читателем технических конференций.
    ОС Таненбаума была чем-то вроде учебного пособия по миру UNIX.
    Поэтому в ней было много ограничений.
    Это не могли исправить ни патчи, ни дополнительные программы.
    Линуса раздражали в Minix многие вещи, но больше всего – эмулятор терминала, сделанный просто ужасно.
    Уже привыкший все нужные программы писать для себя самостоятельно, Торвальдс взялся за разработку нормального терминала.

    Самодельный эмулятор терминала быстро обрастал наворотами.
    Когда он, наконец, был готов, Линус решил разбавить его новыми возможностями.
    Например, функциями upload и download.
    Для этого требовалось написать драйвер дисковода, а для него, в свою очередь – создать файловую систему.
    Сложная, трудоемка работа, но закаленному ночными посиделками программисту нравилось решать такие задачи и процесс пошел.

    Так как в университете весной 1991 года делать было, в общем-то, нечего, Линус целыми днями не выходил из своей комнаты.
    От написания файловой системы его отвлекали разве что сон и иногда еда.
    Через несколько недель проект, первоначально задуманный как продвинутая терминальная программа, уже больше напоминал целую операционную систему.
    Когда автор понял, что зашел слишком далеко, останавливаться было уже поздно.
    Линус Торвальдс вообще был не из тех, кто мог бросить все на полпути.

    В начале сентября оболочка будущей операционной системы, наконец, заработала.
    Несмотря на то, что про себя Торвальдс называл ее Linux, для официального релиза готовилось имя Freax – автор не хотел, чтобы его считали нескромным.
    Тем не менее, Ари Лемке – преподавателю одного из вузов Хельсинки, согласившемуся выделить для системы место на институтском компьютере, название Linux понравилось больше, и уже скоро на ftp.funet.fi/pub/OS/Linux появилась первая версия системы со знаком 0.01.
    Эту версию мало кто «щупал» – он была еще очень «сырой», и чтобы заставить ее работать, нужно было потратить много времени и нервов.

    В октябре вышла Linux 0.02, а в ноябре – 0.03. Первыми бета-тестерами Linux стали читатели comp.os.minix , которые, хоть и посылали сообщения об ошибках пачками, но всячески хвалили новую ОС.
    Однако по настоящему завоевывать популярность Linux начала, когда в конце ноября стала полностью автономной.
    Армия линуксоидов стремительно росла.
    Многие предлагали свою помощь, присылали программы и патчи для Linux.

    К тому времени, как вышла версия Linux 1.0, о системе уже знал весь мир.
    Популярность привлекла к ней внимание многих крупных компаний.
    Благодаря своей гибкости и потенциалу, она поселилась на сотнях тысяч серверов в качестве основной ОС.
    Поддержать Linux решили тысячи хакеров со всего мира, которые все вместе трудятся над улучшениями.
    Помимо основной версии, разрабатываемой автором, появилось множество дистрибутивов, каждый из которых имеет свои плюсы и минусы.

    \newpage
    \section{Сравнительный анализ операционных систем}
    \subsection{Краткий обзор каждой операционных систем}
    Обе операционные системы предназначены как для персональных систем, так и для web-серверов, вычислительных кластеров и других систем.

    Windows удалось завоевать первенство на настольных и персональных системах (около 90\% настольных компьютеров) тогда как Linux популярна на web-серверах, вычислительных кластерах и в суперкомпьютерах (50-90\%).

    Эти системы разнятся в лежащей в основе их философии, стоимости, простоте использования, удобстве и стабильности.
    При их сравнении приходится принимать во внимание корни, исторические факторы и способ распространения.

    Windows и Linux трудно сравнивать на равных из-за следующих факторов:
    \begin{enumerate}
        \item Linux --- это не определенная ОС, их более 600, среди них есть те, которые отличаются друг от друга значительно, а некоторые совсем немного, кроме того, на популярные дистрибутивы может существовать до 100 версий;
        \item Словом Linux могут обозначаться разные понятия, в некоторых случаях это просто ядро операционной системы, в других случаях --- полноценные операционные системы в дистрибутиве с графическим интерфейсом;
        \item Оба порядка систем поставляются в различных конфигурациях, особенно Linux, для которой существует огромное количество вариантов, некоторые из них предназначены для узкого круга задач;
        \item Цена и широта технической поддержки различаются у разных поставщиков, а также в зависимости от версии и дистрибутива;
        \item Производители оборудования могут устанавливать дополнительное ПО с операционной системой, которое делает доступные функции системы разнообразнее, иногда они даже спонсируют продавца, снижая цену продукта для пользователя;
        \item Данные, полученные от маркетинговых подразделений, и результаты тестирования могут расходиться;
        \item Microsoft распространяет Windows под разными лицензиями (закрытыми); дистрибутивы Linux, со своей стороны, могут содержать проприетарные компоненты.
    \end{enumerate}

    \newpage

    Дальнейшее сравнение представлено в виде таблиц:
    \begin{table}[H]
        \caption{Настольные версии}
        \begin{tabularx}{\textwidth}{|X|X|X|}
            \hline
            Параметр сравнения & Windows & Linux \\
            \hline
            Доля при продаже компьютеров & Предустанавливается почти на все продаваемые настольные системы & Предустанавливается на небольшое количество продаваемых систем. \\
            \hline
            Оконные менеджеры/графическая среда & Изначально только системный оконный менеджер. Графическая оболочка необходима для работы подавляющего большинства программ, и её отказ ведет к нарушению их функционирования. Существует ряд программ, которые работают без использования графической оболочки, но служат они преимущественно для технического обслуживания системы (например, восстановления работоспособности). & Среды рабочего стола: GNOME, KDE и другие. Множество оконных менеджеров: Openbox, Fluxbox, и другие. Графическая оболочка не критична для работы ОС, она может переключаться в текстовый режим. Удалённое управление осуществляется, обычно, через SSH, VNC и XDMCP. Используются «виртуальные терминалы», что позволяет избежать перезагрузки системы в случае отказа одного из терминалов. \\
            \hline
            Системная консоль/командная строка & Командная строка существует, но обладает ограниченной функциональностью. Базируется на MS-DOS, наследуя её скромные возможности, мало изменившиеся с 1990-х годов. Разработан также мощный командный процессор Windows PowerShell, реализующий некоторые возможности командной строки UNIX, основанный на .NET . Функции по восстановлению или настройке могут выполняться из командной строки. & Командная строка позволяет опытному пользователю полностью перенастроить все функции ОС. Существует множество утилит для выполнения специализированных функций, тесно интегрированных с системными и прикладными программами. Функции по восстановлению или настройке могут выполняться из командной строки. \\
            \hline
        \end{tabularx}
    \end{table}

    \newpage

    \begin{table}[H]
        \caption{Установка}
        \begin{tabularx}{\textwidth}{|X|X|X|}
            \hline
            Параметр сравнения & Windows & Linux \\
            \hline
            Размер установщика & Как правило, один компакт-диск & От одной дискеты до нескольких DVD дисков. \\
            \hline
            Сложность установки & Довольно проста в установке. & Сильно варьирует между дистрибутивами в основном из-за разной степени предварительного конфигурирования. Существуют варианты с удобной и графической инсталляцией, (SuSE, Mandriva, Ubuntu, Fedora и др.) и варианты с инсталляторами через меню (Debian, Vector Linux, ArchLinux, Slackware) \\
            \hline
            Время установки & Заявленное время составляет около часа (вплоть до 10-30 минут для Windows 7, в зависимости от мощности компьютера). & От 6 минут до часа и более, в зависимости от объёма устанавливаемого ПО, поставляемого с дистрибутивом. В среднем составляет 6-30 минут для распространенных дистрибутивов. \\
            \hline
            Поставляемое программное обеспечение & Несколько программ для работы с мультимедиа и сетью интернет (браузер Internet Explorer, проигрыватель Windows Media Player, текстовые редакторы Notepad, WordPad, графический редактор Paint), почтовый клиент Outlook Express. & Присутствует множество программ для самых разных задач: мультимедиа, графики, интернета, офисной работы, игр, а также системные утилиты и дополнительные визуальные оболочки. \\
            \hline
            Дополнительное ПО & Огромный выбор собственнических и свободно распространяемых программ (Однако нет централизованного хранилища необходимого для работы свободного ПО, поддерживаемого производителем ОС). Как правило, они поставляются со всеми необходимыми библиотеками, устанавливаются с помощью специальной программы-инсталлятора. & Большой выбор свободно распространяемых программ и небольшой выбор коммерческих. Программы, включенные в официальные дистрибутивы и их репозитории, устанавливаются в большинстве вариантов с помощью специальной программы для установки/удаления программ, обеспечивающей наличие необходимых библиотек (система управления пакетами), либо ручной компиляцией из исходных кодов с поиском необходимых библиотек (в случае редких программ. \\
            \hline
            Подготовка диска & По умолчанию устанавливает только себя, затирая возможность запуска других ОС. Разделы с родной файловой системой NTFS легко могут быть расширены и уменьшены. & Возможна установка нескольких операционных систем. В большинстве дистрибутивов есть возможность запуска полноценной системы с компакт диска, а значит работать со всеми функциями, включая мощную графическую утилиту переразметки GPartEd, работающую с большим набором файловых систем, включая NTFS. \\
            \hline
            Программа-загрузчик & Может загружать операционные системы семейства Windows NT/9x по выбору пользователя (NTLDR), но не Linux и другие подобные системы. & Может загружать операционные системы по выбору пользователя с помощью встроенных менеджеров GRUB или LILO. \\
            \hline
        \end{tabularx}
    \end{table}

    \newpage

    \begin{table}[H]
        \caption{Другие возможности}
        \begin{tabularx}{\textwidth}{|X|X|X|}
            \hline
            Параметр сравнения & Windows & Linux \\
            \hline
            Единообразие между различными версиями & Между различными версиями сохраняется высокая степень сходства в интерфейсе. Но в случае Windows Server 2008 появились значительные отличия в интерфейсе, особенно оснасток(snap-in) администрирования & В зависимости от дистрибутива, его версии, графической оболочки и программ, работа интерфейса может быть разной. Тем не менее, доступно множество настроек, и пользователь может переносить их из версии в версию. \\
            \hline
            Единообразие между программами & Все программы, выпущенные Microsoft в один и тот же период, следуют единым принципам построения интерфейса. & Программы, следующие принципам KDE и GNOME за определённый период, наследуют единые принципы. Однако, множество независимых программ может им не соответствовать.  \\
            \hline
            Единообразие процедуры обновления программ и ОС & Во всех последних версиях windows используется процедура автоматического получения обновлений и «заплаток» для самой ОС, драйверов и программ, выпущенных Microsoft. & Системы управления пакетами содержат в себе средства для автоматического обновления программ (самой ОС и установленных пользователем). В качестве источников обновлений обычно служат репозитории дистрибутивов и отдельных проектов.  \\
            \hline
            Настройка & Исходный код может быть приобретён для строго ограниченных целей, а, кроме того, программы сторонних разработчиков могут изменять системные настройки. Другими словами, возможно случайное нарушение лицензионного соглашения. & Весь код системы доступен для модификации. Большая часть сторонних программ также предоставляет исходный код. \\
            \hline
            Особые возможности & Обе системы позволяют настроить особые режимы управления компьютером, такие как укрупненные шрифты, чтение надписей вслух, медленное нажатие на клавиши и др. & \\
            \hline
        \end{tabularx}
    \end{table}

    \newpage

    \subsection{Положительные и отрицательные стороны операционных системы}
    Положительные стороны Linux:
    \begin{enumerate}
        \item Отсутствие самой страшной угрозы для компьютера в целом --- вирусов. В семействе операционных систем Linux, система безопасности реализована кардинально иным способом, на уровне ядра системы, а не на уровне сторонних приложений. На каждое системное действие требуется ввести пароль администратора (root);
        \item Быстрая установка без проблем;
        \item Сравнительное быстродействие системы --- возможность выполнения сразу нескольких задач одновременно;
        \item Более быстрая и адекватная работа в локальной и внешней сетях;
        \item Стабильность и гибкость ОС, реестр конфигураций не засоряется, а вся информация сохраняется в виде текстовых файлов, что не «тормозит» работу системы;
        \item Разнообразие дистрибутивов и репозиториев с набором игр под Linux и программ на любой выбор;
        \item Возможность запуска Windows программ в режиме эмуляции Windows.
    \end{enumerate}

    Отрицательные стороны Linux:
    \begin{enumerate}
        \item Не хватает программ для решения узкопрофильных задач, например, нет полноценного аналога программ Adobe Photoshop, Autocad и т.д.(аналоги Windows программ в Linux);
        \item Нередки проблемы с поддержкой внешних и внутренних устройств (принтеров, сканнеров);
        \item Слабая распространенность среди пользователей, которые бояться изменить своим привычкам, выработанным в Windows.
    \end{enumerate}

    Положительные стороны Windows:
    \begin{enumerate}
        \item Потребление ОС Windows 7 ресурсов компьютера. Компания Microsoft для 32-битной версии рекомендует 1-Гц процессор, 1 Гб ОП, 16 Гб, видеокарта 128 Мб, а для 64-битной 1-Гц процессор, 2 Гб ОП, накопитель объёмом 20 Гб видеокарта 128 Мб. Как видим, заявленные требования практически аналогичны требованиям ОС Vista. Одним из больших минусов Windows Vista было то, что интерфейс Aero потреблял огромное количество ресурсов комптютера, в 7 версии эта проблема практически устранена.
        \item Панель задач новой ОС Windows 7 предоставляет возможность лучшего просмотра, имеет большое количество функций, которые помогают пользователям ускорить процесс управления приложениями и окнами. В целом же инновации Панели задач можно только приветствовать, хотя границы вокруг активных приложений недостаточно ясно различимы.
        \item  В Windows 7 снята проблема предыдущей операционной системы по поводу настройки безопасности пользователя (которая очень сильно мешала нормально и конструктивно работать), теперь в ОС появилось 2 промежуточных параметра, которые бьют тревогу, только если программа изменяет параметры установки.
        \item В ОС Windows 7 появилась новая функция --- "Библиотеки", которая предоставляет виртуальные папки для документов, музыки, фотографий и видео, призвана заменить устаревшую функцию "Мои документы".
        \item Несомненным плюсом Windows 7 является поддержка multitouch-ввода (при работе на сенсорном экране, пользователь видит версию меню с меньшей точностью навигации), а также "центр мобильности".
    \end{enumerate}

    Отрицательные стороны Windows
    \begin{enumerate}
        \item Новая функция HomeGroups – это способ делиться папками полными медиафайлов и документов между компьютерами через сеть работает плохо;
        \item Не предусмотрено обновлений с прошлых версий Windows;
        \item Проблемы в работе ожидаются следующие: несовместимость драйверов и другого ПО;
    \end{enumerate}

    \newpage
    \subsection{Сравнительный анализ}

    Различные организации выбирают Linux из-за фактов. Возвращаясь к теме фактов о Linux, следует сказать, что Linux действительно является надежной, гибкой и высокоэффективной ОС. Вот несколько характерных примеров применения:
    \begin{enumerate}
        \item отдел нуждается в Web или e-mail сервере, и выбор Linux будет лучшим выбором для этих целей;
        \item команде (например, при производстве компьютерной графики для фильма Titanic) требуются эффективные в ценовом отношении вычисления, для чего создается высокоэффективный вычислительный комплекс.
        \item инженеры проводящие многие часы за клавиатурой переходят с Windows на Linux, раздраженные постоянной необходимостью перезагрузки;
        \item интернет-провайдеры (ISP) переходят с Windows на Linux, из-за лучшей управляемости последнего, 24x7, при обслуживании десятков тысяч пользователей;
    \end{enumerate}

    Windows Seven, с другой стороны, традиционно держала рамку первенства, когда требовалась простота использования, легкость установки, прогнозируемость обслуживания, и количество приложений. Но эти различия, похоже, стираются. Многие организации предпочитают поддержку, которую обеспечивает Red Hat или какой другой поставщик Linux, поддержке Microsoft.

    Сейчас Linux лучше, чем Windows справляется с установкой plug-and-play устройств. Рабочий стол Linux можно настроить, чтобы он выглядел не только как Windows, но и можно запускать пакеты приложений, которые по функциональности эквивалентны Microsoft Office. Реализация новых стандартов и протоколов происходит раньше в Linux. Это из-за того, что исходный код легко доступен, заплаты, для дефектов в аппаратуре, для Linux иногда выходят в тот же день.

    Windows остается предпочтительной в многих случаях. Для организаций, которых она устраивает, тех кто имеет совместимое или достаточно мощное аппаратное обеспечение, и особенно для тех, кто полагается на ActiveX или другие собственные протоколы Microsoft, Linux не даст больших преимуществ.

    Например, команда разработчиков Wired HotBot, подтвердила, что они пожертвовали надежностью и эффективностью в недавнем переходе к Windows, но сделав это получила доступ к ряду новых технологий Windows.

    Windows хвалится репликацией службы каталогов, криптографическим API с экспортной лицензией, обработкой транзакций, и рядом других новинок.

    Конкретная система подходит оптимальным образом, а где проявляются ее недостатки. Бессмысленно говорить о преимуществах операционной системы абстрактно, в отрыве от решаемых задач.

    \newpage
    \section{Заключение}

    В каждой операционной системе как сказано выше есть свои положительные и отрицательные стороны. Как WindowsSeven, так и Linux «хороши» сами по себе. Фирма Microsoft сделала акцент на простоте использования операционной системы пользователями за счет графических представлений.

    Windows подойдет людям, которым нужен мультимедийный центр (музыка, кино, интернет, игры). И для тех, кому нужен не дорогой и не слишком сложный в использовании компьютер для работы.

    Linux, да и вообще UNIX-подобные системы --- лучший вариант для серверов. Профессионалы (программисты и системные администраторы) любят эти системы за высокую гибкость и надежность.

    По занимаемому месту дискового пространства если оставить «голую» операционную систему, то Windows Seven все же будет занимать 8 Гб памяти, в то время как Linux– 2Гб.

    Выбор операционной системы каждому предстоит сделать самому, пользователь должен оценить свои навыки, умения и знания, также оценить те требования которые он желает использовать от компьютера.

    \newpage
    \section{Список источников}
    \begin{enumerate}
        \item https://pcpro100.info/chto-luchshe-windows-ili-linux/ --- Что лучше Linux или Windows
        \item http://pro-spo.ru/linux-windows/2035--linux-windows --- Сравнение Linux и Windows
        \item https://pingvinus.ru/note/linux-better-windows --- Почему Linux лучше Windows
        \item https://keddr.com/2017/05/bolee-podrobno-o-linux-i-windows/ --- Более подробно о Linux и Windows
        \item https://losst.ru/8-prichin-smenit-windows-10-na-linux --- 9 причин сменить Windows 10 на Linux
        \item Командная строка Linux. Полное руководство
        \item Ten Steps to Linux Survival
        \item Linux. Карманный справочник
        \item Внутреннее устройство Windows. 7-е издание
        \item Windows 10 для IT-специалистов
        \item Революционная десятка. Все секреты и тайны операционной системы Windows 10
    \end{enumerate}
\end{document}
