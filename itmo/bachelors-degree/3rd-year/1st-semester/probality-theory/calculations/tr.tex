\documentclass[14pt]{article}

\usepackage{mathtext}
\usepackage{amsmath}
\usepackage[english, russian]{babel}
\usepackage[TS1]{fontenc}
\usepackage[utf8]{inputenc}
\usepackage[left=2cm,right=2cm, top=1cm,bottom=1.5cm,bindingoffset=0cm]{geometry}

% \usepackage{multirow}
% \usepackage{hhline}

% \usepackage{indentfirst}

\usepackage{enumitem,kantlipsum}

% \usepackage{graphicx}
% \graphicspath{{pictures/}}
% \DeclareGraphicsExtensions{.pdf,.png,.jpg}

\usepackage{tikz}
\usetikzlibrary{patterns}
\usepackage{pgfplots}
% \pgfplotsset{compat=1.9}
% \usepgfplotslibrary{fillbetween}

% \usepackage{ulem}

% \usepackage{hyperref}

% \usepackage{circuitikz}

% \usepackage{fp}
% \usepackage{xfp}

% \usepackage{siunitx}
% \sisetup{output-decimal-marker={,}}

% \usepackage{minted}

% \let\oldref\ref
% \renewcommand{\ref}[1]{(\oldref{#1})}

\begin{document}
    \pagestyle{empty}
    \begin{center}
        \textbf{Федеральное государственное автономное образовательное учреждение высшего образования}

        \vspace{5pt}

        {\small
        \textbf{САНКТ-ПЕТЕРБУРГСКИЙ НАЦИОНАЛЬНЫЙ ИССЛЕДОВАТЕЛЬСКИЙ УНИВЕРСИТЕТ ИНФОРМАЦИОННЫХ ТЕХНОЛОГИЙ, МЕХАНИКИ И ОПТИКИ}

        \textbf{ФАКУЛЬТЕТ ПРОГРАММНОЙ ИНЖЕНЕРИИ И КОМПЬЮТЕРНОЙ ТЕХНИКИ}%
        }

        \vspace{140pt}

        {\Large
        \textbf{Типовой расчёт 1}

        \vspace{7pt}

        \textbf{по теории вероятностей и математической статистике}%
        }

        \vspace{10pt}

        \vspace{175pt}

        \begin{tabular}{lll}
            Проверил: & \hspace{70pt} & Выполнил:                                             \\
            Хартов А. А.                            \rule[0.66\baselineskip]{2.3cm}{0.4pt}              & & Студент группы P3355                                  \\
            «\rule[0.66\baselineskip]{1cm}{0.4pt}»  \rule[0.66\baselineskip]{2cm}{0.4pt} \the\year г. & & Федюкович С. А. \rule[0.66\baselineskip]{2cm}{0.4pt}  \\
            & & Вариант 21                                                     \\
            Оценка          \hspace{12pt}           \rule[0.66\baselineskip]{2.7cm}{0.4pt}              & &                                                       \\
        \end{tabular}

        \vspace*{\fill}

        Санкт-Петербург

        \the\year
    \end{center}
    \newpage
    \pagestyle{plain}
    \setcounter{page}{1}

    \section*{Тема 1.
    Непосредственный подсчёт вероятностей в рамках классической схемы.}

    \subsection*{Задача 1}
    \subsubsection*{Условие}

    Какова вероятность того, что в наудачу выбранном четрырёхзначном числе нет повторяющихся чисел?

    \subsubsection*{Решение}

    \begin{enumerate}[wide, labelwidth=!, labelindent=0pt]
        \item Пространство элементарных событий $\omega$ состоит из множества четырёхзначных чисел $\{1000, 1001, ... 9999\}$.
        Его мощность равна:
        \[card (\omega)= 9 \cdot 10 \cdot 10 \cdot 10 = 9000 \]
        \item Событие $A$, вероятность которого нам нужно найти, состоит из множества четырёхзначных чисел, в которых цифры числа не повторяюстя.
        Тогда:
        \[card (A) = 9 \cdot 9 \cdot 8 \cdot 7 = 4536\]
        \item Таким образом, искомая вероятность события $A$ будет равна:
        \[P(A) = \frac{card(A)}{card(\omega)} = \frac{4536}{9000} = \frac{63}{125} = 0,504\]
    \end{enumerate}

    \hspace{290pt}\textbf{Ответ:} $P(A) = 0,504$

    \subsection*{Задача 2}
    \subsubsection*{Условие}

    В коробке лежат карандаши: двенадцать красных и восемь зелёных. Наудачу извлекают три.
    Какова вероятность того, что среди извлечённых будет хотя бы один красный карандаш?

    \subsubsection*{Решение}

    \begin{enumerate}[wide, labelwidth=!, labelindent=0pt]
        \item Пространство элементарных событий $\omega$ состоит из всех сочетаний множества $\{1, 2, ..., 20\}$ по $3$.
        Его мощность равна числу этих сочетаний:
        \[card (\omega)= C_{20}^3 = \frac{20!}{(20 - 3)!3!} = 1140\]
        \item Пусть $A$ --- искомое событие. Найдём вероятность $\overline{A}$, которое будет состоять из всех сочетаний множества $\{1,2,...8\}$ по $3$.
        Тогда:
        \[card (\overline{A}) = C_{8}^3 = \frac{8!}{(8 - 3)!3!} = 56\]
        \item Тогда вероятность события $\overline{A}$ будет равна:
        \[P(\overline{A}) = \frac{card(\overline{A})}{card(\omega)} = \frac{56}{1140} = \frac{14}{285}\]
        \item Таким образом, вероятность события $A$ будет равна:
        \[P(A) = 1 - P(\overline{A}) = 1 - \frac{14}{285} = \frac{271}{285}\]
    \end{enumerate}
    \hspace{290pt}\textbf{Ответ:} $P(A) = \frac{271}{285}$

    \newpage

    \subsection*{Задача 3}
    \subsubsection*{Условие}

    Цифры от 1 до 9 располагаются в случайном порядке.
    Какова вероятность того, что все нечётные цифры окажутся на нечётных местах?

    \subsubsection*{Решение}

    \begin{enumerate}[wide, labelwidth=!, labelindent=0pt]
        \item Пространство элементарных событий $\omega$ состоит из всех перестановок множества $\{1, 2, ..., 9\}$.
        Его мощность равна числу этих перестановок:
        \[card (\omega)= P_9 = 9! = 362880\]
        \item Событие $A$, вероятность которого нам нужно найти, состоит из множества девятизначных чисел, в которых нечётные цифры стоят на нечётных местах.
        Тогда:
        \[card (A) = 5 \cdot 4 \cdot 4 \cdot 3 \cdot 3 \cdot 2 \cdot 2 \cdot 1 \cdot 1 = 2880\]
        \item Таким образом, искомая вероятность события $A$ будет равна:
        \[P(A) = \frac{card(A)}{card(\omega)} = \frac{2880}{362880} = \frac{1}{126}\]
    \end{enumerate}
    \hspace{290pt}\textbf{Ответ:} $P(A) = \frac{1}{126}$

    \subsection*{Задача 4}
    \subsubsection*{Условие}

    Станция метро оборудована тремя независимо работающими эскалаторами.
    Вероятность безотказной работы в течение дня для первого эскалатора равна $9/10$, для второго --- $8/10$, для третьего --- $85/100$.
    Найти вероятность того, что в течение дня произойдёт поломка не более одного эскалатора.

    \subsubsection*{Решение}

    \begin{enumerate}[wide, labelwidth=!, labelindent=0pt]
        \item Пусть $A$ --- искомое событие. Обозначим события безотказной работы с первого по третий эскалатор $A_1$, $A_2$ и $A_3$ соответственно.
        Тогда искомая вероятность будет равна:
        \begin{gather*}
            P(A) = P(A_1 A_2 A_3) + P(\overline{A_1} A_2 A_3) + P(A_1 \overline{A_2} A_3) + P(A_1 A_2 \overline{A_3}) =\\
            = 0,9 \cdot 0,8 \cdot 0,85 + 0,1 \cdot 0,8 \cdot 0,85 + 0,9 \cdot 0,2 \cdot 0,85 + 0,9 \cdot 0,8 \cdot 0,15 =\\
            = 0,941
        \end{gather*}
    \end{enumerate}
    \hspace{290pt}\textbf{Ответ:} $P(A) = 0,941$

    \subsection*{Задача 21}
    \subsubsection*{Условие}

    Предприятием послана автомашина за различными материалами на четыре базы.
    Вероятность наличия нужного материала на первой базе равна $9/10$, на второй --- $85/100$, на третьей --- $7/10$, на четвертой --- $65/100$.
    Найти вероятность того, что только на первой базе не окажется нужного материала.

    \subsubsection*{Решение}

    \begin{enumerate}[wide, labelwidth=!, labelindent=0pt]
        \item Пусть $A$ --- искомое событие. Обозначим события наличия материала с первой по четвертую базу $A_1$, $A_2$, $A_3$ и $A_4$ соответственно.
        Тогда искомая вероятность будет равна:
        \begin{gather*}
            P(A) = P(\overline{A_1} A_2 A_3 A_4) + P(A_1 \overline{A_2} A_3 A_4) + P(A_1 A_2 \overline{A_3} A_4) + P(A_1 A_2 A_3 \overline{A_4}) =\\
            = 0,1 \cdot 0,85 \cdot 0,7 \cdot 0,65 + 0,9 \cdot 0,15 \cdot 0,7 \cdot 0,65 + 0,9 \cdot 0,85 \cdot 0,3 \cdot 0,65 + 0,9 \cdot 0,85 \cdot 0,7 \cdot 0,35 =\\
            = 0,4367
        \end{gather*}
    \end{enumerate}
    \hspace{290pt}\textbf{Ответ:} $P(A) = 0,4367$

    \newpage

    \section*{Тема 2. Геометрические вероятности}

    \subsection*{Задача 21}
    \subsubsection*{Условие}

    Из отрезка $[0,1]$ наудачу выбираются три числа.
    Какова вероятность того, что их сумма не будет превышать единицу?

    \subsubsection*{Решение}
    \begin{enumerate}[wide, labelwidth=!, labelindent=0pt]
        \item Областью Лебега $\Omega$ является куб со стороной $1$:
        \[\mu(\Omega) = 1\]
        \item Обозначим три числа $x$, $y$ и $z$, а интересующую нас область $A$:
        \begin{gather*}
            A =
            \begin{cases}
                x + y + z \leq 1 \\
                0 \leq x \leq 1 \\
                0 \leq y \leq 1 \\
                0 \leq z \leq 1 \\
            \end{cases}
        \end{gather*}
        \item Найдём объём области $A$:
        \begin{gather*}
            \mu(A) = \iiint \limits_{A} \,dx \, dy \, dz =  \int \limits_0^1 \, dx \int \limits_0^{1 - x} \, dy \int \limits_0^{1 - x - y} \, dz = \\
            = \int \limits_0^1 \, dx \int \limits_0^{1 - x} (1 - x - y) \, dy = \frac{1}{2} \int \limits_0^1  (x - 1)^2 \, dx = \frac{1}{6}
        \end{gather*}
        \item Таким образом, искомая вероятность будет равна:
        \[P(A) = \frac{\mu(A)}{\mu(\Omega)} = \frac{1/6}{1} = \frac{1}{6}\]
    \end{enumerate}
    \hspace{290pt}\textbf{Ответ:} $P(A) = \frac{1}{6}$

    \newpage

    \section*{Тема 3. Формула полной вероятности. Формула Байеса}

    \subsection*{Задача 21}
    \subsubsection*{Условие}

    На шахматную доску $4 \times 4$ ставят два ферзя.
    Какова вероятность того, что они бьют друг друга?

    \subsubsection*{Решение}

    \begin{enumerate}[wide, labelwidth=!, labelindent=0pt]
        \item Пусть $H_г$ --- событие, состоящее в том, что ферзи расположились на одной горизонтали; $H_в$ --- одной вертикали; $H_д$ --- одной диагонали.
        Найдём их вероятности:
        \begin{gather*}
            P(H_г) = P(H_в) = 4 \cdot \frac{C_4^2}{C_{16}^2} = \frac{1}{5}\\
            P(H_д) = 2 \cdot \frac{C_4^2 + 2 \cdot C_3^2 + 2 \cdot C_2^2}{C_{16}^2} = \frac{4}{15}\\
            P(A|H_г) = P(A|H_в) = P(A|H_д) = 1
        \end{gather*}
        \item Таким образом:
        \begin{gather*}
            P(A) = P(A|H_г) \cdot P(H_г) + P(A|H_в) \cdot P(H_в) + P(A|H_д) \cdot P(H_д) =\\
            = 1 \cdot \frac{1}{5} + 1 \cdot \frac{1}{5} + 1 \cdot \frac{4}{15} = \frac{2}{3}\\
        \end{gather*}
    \end{enumerate}
    \hspace{290pt}\textbf{Ответ:} $P(A) = \frac{2}{3}$

    \section*{Тема 4. Схема Бернулли}

    \subsection*{Задача 21}
    \subsubsection*{Условие}

    Вероятность того, что покупателю потребуется обувь $40\text{-го}$ размера равна $0,4$.
    В обувной отдел вошли пять покупателей.
    Найти вероятность того, что по крайней мере двум из них потребуется обувь $40\text{-го}$ размера.

    \subsubsection*{Решение}

    \begin{enumerate}[wide, labelwidth=!, labelindent=0pt]
        \item Пусть $A$ --- событие того, что в обувной отдел вошёл покупатель, которому требуется обувь $40\text{-го}$ размера:
        \begin{gather*}
            p = P(A) = 0,6 \\
            q = 1 - p = 0,4 \\
            n = 5
        \end{gather*}
        \item Найдём вероятность $P_5(2, 5)$ того, что количество покупателей, которым нужна обувь $40\text{-го}$ размера заключено между $2$ и $5$:
        \begin{gather*}
            P_5(2, 5) = C_5^2 \cdot 0,6^2 \cdot 0,4^3 + C_5^3 \cdot 0,6^3 \cdot 0,4^2 + C_5^4 \cdot 0,6^4 \cdot 0,4^1 + C_5^5 \cdot 0,6^5 = \\
            = 10 \cdot 0,36 \cdot 0,064 + 10 \cdot 0,216 \cdot 0,16 + 5 \cdot 0,1296 \cdot 0,4 + 1 \cdot 0,07776 = 0,91296
        \end{gather*}
    \end{enumerate}
    \hspace{290pt}\textbf{Ответ:} $P_5(2, 5) = 0,91296$

    \newpage

    \section*{Тема 5. Дискретные случайные величины}

    \subsection*{Задача 21}
    \subsubsection*{Условие}

    Батарея состоит из четырёх орудий.
    Вероятность попадания в мишень при одном выстреле равна $0,9$ для первого орудья, для второго такая вероятность равна $0,8$, для третьего и четвертого $0,6$.
    Наугад выбирают три орудия, и каждое из них стреляет один раз.
    Построить ряд распределения, найти функцию распределения, математическое ожидание, среднее квадратичное отклонение, моду и медиану числа попаданий в мишень.
    Найти вероятность хотя-бы одного попадания и хотя бы одного непопадания в мишень.
    \subsubsection*{Решение}

    \begin{enumerate}[wide, labelwidth=!, labelindent=0pt]
        \item Для построения ряда распределения найдём вероятности количества попаданий в мишень.
        Пусть $A_x$ --- вероятность $x$ попаданий в мишень, тогда:
        \begin{gather*}
            P(A_0) = \frac{1}{4} ( 0,1 \cdot 0,2 \cdot 0,4 + 0,1 \cdot 0,2 \cdot 0,4 + 0,2 \cdot 0,4 \cdot 0,4 + 0,1 \cdot 0,4 \cdot 0,4 ) = 0.016  \\
            P(A_1) = \frac{1}{4} ( 0,9 \cdot 0,2 \cdot 0,4 + 0,1 \cdot 0,8 \cdot 0,4 + 0,1 \cdot 0,2 \cdot 0,6
            + 0,9 \cdot 0,2 \cdot 0,4 + 0,1 \cdot 0,8 \cdot 0,4 + 0,1 \cdot 0,2 \cdot 0,6 + \\
            + 0,8 \cdot 0,4 \cdot 0,4 + 0,2 \cdot 0,6 \cdot 0,4 + 0,2 \cdot 0,4 \cdot 0,6
            + 0,9 \cdot 0,4 \cdot 0,4 + 0,1 \cdot 0,6 \cdot 0,4 + 0,1 \cdot 0,4 \cdot 0,6 ) = 0,162 \\
            P(A_2) = \frac{1}{4} ( 0,9 \cdot 0,8 \cdot 0,4 + 0,1 \cdot 0,8 \cdot 0,6 + 0,9 \cdot 0,2 \cdot 0,6
            + 0,9 \cdot 0,8 \cdot 0,4 + 0,1 \cdot 0,8 \cdot 0,6 + 0,9 \cdot 0,2 \cdot 0,6 + \\
            + 0,8 \cdot 0,6 \cdot 0,4 + 0,2 \cdot 0,6 \cdot 0,6 + 0,8 \cdot 0,4 \cdot 0,6
            + 0,9 \cdot 0,6 \cdot 0,4 + 0,1 \cdot 0,6 \cdot 0,6 + 0,9 \cdot 0,4 \cdot 0,6 ) = 0.453 \\
            P(A_3) = \frac{1}{4} ( 0,9 \cdot 0,8 \cdot 0,6 + 0,9 \cdot 0,8 \cdot 0,6 + 0,8 \cdot 0,6 \cdot 0,6 + 0,9 \cdot 0,6 \cdot 0,6 ) = 0,369 \\
        \end{gather*}
        \item Получаем закон распределения:

        \begin{center}
            \begin{tabular}{|c|c|c|c|c|}
                \hline
                $x$     & $0$       & $1$       & $2$       & $3$ \\
                \hline
                $x^2$   & $0$       & $1$       & $4$       & $9$ \\
                \hline
                $P$     & $0,016$   & $0,162$   & $0,453$   & $0,369$ \\
                \hline
            \end{tabular}
        \end{center}
        \item Построим функцию распределения:

        \begin{center}
            \begin{tikzpicture}
                \begin{axis}[height=10cm,
                axis lines = center,
                ylabel = {$F(x)$},
                xlabel = {$x$},
                xtick={0,1,2,3},
                ytick={0.016, 0.178, 0.631, 1},
                yticklabels={$0\text{,}016$, $0\text{,}178$, $0\text{,}631$, $1\text{,}000$},
                ymin=-0.1, ymax=1.1,
                xmin=-1, xmax=4]
                    \addplot[very thick,black] coordinates {(-1,0) (0,0)};

                    \addplot[very thick,black] coordinates {(0,0.016) (1,0.016)};
                    \addplot[mark=*,black,mark options={fill=black}] coordinates {(0,0.016)};

                    \addplot[very thick,black] coordinates {(1,0.178) (2,0.178)};
                    \addplot[mark=*,black,mark options={fill=black}] coordinates {(1,0.178)};

                    \addplot[very thick,black] coordinates {(2,0.631) (3,0.631)};
                    \addplot[mark=*,black,mark options={fill=black}] coordinates {(2,0.631)};

                    \addplot[very thick,black] coordinates {(3,1) (4,1)};
                    \addplot[mark=*,black,mark options={fill=black}] coordinates {(3,1)};
                \end{axis}
            \end{tikzpicture}
        \end{center}

        \newpage
        \item Найдём математическое ожидание:
        \begin{gather*}
            M(X) = 0 \cdot 0,016 + 1 \cdot 0,162 + 2 \cdot 0,453 + 3 \cdot 0,369 = 2,175
        \end{gather*}

        \item Найдём среднее квадратическое отклонение:
        \begin{gather*}
            \sigma(X) = \sqrt{0 \cdot 0,016 + 1 \cdot 0,162 + 4 \cdot 0,453 + 9 \cdot 0,369 - 2,175^2} = \sqrt{0.564375}
        \end{gather*}

        \item Мода и медиана будут равны $2$.
        \item Вероятность хотя-бы одного попадания будет равна $1 - P(A_0) = 1 - 0,016 = 0,984$; хотя бы одного непопадания $1 - P(A_3) = 1 - 0,369 = 0,613$.
    \end{enumerate}
    \hspace{200pt}\textbf{Ответ:} $M(X) = 2,175$; $\sigma(X)= \sqrt{0.564375}$.
    Вероятность хотя-бы одного попадания равна $0,984$; хотя бы одного непопадания $0,613$

    \newpage

    \section*{Тема 6. Непрерывные случайные величины}

    \subsection*{Задача 21}
    \subsubsection*{Условие}

    Дана плотность распределения случайной величины $X$:
    \begin{gather*}
        f(x) = \frac{a}{e^x + e^{-x}}
    \end{gather*}
    Найти коэффициент $a$, функцию распределения $F(x)$, вероятность $P\{X \geq 0 \}$, моду $mod(X)$ и медиану $med(X)$.
    \subsubsection*{Решение}
    \begin{enumerate}[wide, labelwidth=!, labelindent=0pt]
        \item Для нахождения коэффициента $a$ возьмём интеграл;
        \begin{gather*}
            \int_{-\infty}^{\infty} f(x) \, dx = a \int_{-\infty}^{\infty} \frac{dx}{e^x + e^{-x}} = a \cdot (\lim_{t\rightarrow\infty} arctg(e^x)\big|_{-t}^{t}) = a \cdot \frac{\pi}{2} = 1
            \Rightarrow a = \frac{2}{\pi}
        \end{gather*}
        \item Найдём функцию $F(x)$:
        \begin{gather*}
            F(x) = \int_{-\infty}^{x} f(x) \, dx = \frac{2}{\pi} \int_{-\infty}^{x} \frac{dx}{e^x + e^{-x}} = \frac{2}{\pi} \cdot (\lim_{t\rightarrow\infty} arctg(e^x)\big|_{-t}^{x}) = \frac{2 \cdot arctg(e^x)}{\pi}
        \end{gather*}
        \item Найдём вероятность $P\{X \geq 0 \}$:
        \begin{gather*}
            F(0) = \frac{2 \cdot arctg(e^0)}{\pi} = \frac{1}{2}
        \end{gather*}
        \item Мода $mod(X) = 0$(максимум плотности распределения).
        \item Найдём медиану $med(X)$:
        \begin{gather*}
            F(med(X)) = \frac{1}{2} \Rightarrow  \frac{2 \cdot arctg(e^{med(X)})}{\pi} = \frac{1}{2} \Rightarrow med(X) = 0
        \end{gather*}
    \end{enumerate}

    \hspace{150pt}\textbf{Ответ:} $a = \frac{2}{\pi}$; $F(x) = \frac{2 \cdot arctg(e^x)}{\pi}$; $P\{X \geq 0 \} = \frac{1}{2}$; $mod(X) = 0$; $med(X) = 0$.

\end{document}\
