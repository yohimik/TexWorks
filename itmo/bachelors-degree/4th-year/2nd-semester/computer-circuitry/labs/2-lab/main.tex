\documentclass[12pt]{article}


\usepackage[english, russian]{babel}
\usepackage[T2A]{fontenc}
\usepackage[utf8]{inputenc}
\usepackage[left=2cm,right=2cm, top=1cm,bottom=1.5cm,bindingoffset=0cm]{geometry}

\usepackage{amsmath}

% \usepackage{multirow}
% \usepackage{hhline}

% \usepackage{indentfirst}

% \usepackage{enumitem,kantlipsum}

\usepackage{graphicx}
\graphicspath{{pictures/}}
\DeclareGraphicsExtensions{.pdf,.png,.jpg}

\usepackage{tikz}
\usetikzlibrary{patterns}
\usepackage{pgfplots}
\pgfplotsset{compat=1.9}
% \usepgfplotslibrary{fillbetween}

% \usepackage{ulem}

% \usepackage{hyperref}

% \usepackage{circuitikz}

% \usepackage{fp}
% \usepackage{xfp}

% \usepackage{siunitx}
% \sisetup{output-decimal-marker={,}}

% \usepackage{minted}

% \let\oldref\ref
% \renewcommand{\ref}[1]{(\oldref{#1})}

\begin{document}
    \pagestyle{empty}
    \begin{center}
        \textbf{Федеральное государственное автономное образовательное учреждение высшего образования}

        \vspace{5pt}

        {\small
        \textbf{САНКТ-ПЕТЕРБУРГСКИЙ НАЦИОНАЛЬНЫЙ}

        \textbf{ИССЛЕДОВАТЕЛЬСКИЙ УНИВЕРСИТЕТ ИТМО}

        \textbf{ФАКУЛЬТЕТ ПРОГРАММНОЙ ИНЖЕНЕРИИ И КОМПЬЮТЕРНОЙ ТЕХНИКИ}%
        }

        \vspace{140pt}

        {\Large
        \textbf{ЛАБОРАТОРНАЯ}

        \vspace{7pt}

        \textbf{РАБОТА №2}%
        }

        \vspace{10pt}

        {\large
        \textbf{Схемотехника ЭВМ}

        \vspace{5pt}

        \textbf{«Стабилитрон»}%
        }

        \vspace{170pt}

        \begin{tabular}{lll}
            Проверил:                                                                                   & \hspace{70pt} & Выполнил:                                             \\
            ........................                \rule[0.66\baselineskip]{2cm}{0.4pt}                &               & Студент группы P3455                                  \\
            «\rule[0.66\baselineskip]{1cm}{0.4pt}»  \rule[0.66\baselineskip]{2cm}{0.4pt} \the\year г.   &               & Федюкович С. А. \rule[0.66\baselineskip]{2cm}{0.4pt}  \\
            &               &                                                       \\
            Оценка          \hspace{12pt}           \rule[0.66\baselineskip]{2.7cm}{0.4pt}              &               &                                                       \\
        \end{tabular}

        \vspace*{\fill}

        Санкт-Петербург

        \the\year
    \end{center}

    \newpage

    \pagestyle{plain}
    \setcounter{page}{1}

    \section*{Цель}

    Изучить стабилитрон:

    \begin{enumerate}
        \item Построить обратную ветвь вольтамперной характеристики стабилитрона и определить напряжения стабилизации.
        \item Вычислить ток и мощность, рассеиваемой стабилитроном.
        \item Определить дифференциальное сопротивление стабилитрона по вольтамперной характеристике.
        \item Исследовать изменения напряжения стабилитрона при изменении входного напряжения в схеме параметрического стабилизатора.
        \item Исследовать изменения напряжения на стабилитроне при изменении сопротивления в схеме параметрического стабилизатора.
    \end{enumerate}

    \section*{Задачи}

    \subsection*{Эксперимент 1}

    \begin{enumerate}
        \item Построить схему, изображенную на Рис. \ref{fig:s:1}, в программе Multisim.
        \item Измерить силу тока и напряжение на диоде при разных значениях ЭДС, внести данные в таблицу и построить график.
        \item По графику определить напряжение стабилизации, ток стабилизации и посчитать мощность стабилизации.
    \end{enumerate}

    \subsection*{Эксперимент 2}

    \begin{enumerate}
        \item Построить схему, изображенную на Рис. \ref{fig:s:2}, в программе Multisim.
        \item Измерить силу тока и напряжение на диоде при коротком замыкании и при значениях резистора $R_1$(100, 300, 600, 1000 $\Omega$).
    \end{enumerate}

    \subsection*{Эксперимент 3}

    \begin{enumerate}
        \item Построить схему, изображенную на Рис. \ref{fig:s:3}, в программе Multisim.
        \item Пронаблюдать ВАХ стабилитрона на экране осциллографа.
    \end{enumerate}

    \newpage

    \section*{Схемы}

    \begin{figure}[ht]
        \centering
        \includegraphics[scale=0.65]{images/s1.png}
        \caption{Схема 1}
        \label{fig:s:1}
    \end{figure}

    \begin{figure}[ht]
        \centering
        \includegraphics[scale=0.65]{images/s2.png}
        \caption{Схема 2}
        \label{fig:s:2}
    \end{figure}

    \begin{figure}[ht]
        \centering
        \includegraphics[scale=0.5]{images/s3.png}
        \caption{Схема 3}
        \label{fig:s:3}
    \end{figure}

    \newpage

    \section*{Выводы}

    \subsection*{Эксперимент 1}

    В ходе работы были получены следующие данные:

    \begin{table}[ht]
        \centering
        \begin{tabular}{|c|c|c|}
            \hline
            $E, V$ & $U, V$ & $I, mA$ \\
            \hline
            -30,00 & -5,11 & -100,00 \\
            \hline
            -25,00 & -5,11 & -83,33 \\
            \hline
            -20,00 & -5,10 & -66,67 \\
            \hline
            -15,00 & -5,09 & -50,00 \\
            \hline
            -10,00 & -5,07 & -33,34 \\
            \hline
            0,00 & 0,00 & 0,00 \\
            \hline
            4,00 & 0,56 & 13,34 \\
            \hline
            6,00 & 0,57 & 20,00 \\
            \hline
            10,00 & 0,58 & 33,33 \\
            \hline
            15,00 & 0,59 & 50,00 \\
            \hline
            20,00 & 0,60 & 66,67 \\
            \hline
            25,00 & 0,61 & 83,33 \\
            \hline
            30,00 & 0,61 & 100,00 \\
            \hline
            35,00 & 0,62 & 116,67 \\
            \hline
            36,00 & 0,62 & 120,00 \\
            \hline
            37,00 & 0,62 & 123,33 \\
            \hline
            40,00 & 0,62 & 133,33 \\
            \hline
            45,00 & 0,62 & 150,92 \\
            \hline
            50,00 & 0,63 & 166,67 \\
            \hline
        \end{tabular}
    \end{table}

    \newpage

    \begin{figure}[ht]
        \centering
        \begin{tikzpicture}
            \begin{axis}[
            xmin = -5.5,
            ymin = -110,
            xmax = 1,
            ymax = 170,
            grid = both,
            axis lines=middle,
            axis line style={->},
            x label style={at={(axis description cs:0.5,-0.2)},anchor=east},
            xlabel = {$U, V$},
            y label style={at={(axis description cs:-0.2,0.5)},anchor=north},
            ylabel = {$I, mA$},
            width=10cm,
            height=10cm
            ]
            \addplot[
                mark=square,
            ]
            coordinates {
                (-5.11, -100)
                (-5.11, -83.33)
                (-5.1, -66.67)
                (-5.09, -50)
                (-5.07, -33.34)
                (0, 0)
                (0.56, 13.34)
                (0.57, 20)
                (0.58, 33.33)
                (0.59, 50)
                (0.6, 66.67)
                (0.61, 83.33)
                (0.61, 100)
                (0.62, 116.67)
                (0.62, 120)
                (0.62, 123.33)
                (0.62, 133.33)
                (0.62, 150.92)
                (0.63, 166.67)
            };
            \end{axis}
        \end{tikzpicture}
    \end{figure}

    $$ U_{\text{min ст.}} = -5,07\ B$$

    $$ U_{\text{max ст.}} = -5,11\ B;\ I_{\text{ст. ном.}} = -50,00\ mA$$

    $$ I_{\text{max ст.}} = -100,00\ mA $$

    $$ P_{\text{cт.}} = I_{\text{ст.}} \cdot U_{\text{ст.}} = 0,25\ B $$

    \newpage

    \subsection*{Эксперимент 2}

    В ходе работы были получены следующие данные:

    \begin{table}[ht]
        \centering
        \begin{tabular}{|c|c|c|}
            \hline
            $R,\ \Omega$ & $U,\ mV$ & $I,\ mA$ \\
            \hline
            100,00 & 599,55 & 66,67 \\
            \hline
            300,00 & 601,25 & 66,67 \\
            \hline
            600,00 & 601,66 & 66,67 \\
            \hline
            1000,00 & 601,82 & 66,67 \\
            \hline
        \end{tabular}
    \end{table}

    $$ U_{\text{ст.}} = 602,02\ mV $$

    \subsection*{Эксперимент 3}

    В ходе работы на экране осциллографа была получена следующая ВАХ:

    \begin{figure}[ht]
        \centering
        \includegraphics[scale=0.9]{images/o1.png}
        \caption{ВАХ стабилитрона}
        \label{fig:o:1}
    \end{figure}


\end{document}