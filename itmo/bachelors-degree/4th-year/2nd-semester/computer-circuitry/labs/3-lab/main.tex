\documentclass[12pt]{article}


\usepackage[english, russian]{babel}
\usepackage[T2A]{fontenc}
\usepackage[utf8]{inputenc}
\usepackage[left=2cm,right=2cm, top=1cm,bottom=1.5cm,bindingoffset=0cm]{geometry}

\usepackage{amsmath}

% \usepackage{multirow}
% \usepackage{hhline}

% \usepackage{indentfirst}

% \usepackage{enumitem,kantlipsum}

\usepackage{graphicx}
\graphicspath{{pictures/}}
\DeclareGraphicsExtensions{.pdf,.png,.jpg}

% \usepackage{tikz}
% \usetikzlibrary{patterns}
% \usepackage{pgfplots}
% \pgfplotsset{compat=1.9}
% \usepgfplotslibrary{fillbetween}

% \usepackage{ulem}

% \usepackage{hyperref}

% \usepackage{circuitikz}

% \usepackage{fp}
% \usepackage{xfp}

% \usepackage{siunitx}
% \sisetup{output-decimal-marker={,}}

% \usepackage{minted}

% \let\oldref\ref
% \renewcommand{\ref}[1]{(\oldref{#1})}

\begin{document}
    \pagestyle{empty}
    \begin{center}
        \textbf{Федеральное государственное автономное образовательное учреждение высшего образования}

        \vspace{5pt}

        {\small
        \textbf{САНКТ-ПЕТЕРБУРГСКИЙ НАЦИОНАЛЬНЫЙ}

        \textbf{ИССЛЕДОВАТЕЛЬСКИЙ УНИВЕРСИТЕТ ИТМО}

        \textbf{ФАКУЛЬТЕТ ПРОГРАММНОЙ ИНЖЕНЕРИИ И КОМПЬЮТЕРНОЙ ТЕХНИКИ}%
        }

        \vspace{140pt}

        {\Large
        \textbf{ЛАБОРАТОРНАЯ}

        \vspace{7pt}

        \textbf{РАБОТА №3}%
        }

        \vspace{10pt}

        {\large
        \textbf{Схемотехника ЭВМ}

        \vspace{5pt}

        \textbf{«Однополупериодные и двухполупериодные выпрямители»}%
        }

        \vspace{170pt}

        \begin{tabular}{lll}
            Проверил:                                                                                   & \hspace{70pt} & Выполнил:                                             \\
            ........................                \rule[0.66\baselineskip]{2cm}{0.4pt}                &               & Студент группы P3455                                  \\
            «\rule[0.66\baselineskip]{1cm}{0.4pt}»  \rule[0.66\baselineskip]{2cm}{0.4pt} \the\year г.   &               & Федюкович С. А. \rule[0.66\baselineskip]{2cm}{0.4pt}  \\
            &               &                                                       \\
            Оценка          \hspace{12pt}           \rule[0.66\baselineskip]{2.7cm}{0.4pt}              &               &                                                       \\
        \end{tabular}

        \vspace*{\fill}

        Санкт-Петербург

        \the\year
    \end{center}

    \newpage

    \pagestyle{plain}
    \setcounter{page}{1}

    \section*{Цель}

    Изучить однополупериодный и двухполупериодный выпрямитель:

    \begin{enumerate}
        \item Провести анализ процессов в схемах однополупериодного и двухполупериодного выпрямителей.
        \item Сравнить формы входного и выходного напряжения для однополупериодного и двухполупериодного выпрямителя.
        \item Определить частоты выходного сигнала в схемах однополупериодного выпрямителя и двухполупериодного выпрямителя с
        выводом средней точки трансформатора.
        \item Сравнить максимальные значения выходного напряжения для схем двухполупериодного и однополупериодного выпрямителей.
        \item Сравнить частоты выходного сигнала для схем двухполупериодного и однополупериодного выпрямителей.
    \end{enumerate}

    \section*{Задачи}

    \subsection*{Эксперимент 1}

    \begin{enumerate}
        \item Построить схему, изображенную на Рис. \ref{fig:s:1}, в программе Multisim.
        \item Измерить максимальные входные и выходные напряжения и период $T$ выходного напряжения по осциллограмме.
        \item Вычислить частоту выходного сигнала.
    \end{enumerate}

    \subsection*{Эксперимент 2}

    \begin{enumerate}
        \item Построить схему, изображенную на Рис. \ref{fig:s:2}, в программе Multisim.
        \item Измерить максимальные входные и выходные напряжения и период $T$ выходного напряжения по осциллограмме.
        \item Вычислить частоту выходного сигнала.
    \end{enumerate}

    \newpage

    \section*{Схемы}

    \begin{figure}[ht]
        \centering
        \includegraphics[scale=0.5]{images/s1.png}
        \caption{Схема 1}
        \label{fig:s:1}
    \end{figure}

    \begin{figure}[ht]
        \centering
        \includegraphics[scale=0.5]{images/s2.png}
        \caption{Схема 2}
        \label{fig:s:2}
    \end{figure}

    \newpage

    \section*{Выводы}

    \subsection*{Эксперимент 1}

    В ходе работы на экране осциллографа были получены следующие ВАХ:

    \begin{figure}[ht]
        \centering
        \includegraphics[scale=0.7]{images/o1.png}
        \caption{ Осциллограмма входного сигнала однополупериодного выпрямителя}
        \label{fig:o:1}
    \end{figure}

    \begin{figure}[ht]
        \centering
        \includegraphics[scale=0.7]{images/o2.png}
        \caption{Осциллограмма выходного сигнала однополупериодного выпрямителя
        }
        \label{fig:o:2}
    \end{figure}

    $$ U_{\text{max(вх)}} = 16,96\ V;\ U_{\text{max(вх)}} =  16,29\ V $$
    $$ T = 16,67\ ms;\ f = \frac{1}{T} = \frac{1}{16,67} \approx 0,06 $$

    \newpage

    \subsection*{Эксперимент 2}

    В ходе работы на экране осциллографа была получена следующиая ВАХ:

    \begin{figure}[ht]
        \centering
        \includegraphics[scale=0.7]{images/o3.png}
        \caption{Осциллограмма двухполупериодного выпрямителя}
        \label{fig:o:3}
    \end{figure}

    $$ U_{\text{max(вх)}} = 8,45\ V;\ U_{\text{max(вх)}} =  3,00 V $$
    $$ T = 16,87\ ms;\ f = \frac{1}{T} = \frac{1}{16,87} \approx 0,06 $$

    \newpage




\end{document}