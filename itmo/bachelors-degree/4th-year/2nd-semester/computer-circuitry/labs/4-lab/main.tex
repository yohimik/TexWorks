\documentclass[12pt]{article}


\usepackage[english, russian]{babel}
\usepackage[T2A]{fontenc}
\usepackage[utf8]{inputenc}
\usepackage[left=2cm,right=2cm, top=1cm,bottom=1.5cm,bindingoffset=0cm]{geometry}

\usepackage{amsmath}

% \usepackage{multirow}
% \usepackage{hhline}

% \usepackage{indentfirst}

% \usepackage{enumitem,kantlipsum}

\usepackage{graphicx}
\graphicspath{{pictures/}}
\DeclareGraphicsExtensions{.pdf,.png,.jpg}

% \usepackage{tikz}
% \usetikzlibrary{patterns}
% \usepackage{pgfplots}
% \pgfplotsset{compat=1.9}
% \usepgfplotslibrary{fillbetween}

% \usepackage{ulem}

% \usepackage{hyperref}

% \usepackage{circuitikz}

% \usepackage{fp}
% \usepackage{xfp}

% \usepackage{siunitx}
% \sisetup{output-decimal-marker={,}}

% \usepackage{minted}

% \let\oldref\ref
% \renewcommand{\ref}[1]{(\oldref{#1})}

\begin{document}
    \pagestyle{empty}
    \begin{center}
        \textbf{Федеральное государственное автономное образовательное учреждение высшего образования}

        \vspace{5pt}

        {\small
        \textbf{САНКТ-ПЕТЕРБУРГСКИЙ НАЦИОНАЛЬНЫЙ}

        \textbf{ИССЛЕДОВАТЕЛЬСКИЙ УНИВЕРСИТЕТ ИТМО}

        \textbf{ФАКУЛЬТЕТ ПРОГРАММНОЙ ИНЖЕНЕРИИ И КОМПЬЮТЕРНОЙ ТЕХНИКИ}%
        }

        \vspace{140pt}

        {\Large
        \textbf{ЛАБОРАТОРНАЯ}

        \vspace{7pt}

        \textbf{РАБОТА №4}%
        }

        \vspace{10pt}

        {\large
        \textbf{Схемотехника ЭВМ}

        \vspace{5pt}

        \textbf{«Мультивибратор на операционном усилителе»}%
        }

        \vspace{170pt}

        \begin{tabular}{lll}
            Проверил:                                                                                   & \hspace{70pt} & Выполнил:                                             \\
            ........................                \rule[0.66\baselineskip]{2cm}{0.4pt}                &               & Студент группы P3455                                  \\
            «\rule[0.66\baselineskip]{1cm}{0.4pt}»  \rule[0.66\baselineskip]{2cm}{0.4pt} \the\year г.   &               & Федюкович С. А. \rule[0.66\baselineskip]{2cm}{0.4pt}  \\
            &               &                                                       \\
            Оценка          \hspace{12pt}           \rule[0.66\baselineskip]{2.7cm}{0.4pt}              &               &                                                       \\
        \end{tabular}

        \vspace*{\fill}

        Санкт-Петербург

        \the\year
    \end{center}

    \newpage

    \pagestyle{plain}
    \setcounter{page}{1}

    \section*{Цель}

    Изучить мультивибратор на операционном усилителе:

    \begin{enumerate}
        \item Используя пакет Multisim собрать мультивибратор, включив в нее модель 741 (или LF147 для симметричного мультивибратора).

        \item Исследовать схему мультивибратора в режиме анализа переходных процессов.

        \item Составить отчет, в который должны быть включены: схемы мультивибраторов и ВАХ для входного и выходного сигнала.
    \end{enumerate}

    \section*{Задачи}

    \subsection*{Эксперимент 1}

    \begin{enumerate}
        \item Построить схему, изображенную на Рис. \ref{fig:s:1}, в программе Multisim.
        \item Пронаблюдать временные диаграммы входного и выходного сигналов.
    \end{enumerate}

    \subsection*{Эксперимент 2}

    \begin{enumerate}
        \item Построить схему, изображенную на Рис. \ref{fig:s:2}, в программе Multisim.
        \item Пронаблюдать временные диаграммы входного и выходного сигналов.
    \end{enumerate}

    \newpage

    \section*{Схемы}

    \begin{figure}[ht]
        \centering
        \includegraphics[scale=0.5]{images/s1.png}
        \caption{Схема 1}
        \label{fig:s:1}
    \end{figure}

    \begin{figure}[ht]
        \centering
        \includegraphics[scale=0.5]{images/s2.png}
        \caption{Схема 2}
        \label{fig:s:2}
    \end{figure}

    \newpage

    \section*{Выводы}

    \subsection*{Эксперимент 1}

    В ходе работы на экране осциллографа были получены следующие ВАХ:

    \begin{figure}[ht]
        \centering
        \includegraphics[scale=0.7]{images/o1.png}
        \caption{ Осциллограмма входного сигнала}
        \label{fig:o:1}
    \end{figure}

    \begin{figure}[ht]
        \centering
        \includegraphics[scale=0.7]{images/o2.png}
        \caption{Осциллограмма выходного сигнала}
        \label{fig:o:2}
    \end{figure}

    \newpage

    \subsection*{Эксперимент 2}

    В ходе работы на экране осциллографа была получена следующие ВАХ:

    \begin{figure}[ht]
        \centering
        \includegraphics[scale=0.7]{images/o3.png}
        \caption{Осциллограмма входного сигнала}
        \label{fig:o:3}
    \end{figure}

    \begin{figure}[ht]
        \centering
        \includegraphics[scale=0.7]{images/o4.png}
        \caption{Осциллограмма выходного сигнала}
        \label{fig:o:4}
    \end{figure}


\end{document}