\documentclass[12pt]{article}


\usepackage[english, russian]{babel}
\usepackage[T2A]{fontenc}
\usepackage[utf8]{inputenc}
\usepackage[left=2cm,right=2cm, top=1cm,bottom=1.5cm,bindingoffset=0cm]{geometry}

\usepackage{amsmath}

% \usepackage{multirow}
% \usepackage{hhline}

% \usepackage{indentfirst}

% \usepackage{enumitem,kantlipsum}

\usepackage{graphicx}
\graphicspath{{pictures/}}
\DeclareGraphicsExtensions{.pdf,.png,.jpg}

\usepackage{tikz}
\usetikzlibrary{patterns}
\usepackage{pgfplots}
\pgfplotsset{compat=1.9}
% \usepgfplotslibrary{fillbetween}

% \usepackage{ulem}

% \usepackage{hyperref}

% \usepackage{circuitikz}

% \usepackage{fp}
% \usepackage{xfp}

% \usepackage{siunitx}
% \sisetup{output-decimal-marker={,}}

% \usepackage{minted}

% \let\oldref\ref
% \renewcommand{\ref}[1]{(\oldref{#1})}

\begin{document}
    \pagestyle{empty}
    \begin{center}
        \textbf{Федеральное государственное автономное образовательное учреждение высшего образования}

        \vspace{5pt}

        {\small
        \textbf{САНКТ-ПЕТЕРБУРГСКИЙ НАЦИОНАЛЬНЫЙ}

        \textbf{ИССЛЕДОВАТЕЛЬСКИЙ УНИВЕРСИТЕТ ИТМО}

        \textbf{ФАКУЛЬТЕТ ПРОГРАММНОЙ ИНЖЕНЕРИИ И КОМПЬЮТЕРНОЙ ТЕХНИКИ}%
        }

        \vspace{140pt}

        {\Large
        \textbf{ЛАБОРАТОРНАЯ}

        \vspace{7pt}

        \textbf{РАБОТА №5}%
        }

        \vspace{10pt}

        {\large
        \textbf{Схемотехника ЭВМ}

        \vspace{5pt}

        \textbf{«Ждущий мультивибратор»}%
        }

        \vspace{170pt}

        \begin{tabular}{lll}
            Проверил:                                                                                   & \hspace{70pt} & Выполнил:                                             \\
            ........................                \rule[0.66\baselineskip]{2cm}{0.4pt}                &               & Студент группы P3455                                  \\
            «\rule[0.66\baselineskip]{1cm}{0.4pt}»  \rule[0.66\baselineskip]{2cm}{0.4pt} \the\year г.   &               & Федюкович С. А. \rule[0.66\baselineskip]{2cm}{0.4pt}  \\
            &               &                                                       \\
            Оценка          \hspace{12pt}           \rule[0.66\baselineskip]{2.7cm}{0.4pt}              &               &                                                       \\
        \end{tabular}

        \vspace*{\fill}

        Санкт-Петербург

        \the\year
    \end{center}

    \newpage

    \pagestyle{plain}
    \setcounter{page}{1}

    \section*{Цель}

    Экспериментально исследовать ждущий мультивибратор на операционных усилителях:

    \begin{enumerate}
        \item Используя пакет Multisim собрать схему ждущего мультивибратора.

        \item Исследовать схему мультивибратора и его зависимость длительности импульса от сопротивления.

        \item Составить отчет, в который должны быть включены: схема мультивибратора, таблица с данными и график.
    \end{enumerate}

    \section*{Задачи}

    \subsection*{Эксперимент 1}

    \begin{enumerate}
        \item Построить схему, изображенную на Рис. \ref{fig:s:1}, в программе Multisim.
        \item Измерить длительность импульса $T$ на выходе ждущего мультивибратора при помощи осциллографа.
        \item Изменить сопротивление резистора $R_2$ так, чтобы получить длительности равные $0.3T$, $0.5T$, $0.7T$, $0.85T$, $1.5T$, $2T$, $2.5T$, $3T$.
        \item Внести полученные данные в таблицу и построить график.
    \end{enumerate}

    \section*{Схемы}

    \begin{figure}[ht]
        \centering
        \includegraphics[scale=0.6]{images/s1.png}
        \caption{Схема 1}
        \label{fig:s:1}
    \end{figure}

    \newpage

    \section*{Выводы}

    \subsection*{Эксперимент 1}

    В ходе работы при помощи осциллографа было получено следующее значение $T$ (Рис. \ref{fig:o:1}):

    $$ T = 13.636 \text{ мс} $$

    \begin{figure}[ht]
        \centering
        \includegraphics[scale=0.75]{images/o1.png}
        \caption{Длительность импульса $T$ на выходе ждущего мультивибратора}
        \label{fig:o:1}
    \end{figure}

    \newpage

    Изменяя сопротивление резистора $R_2$ были получены следующие значения длительности импульса:

    \begin{table}[ht]
        \centering
        \begin{tabular}{|c|c|c|c|c|c|c|c|c|c|}
            \hline
            & $0.3T$ & $0.5T$ & $0.7T$ & $0.85T$ & $T$ & $1.5T$ & $2T$ & $2.5T$ & $3T$ \\
            \hline
            $T, \text{мс}$ & $4.090$ & $6.818$ & $9.545$ & $11.590$ & $13.636$ & $20.454$ & $27.272$ & $34.090$ & $40.908$ \\
            \hline
            $R_2, k\Omega$ & $15.000$ & $25.000$ & $36.000$ & $44.000$ & $52.000$ & $78.000$ & $105.000$ & $134.000$ & $159.000$ \\
            \hline
        \end{tabular}
    \end{table}

    \begin{figure}[ht]
        \centering
        \begin{tikzpicture}
            \begin{axis}[
            xmin = 0,
            ymin = 0,
            xmax = 45,
            ymax = 175,
            grid = both,
            axis lines=middle,
            axis line style={->},
            x label style={at={(axis description cs:0.5,-0.2)},anchor=east},
            xlabel = {$T\text{, мс}$},
            y label style={at={(axis description cs:-0.2,0.5)},anchor=north},
            ylabel = {$R_2, k\Omega$},
            width=10cm,
            height=10cm
            ]
            \addplot[
                mark=square,
            ]
            coordinates {
                (4.0908, 15)
                (6.818, 25)
                (9.5452, 36)
                (11.5906,44)
                (13.636, 52)
                (20.454, 78)
                (27.272, 105)
                (34.09, 134)
                (40.908, 159)
            };
            \end{axis}
        \end{tikzpicture}
    \end{figure}

\end{document}