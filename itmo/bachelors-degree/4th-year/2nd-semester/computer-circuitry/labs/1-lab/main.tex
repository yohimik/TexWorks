\documentclass[12pt]{article}


\usepackage[english, russian]{babel}
\usepackage[T2A]{fontenc}
\usepackage[utf8]{inputenc}
\usepackage[left=2cm,right=2cm, top=1cm,bottom=1.5cm,bindingoffset=0cm]{geometry}

\usepackage{amsmath}

% \usepackage{multirow}
% \usepackage{hhline}

% \usepackage{indentfirst}

% \usepackage{enumitem,kantlipsum}

\usepackage{graphicx}
\graphicspath{{pictures/}}
\DeclareGraphicsExtensions{.pdf,.png,.jpg}

\usepackage{tikz}
\usetikzlibrary{patterns}
\usepackage{pgfplots}
\pgfplotsset{compat=1.9}
% \usepgfplotslibrary{fillbetween}

% \usepackage{ulem}

% \usepackage{hyperref}

% \usepackage{circuitikz}

% \usepackage{fp}
% \usepackage{xfp}

% \usepackage{siunitx}
% \sisetup{output-decimal-marker={,}}

% \usepackage{minted}

% \let\oldref\ref
% \renewcommand{\ref}[1]{(\oldref{#1})}

\begin{document}
    \pagestyle{empty}
    \begin{center}
        \textbf{Федеральное государственное автономное образовательное учреждение высшего образования}

        \vspace{5pt}

        {\small
        \textbf{САНКТ-ПЕТЕРБУРГСКИЙ НАЦИОНАЛЬНЫЙ}

        \textbf{ИССЛЕДОВАТЕЛЬСКИЙ УНИВЕРСИТЕТ ИТМО}

        \textbf{ФАКУЛЬТЕТ ПРОГРАММНОЙ ИНЖЕНЕРИИ И КОМПЬЮТЕРНОЙ ТЕХНИКИ}%
        }

        \vspace{140pt}

        {\Large
        \textbf{ЛАБОРАТОРНАЯ}

        \vspace{7pt}

        \textbf{РАБОТА №1}%
        }

        \vspace{10pt}

        {\large
        \textbf{Схемотехника ЭВМ}

        \vspace{5pt}

        \textbf{«Полупроводниковый диод»}%
        }

        \vspace{170pt}

        \begin{tabular}{lll}
            Проверил:                                                                                   & \hspace{70pt} & Выполнил:                                             \\
            ........................                \rule[0.66\baselineskip]{2cm}{0.4pt}                &               & Студент группы P3455                                  \\
            «\rule[0.66\baselineskip]{1cm}{0.4pt}»  \rule[0.66\baselineskip]{2cm}{0.4pt} \the\year г.   &               & Федюкович С. А. \rule[0.66\baselineskip]{2cm}{0.4pt}  \\
            &               &                                                       \\
            Оценка          \hspace{12pt}           \rule[0.66\baselineskip]{2.7cm}{0.4pt}              &               &                                                       \\
        \end{tabular}

        \vspace*{\fill}

        Санкт-Петербург

        \the\year
    \end{center}

    \newpage

    \pagestyle{plain}
    \setcounter{page}{1}

    \section*{Цель}

    Изучить полупроводниковой диод:

    \begin{enumerate}
        \item Исследовать напряжение и силу тока диода при прямом и обратном смещении $p-n$ перехода.
        \item Построить и исследовать вольтамперную характеристику (ВАХ) для полупроводникового диода.
        \item Исследовать сопротивление диода при прямом и обратном смещении по вольтамперной характеристике.
        \item Провести анализ сопротивления диода (прямое и обратное смещение) на переменном и постоянном токе.
    \end{enumerate}

    \section*{Задачи}

    \subsection*{Эксперимент 1}

    \begin{enumerate}
        \item Построить схему, изображенную на Рис. \ref{fig:s:1}, в программе Multisim.
        \item Измерить напряжение на диоде при прямом смещении.
        \item Перевернуть диод, чтобы получилась схема, как на Рис. \ref{fig:s:2}.
        \item Измерить напряжение на диоде при обратном смещении.
        \item Вычислить силу тока диода при прямом и обратном смещении.
    \end{enumerate}

    \subsection*{Эксперимент 2}

    \begin{enumerate}
        \item Построить схему, изображенную на Рис. \ref{fig:s:1}, в программе Multisim.
        \item Измерить силу тока на диоде при прямом смещении.
        \item Перевернуть диод, чтобы получилась схема, как на Рис. \ref{fig:s:2}.
        \item Измерить силу тока на диоде при обратном смещении.
    \end{enumerate}

    \subsection*{Эксперимент 3}

    \begin{enumerate}
        \item Построить схему, изображенную на Рис. \ref{fig:s:1}, в программе Multisim.
        \item Измерить силу тока и напряжение на диоде при прямом смещении на разных значениях ЭДС (0; 0,5; 1; 2; 3; 4; 5), занести данные в таблицу и построить график.
        \item Перевернуть диод, чтобы получилась схема, как на Рис. \ref{fig:s:2}.
        \item Измерить силу тока на диоде при обратном смещении на разных значениях ЭДС (0; 0,5; 1; 2; 3; 4; 5), занести данные в таблицу и построить график.
    \end{enumerate}

    \newpage

    \section*{Схемы}

    \begin{figure}[ht]
        \centering
        \includegraphics[scale=0.5]{images/s1.jpg}
        \caption{Схема 1}
        \label{fig:s:1}
    \end{figure}

    \begin{figure}[ht]
        \centering
        \includegraphics[scale=0.5]{images/s2.jpg}
        \caption{Схема 2}
        \label{fig:s:2}
    \end{figure}

    \newpage

    \section*{Выводы}

    \subsection*{Эксперимент 1}

    В ходе работы были получены следующие данные:

    $$ U_{\text{об}} = 9V; I_{\text{об}} = \frac{E - U_{\text{об}}}{R} = \frac{9 - 9}{100} = 0A$$

    $$ U_{\text{пр}} = 769.17mV; I_{\text{пр}} = \frac{E - U_{\text{пр}}}{R} = \frac{9 - 0,76917}{100} = 0,0823083A$$

    \subsection*{Эксперимент 2}

    В ходе работы были получены следующие данные:

    $$ I_{\text{об}} = 18.010pA$$

    $$ I_{\text{пр}} = 82.308mA$$

    \subsection*{Эксперимент 3}

    В ходе работы при прямом смещении были получены следующие данные:

    \begin{table}[ht]
        \centering
        \begin{tabular}{|c|c|c|}
            \hline
            $E, V$ & $U_{\text{пр}}, mV$ & $I_{\text{пр}}, mA$ \\
            \hline
            0,000 & 0,000 & 0,000 \\
            \hline
            0,500 & 499,750 & 0,002 \\
            \hline
            1,000 & 684,790 & 3,152 \\
            \hline
            2,000 & 721,020 & 12,790 \\
            \hline
            3,000 & 735,790 & 22,642 \\
            \hline
            4,000 & 745,180 & 32,548 \\
            \hline
            5,000 & 752,060 & 42,479 \\
            \hline
        \end{tabular}
    \end{table}

    \begin{figure}[ht]
        \centering
        \begin{tikzpicture}
            \label{dsadsa}
            \begin{axis}[
            xmin = -20,
            ymin = -2,
            xmax = 800,
            ymax = 50,
            axis lines=left,
            axis line style={->},
            x label style={at={(axis description cs:0.7,-0.2)},anchor=east},
            xlabel = {$U_{\text{пр}}, mV$},
            y label style={at={(axis description cs:-0.2,0.5)},anchor=north},
            ylabel = {$I_{\text{пр}}, mA$},
            ]
            \addplot[
                mark=square,
            ]
            coordinates {
                (0,0)
                (499.75, 0.002)
                (684.79, 3.152)
                (721.02, 12.79)
                (735.79, 22.642)
                (745.18, 32.548)
                (752.06, 42.479)
            };
            \end{axis}
        \end{tikzpicture}
    \end{figure}

    \newpage

    При обратном смещении были получены следующие данные:

    \begin{table}[ht]
        \centering
        \begin{tabular}{|c|c|c|}
            \hline
            $E, V$ & $U_{\text{обр}}, V$ & $I_{\text{обр}}, pA$ \\
            \hline
            0,0 & 0,0 & 0,0 \\
            \hline
            0,500 & -0,500 & -1,010 \\
            \hline
            1,000 & -1,000 & -2,010 \\
            \hline
            2,000 & -2,000 & -4,010 \\
            \hline
            3,000 & -3,000 & -6,010 \\
            \hline
            4,000 & -4,000 & -8,010 \\
            \hline
            5,000 & -5,000 & -10,010 \\
            \hline
        \end{tabular}
    \end{table}

    \begin{figure}[ht]
        \centering
        \begin{tikzpicture}
            \label{dsadsa}
            \begin{axis}[
            xmin = -6,
            ymin = -11,
            xmax = 1,
            ymax = 1,
            axis lines=left,
            axis line style={->},
            x label style={at={(axis description cs:0.7,-0.2)},anchor=east},
            xlabel = {$U_{\text{обр}}, mV$},
            y label style={at={(axis description cs:-0.2,0.5)},anchor=north},
            ylabel = {$I_{\text{обр}}, mA$},
            ]
            \addplot[
                mark=square,
            ]
            coordinates {
                (0,0)
                (-0.5, -1.010)
                (-1 -2.01)
                (-2, -4.01)
                (-3, -6.01)
                (-4, -8.01)
                (-5, -10.01)
            };
            \end{axis}
        \end{tikzpicture}
    \end{figure}





\end{document}