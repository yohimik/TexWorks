\documentclass[14pt]{extarticle}
\usepackage[activate={true,nocompatibility},final,tracking=true,kerning=true,spacing=true,factor=1100,stretch=10,shrink=10]{microtype}
\usepackage[english, russian]{babel}
\usepackage[T2A]{fontenc}
\usepackage[utf8]{inputenc}
\usepackage[
    a4paper,
    left=3cm,
    right=2cm,
    top=2cm,
    bottom=3cm,
]{geometry}
\setlength{\parskip}{0.3cm}

\setlength{\parindent}{0cm}

\linespread{1.3}


\usepackage{caption}
\captionsetup[figure]{name={Рисунок}}

% \usepackage{amsmath}

% \usepackage{multirow}
% \usepackage{hhline}

\usepackage{indentfirst}

% \usepackage{enumitem,kantlipsum}

\usepackage{graphicx}
\graphicspath{{images/}}
\DeclareGraphicsExtensions{.pdf,.png,.jpg}

\usepackage{tikz}
% \usetikzlibrary{patterns}
% \usepackage{pgfplots}
% \pgfplotsset{compat=1.9}
% \usepgfplotslibrary{fillbetween}

% \usepackage{ulem}

% \usepackage{hyperref}

% \usepackage{circuitikz}

% \usepackage{fp}
% \usepackage{xfp}

% \usepackage{siunitx}
% \sisetup{output-decimal-marker={,}}

\usepackage{amsmath}

\usepackage{minted}
\setminted[javascript]{
    breaklines=true,
    encoding=utf8,
}
\usepackage{etoolbox}
\makeatletter
\AtBeginEnvironment{minted}{\dontdofcolorbox}
\def\dontdofcolorbox{\renewcommand\fcolorbox[4][]{##4}}
\makeatother

% \let\oldref\ref
% \renewcommand{\ref}[1]{(\oldref{#1})}

\begin{document}
    \newgeometry{
        left=3cm,
        right=2cm,
        top=2cm,
        bottom=1.5cm,
    }
    \thispagestyle{empty}
    \begin{center}
        МИНИСТЕРСТВО ОБРАЗОВАНИЯ И НАУКИ РОССИЙСКОЙ ФЕДЕРАЦИИ

        САНКТ-ПЕТЕРБУРГСКИЙ НАЦИОНАЛЬНЫЙ ИССЛЕДОВАТЕЛЬСКИЙ УНИВЕРСИТЕТ ИТМО

        ФАКУЛЬТЕТ ПРОГРАММНОЙ ИНЖЕНЕРИИ И КОМПЬЮТЕРНОЙ ТЕХНИКИ

        КАФЕДРА АППАРАТНО-ПРОГРАММНЫХ КОМПЛЕКСОВ ВЫЧИСЛИТЕЛЬНОЙ ТЕХНИКИ

        \vspace{2cm}

        \textbf{Курсовая работа}

        по дисциплине

        «Моделирование»

        на тему

        «Компьютерный расчёт показателей разомкнутой системы массового обслуживания с неограниченным временем ожидания»

        \vspace{2cm}

        \begin{flushright}
            Выполнил: студент группы P3455

            Федюкович С. А.,\par
            \rule[0.66\baselineskip]{3cm}{0.4pt}

            \par

            Проверил: Сентерев Ю. А.

            \par\par

            \rule[0.66\baselineskip]{3cm}{0.4pt}
        \end{flushright}

        \vspace*{\fill}

        Санкт-Петербург

        \the\year г.
    \end{center}

    \newpage
    \restoregeometry
    \pagestyle{plain}
    \setcounter{page}{1}

    \tableofcontents

    \newpage

    \section*{Задание}
    \addcontentsline{toc}{section}{Задание}

    В пункте химчистки имеется три аппарата для чистки, $r = 3$.   Интенсивность потока посетителей $\lambda  = 6$ (чел./ ч). Интенсивность обслуживания посетителей одним аппаратом $\mu = 3$ (чел./ч). Среднее число посетителей, покидающих очередь, не дождавшись обслуживания, $\nu = 1$ (чел./ ч). Найти абсолютную пропускную способность пункта химчистки.

    \newpage

    \section*{Введение}
    \addcontentsline{toc}{section}{Введение}

    Теория массового обслуживания --- область прикладной математики, занимающаяся анализом процессов в системах производства, обслуживания, управления, в которых однородные события повторяются многократно, например, на предприятиях бытового обслуживания; в системах приема, переработки и передачи информации; автоматических линиях производства и др.

    Предметом теории массового обслуживания является установление зависимостей между характером потока заявок, числом каналов обслуживания, производительностью отдельного канала и эффективным обслуживанием с целью нахождения наилучших путей управления этими процессами. Задачи теории массового обслуживания носят оптимизационный характер и в конечном итоге включают экономический аспект по определению такого, варианта системы, при котором будет обеспечен минимум суммарных затрат от ожидания обслуживания, потерь времени и ресурсов на обслуживание и от простоев каналов обслуживания.

    \newpage

    \section{Теоретические аспекты СМО}

    \subsection{Системы массового обслуживания с отказами}

    СМО с отказами является такая система, в которой приходящие для обслуживания требования, в случае занятости всех каналов обслуживания, сразу её покидают.

    Вероятности состояний системы определяются из Выражения (1):
    \begin{equation} P_k = \frac{\rho^k}{k!} P_0, \end{equation}
    $$ \text{ где } k = 1, 2, ..., N \text{ --- общее число каналов;}$$
    \begin{equation} \rho = \frac{\lambda}{\nu} \text{ --- нагрузка; } \end{equation}
    $\lambda$ --- интенсивность входящего потока требований, $\nu$ --- интенсивность (производительность) одного канала (прибора) обслуживания, а вероятность отсутствия требований из Выражения (3):
    \begin{equation} P_0 = \Bigg[ \sum_{ i = 0 }^N \frac{\rho^i}{i!} \Bigg]^{-1}. \end{equation}

    К основным характеристикам качества обслуживания рассматриваемой СМО относятся:
    \begin{enumerate}
        \item Вероятность отказа $P_\text{отк}$ из Выражения (4):
        \begin{equation}
            P_\text{отк} = P_N = \frac{\rho^N / N!}{\sum_{i = 0}^N \rho^i / i!}.
        \end{equation}

        \item Среднее число занятых узлов обслуживания $M_\text{зан}$ из Выражения (5):
        \begin{equation}
            M_\text{зан} = \rho (1 - P_N)
        \end{equation}

        \item Среднее число свободных узлов обслуживания $M_\text{св}$ из Выражения (6):
        \begin{equation}
            M_\text{св} = N - M_\text{зан}.
        \end{equation}

        \item В системах с отказами события отказа и обслуживания составляют полную группу событий из Выражения (7):
        \begin{equation}
            P_\text{отк} + P_\text{обс} = 1.
        \end{equation}

        \item Относительная пропускная способность определяется по Выражению (8):
        \begin{equation}
            Q = P_\text{обс} = 1 - P_\text{отк} = 1 - P_N.
        \end{equation}

        \item Абсолютная пропускная способность СМО с отказами считается по Выражению (9):
        \begin{equation}
            A = \lambda P_\text{обс}.
        \end{equation}

        \item Коэффициент занятости узлов обслуживания определяется отношением средним числом занятых каналов к общему числу каналов по Выражению (10):
        \begin{equation}
            K_\text{з} = \frac{M_\text{зан}}{N}.
        \end{equation}
    \end{enumerate}

    \subsection{Система массового обслуживания с ограниченной длиной очереди}

    СМО с ограниченной длиной очереди является такой системой, в которой требование, поступающее на обслуживание, покидает систему, если заняты все каналы обслуживания, и в накопителе заняты все места.

    Вероятности состояний $S_0$, $S_1$, ... , $S_N$ находятся по Выражению (11):
    \begin{equation}
        P_k = \frac{\rho^k}{k!}\text{, где }(k = 1, 2, ..., N).
    \end{equation}

    А вероятности состояний $S_{N+1}$, $S_{N+2}$, ..., $S_{N + l}$ находятся  по Выражению (12):
    \begin{equation}
        P_k = \frac{\rho^k}{N^{k - N} N!}\text{, где }(k = N + 1, ..., N + l), l\text{ --- максимальная длина очереди}.
    \end{equation}

    Вероятность $P_0$ рассчитывается  по Выражению (13):
    \begin{equation}
        P_0 = \Bigg[ \sum_{k = 0}^N \frac{\rho^k}{k!} + \sum_{k = N + 1}^{N + l} \frac{\rho^i}{N^{k - N} N!} \Bigg]^{-1}
    \end{equation}

    В большинстве практических задач должно соблюдаться отношение $\frac{\rho}{N} < 1$, тогда выражение для $P_0$ можно переписать в следующем виде  по Выражению (14):
    \begin{equation}
        P_0 = \Bigg[ \sum_{k = 0}^N \frac{\rho^k}{k!} + \frac{\rho^{N+1}}{N \cdot N!} \frac{1 - (\frac{\rho}{N})^l}{1 - \frac{\rho}{N}} \Bigg]^{-1}.
    \end{equation}

    Вероятность отказа в обслуживании определяется из Выражения (15):
    \begin{equation}
        P_{\text{отк}} = P_{N+l} = \frac{\rho^N}{N!} \Bigg( \frac{\rho}{N} \Bigg)^l P_0.
    \end{equation}

    Среднее число каналов, занятых в обслуживании, и коэффициент занятости определяются  по Выражению (16):
    \begin{equation}
        M_{\text{зан}} = \sum_{k = 1}^N k P_k + N \sum_{i = 1}^l P_{N + i}; K_{\text{з}} = \frac{M_{\text{зан}}}{N}.
    \end{equation}

    Среднее число свободных аппаратов и коэффициент простоя определяются  по Выражению (17):
    \begin{equation}
        M_0 = N - M_{\text{зан}}; K_0 = \frac{M_0}{N}.
    \end{equation}

    Средняя длина очереди определяется с помощью Выражения (18):
    \begin{equation}
        M_{\text{оч}} = \sum_{k = 1}^l k P_{N + k} = \frac{\rho^N}{N!} P_0 \sum_{k = 1}^l k \Bigg( \frac{\rho}{N} \Bigg)^k.
    \end{equation}

    \subsection{Системы массового обслуживания с ожиданием}

    СМО с ожиданием аналогична системе массового обслуживания с ограниченной длиной очереди при условии, что граница очереди отодвигается в бесконечность.

    Вероятность состояний СМО с ожиданием находят по формулам Выражений (19) и (20):
    \begin{equation}
        P_k = \frac{\rho^k}{k!} P_0\text{, для }(k = 1, 2, ..., N),
    \end{equation}
    \begin{equation}
        P_k = \frac{\rho^k}{N! N^{k - N}} P_0\text{, для } (k = N + 1, ..., N + k, ..., N + \infty).
    \end{equation}

    При $\rho / N > 1$ наблюдается явление «взрыва» --- неограниченный рост средней длины очереди, поэтому для определения $P_0$ должно выполняться ограничивающее условие $ \rho / N < 1 $, и с учетом его запишем Выражение (21):
    \begin{equation}
        P_0 = \Bigg[ \sum_{k = 0}^N \frac{\rho^k}{k!} + \frac{\rho^{N + 1}}{N!(N - \rho)}  \Bigg]^{-1}.
    \end{equation}

    К основным характеристикам качества обслуживания СМО с ожиданием относят:
    \begin{enumerate}
        \item Вероятность наличия очереди $P_\text{оч}$, т.е. вероятность того, что число требований в системе больше числа узлов по Выражению (22):
        \begin{equation}
            P_\text{оч} = \frac{\rho^{N + 1}}{N!(N - \rho)} P_0.
        \end{equation}

        \item Вероятность занятости всех узлов системы $P_\text{зан}$ по Выражению (23):
        \begin{equation}
            P_\text{зан} = \frac{\rho^N}{(N - 1)!(N - \rho)} P_0.
        \end{equation}

        \item Среднее число требований в системе $M_\text{ТР}$ по Выражению (24):
        \begin{equation}
            M_\text{ТР} = P_0 \Bigg( \rho \sum_{k = 0}^{N - 1} \frac{\rho^k}{k!} + \frac{\rho^{N + 1} (N + 1 - \rho)}{(N - 1)!(N - \rho)^2} \Bigg).
        \end{equation}

        \item Средняя длина очереди $M_\text{оч}$ по Выражению (25):
        \begin{equation}
            M_\text{оч} = \frac{\rho^{N + 1} P_0}{(N - 1)!(N - \rho)^2}.\end{equation}

        \item Среднее число свободных каналов обслуживания $M_\text{св}$ по Выражению (26):
        \begin{equation}
            M_\text{св} = P_0 \sum_{k = 1}^N k \frac{\rho^k}{(N - k)!}.
        \end{equation}

        \item Среднее число занятых каналов обслуживания $M_\text{зан}$ по Выражению (27):
        \begin{equation}
            M_\text{зан} = N - M_\text{св}.
        \end{equation}

        \item Коэффициент простоя $K_0$ и коэффициент загрузки $K_\text{з}$ каналов обслуживания системы по Выражению (28):
        \begin{equation}
            K_0 = \frac{M_\text{св}}{N}; K_\text{з} = \frac{M_\text{зан}}{N}.
        \end{equation}

        \item Среднее время ожидания начала обслуживания $T_\text{ож}$ для требования, поступившего в систему по Выражению (29):
        \begin{equation}
            T_\text{ож} = \frac{\rho^N}{\mu (N - 1)!(N - \rho)^2} P_0.
        \end{equation}

        \item Общее время, которое проводят в очереди все требования, поступившие в систему за единицу времени $T_\text{оож}$ по Выражению (30):
        \begin{equation}
            T_\text{оож} = \frac{\rho^{N + 1}}{(N - 1)!(N - \rho)^2} P_0.
        \end{equation}

        \item Среднее время $T_\text{тр}$, которое требование проводит в системе обслуживания  по Выражению (31):
        \begin{equation}
            T_\text{тр} = T_\text{ож} + \mu^{-1}.
        \end{equation}

        \item Суммарное время, которое в среднем проводят в системе все требования, поступившие за единицу времени $T_\text{стр}$  по Выражению (32):
        \begin{equation}
            T_\text{стр} = T_\text{оож} + \rho.
        \end{equation}
    \end{enumerate}

    \subsection{Система массового обслуживания с ограниченным временем ожидания}

    В системах массового обслуживания с ограниченным временем ожидания время ожидания в очереди каждого требования ограничено случайной величиной $ t_\text{ож} $, среднее значение которого $ \overline{t_\text{ож}} $.

    Величина, обратная среднему времени ожидания, означает среднее количество требований, покидающих очередь в единицу времени, вызванное появлением в очереди одного требования: $ \nu = 1\ /\  \overline{t_\text{ож}} $.

    При наличии в очереди $k$ требований интенсивность потока покидающих очередь требований составляет $ k \nu $.

    Для дальнейшего рассмотрения СМО с ограниченным временем ожидания введём новый параметр $\beta = \frac{\nu}{\mu}$, означающий среднее число требований, покидающих систему не обслуженными,  приходящиеся на среднюю скорость обслуживания требований.

    Формулы для определения вероятностей состояний такой системы имеют вид  по Выражениям (33) и (34):
    \begin{equation}
        P_k = \frac{\rho^k}{k!} \cdot P_0\text{, при }k = 1, 2,..., N;
    \end{equation}
    \begin{equation}
        P_{N + l} = P_N \cdot \frac{\rho^l}{\prod_{i = 1}^l (N + i \beta) } \text{, при }i = N + 1,...,N + l.
    \end{equation}

    Вероятность $P_0$ определяют по Выражению (35):
    \begin{equation}
        P_0 = \Bigg[ \sum_{k = 0}^N \frac{\rho^k}{k!} + \frac{\rho^N}{N!} \sum_{l = 1}^\infty \frac{\rho^l}{\prod_{i = 1}^l (N + i \beta)} \Bigg]^{-1},
    \end{equation}
    $$
    \text{где } \prod_{i = 1}^l ( N + i \beta) \text { --- произведение сомножителей } N + i \beta .
    $$
    В практических задачах сумму бесконечного ряда вычислить достаточно просто, так как члены ряда быстро убывают с увеличением номера.

    Средняя длина очереди по Выражению (36):
    \begin{equation}
        M_\text{оч} = \frac{\rho^N}{N!} P_0 \sum_{l = 1}^\infty l \frac{\rho^l}{\prod_{i = 1}^l (N + i \beta)}.
    \end{equation}

    Вероятность отказа по Выражению (37):
    \begin{equation}
        P_\text{отк} = \frac{\beta}{\rho} M_\text{оч}.
    \end{equation}

    Среднее число занятых каналов обслуживания и коэффициент загрузки по Выражениям (38) и (39):
    \begin{equation}
        M_\text{зан} = \sum_{k = 1}^N k P_k + N \sum_{l = 1}^\infty P_{N + l};
    \end{equation}
    \begin{equation}
        K_\text{з} = \frac{M_\text{зан}}{N}.
    \end{equation}

    Среднее число свободных каналов обслуживания и коэффициент простоя по Выражению (40):
    \begin{equation}
        M_\text{св} = N - M_\text{зан}; K_0 = \frac{M_\text{зан}}{N}.
    \end{equation}

    Относительная пропускная способность по Выражению (41):
    \begin{equation}
        Q = 1 - P_\text{отк}.
    \end{equation}

    \newpage
    \section{Практическая реализация СМО}

    \subsection{Решение задачи}

    Сперва необходимо решить поставленную в работе задачу на бумаге.

    Имеем: $ r = 3, \lambda = 6, \mu = 3, \nu = 1 $. Находим: $ \rho = \lambda / \mu = 6 / 3 = 2$,
    $$ P_0 = \Bigg[ 1 + \frac{2}{1!} + \frac{2^2}{2!} + \frac{2^3}{3!} + \frac{2^3}{3!} \cdot 1,219 \Bigg]^{-1} \approx 0,126. $$

    Вероятность занятости всех приборов равна $P_\text{зан} = 1 - P_0 = 0,874$. Тогда абсолютная пропускная способность может быть получена как произведение: $ A = NP_\text{зан} = 3 \cdot 0,874 = 2,622 $. Таким образом, $ A =  2,622 $ посетителя в час.

    \subsection{Расчёт показателей}

    Следующим шагом рассчитаем все основные показатели СМО для последующей проверки:
    \begin{enumerate}
        \item Вероятности состояний системы:
        $$ P_1 = \frac{2^1}{1!} \cdot 0,126 = 0,252;\ \rho_1 = \frac{0,252 \cdot 1!}{0,126} = 2; $$
        $$ P_2 = \frac{2^2}{2!} \cdot 0,126 = 0,252;\ \rho_1 = \frac{0,252 \cdot 2!}{0,126} = 4; $$
        $$ P_3 = \frac{2^3}{3!} \cdot 0,126 = 0,168;\ \rho_1 = \frac{0,168 \cdot 3!}{0,126} = 8. $$

        \item Средняя длина очереди:
        $$ M_\text{оч} = \frac{2^3}{3!} \cdot 0,126 \cdot 2,343 \approx 0,392 $$

        \item Среднее число требований:
        $$ \beta = \frac{1}{3} $$

        \item Вероятность отказа:
        $$ P_\text{отк} = \frac{1}{3} \cdot \frac{1}{2} \cdot 0,392 = 0,065 $$

        \item Среднее число занятых каналов обслуживания и коэффициент загрузки:
        $$ M_\text{зан} = 1 \cdot 0,252 + 2 \cdot 0,252 + 3 \cdot 0,168 + 3 \cdot 0,461 \approx 1,869; $$
        $$ K_\text{З} = \frac{1,721}{3} \approx 0,623. $$

        \item Среднее число свободных каналов обслуживания и коэффициент простоя:
        $$ M_\text{св} = 3 - 1,869 = 1,131; $$
        $$ K_0 = \frac{1,721}{3} = K_\text{З} = 0,623.$$

        \item Относительная пропускная способность:
        $$ Q = 1 - P_\text{отк} = 0,935. $$
    \end{enumerate}

    \subsection{Расчёт показателей при помощи электронных таблиц}

    Электронная таблица --- компьютерная программа, позволяющая проводить вычисления с данными, представленными в виде двумерных массивов, имитирующих бумажные таблицы.

    В данной работе в качестве электронной таблицы будет использоваться Google Sheets.

    Для расчёта была создана новая таблица и введена информация о характеристиках СМО в колонки на Рисунке \ref{fig:chars}:
    \begin{figure}[!ht]
        \centering
        \includegraphics[scale=2]{f1.png}
        \caption{Характеристики СМО}
        \label{fig:chars}
    \end{figure}

    Следующим шагом рассчитаем значение $\rho$, введя нужную формулу в ячейку $E2$ на Рисунке \ref{fig:rho}:
    \begin{figure}[!ht]
        \centering
        \includegraphics[scale=2]{f2.png}
        \caption{Расчёт $\rho$ СМО}
        \label{fig:rho}
    \end{figure}

    Дальше рассчитаем значение $\beta$, введя формулу в ячейку $E3$ на Рисунке \ref{fig:beta}:
    \begin{figure}[!ht]
        \centering
        \includegraphics[scale=2]{f4.png}
        \caption{Расчёт $\beta$ СМО}
        \label{fig:beta}
    \end{figure}

    После пишем формулу для вероятности того, что все каналы свободны на Рисунке \ref{fig:p_0}:
    \begin{figure}[!ht]
        \centering
        \includegraphics[scale=2]{f5.png}
        \caption{Расчёт $P_0$ СМО}
        \label{fig:p_0}
    \end{figure}

    В данном случае функция $P\_0$ является макросом, написанным на языке JavaScript, код которого приведён ниже.

    Находим вероятность того, что все каналы заняты на Рисунке \ref{fig:p_b}:
    \begin{figure}[!ht]
        \centering
        \includegraphics[scale=2]{f6.png}
        \caption{Расчёт $P_\text{зан}$ СМО}
        \label{fig:p_b}
    \end{figure}

    В итоге, находим абсолютную пропускную способность пункта химчистки на Рисунке \ref{fig:a}:
    \begin{figure}[!ht]
        \centering
        \includegraphics[scale=2]{f7.png}
        \caption{Расчёт $A$ СМО}
        \label{fig:a}
    \end{figure}

    Находим вероятности состояний системы, написав формулу один раз и перенеся её на три строчки ниже, на Рисунке \ref{fig:P_k}:
    \begin{figure}[!ht]
        \centering
        \includegraphics[scale=2]{f8.png}
        \caption{Расчёт вероятностей СМО}
        \label{fig:P_k}
    \end{figure}

    Аналогично добавим формулы вероятностей на Рисунке \ref{fig:p_k}:
    \begin{figure}[!ht]
        \centering
        \includegraphics[scale=2]{f9.png}
        \caption{Расчёт вероятностей СМО}
        \label{fig:p_k}
    \end{figure}

    Дальше добавим формулу $M_\text{оч}$ на Рисунке \ref{fig:p_m_q}:
    \begin{figure}[!ht]
        \centering
        \includegraphics[scale=2]{f10.png}
        \caption{Расчёт $M_\text{оч}$ СМО}
        \label{fig:p_m_q}
    \end{figure}

    \newpage

    Следующим шагом рассчитаем вероятность отказа $P_\text{отк}$ на Рисунке \ref{fig:p_c}:
    \begin{figure}[!ht]
        \centering
        \includegraphics[scale=2]{f11.png}
        \caption{Расчёт $P_\text{отк}$ СМО}
        \label{fig:p_c}
    \end{figure}

    После рассчитаем $M_\text{зан}$ на Рисунке \ref{fig:m_b}:
    \begin{figure}[!ht]
        \centering
        \includegraphics[scale=2]{f12.png}
        \caption{Расчёт $M_\text{зан}$ СМО}
        \label{fig:m_b}
    \end{figure}

    Дальше найдём коэффициент загрузки $K_\text{з}$ на Рисунке \ref{fig:k_b}:
    \begin{figure}[!ht]
        \centering
        \includegraphics[scale=2]{f13.png}
        \caption{Расчёт $K_\text{з}$ СМО}
        \label{fig:k_b}
    \end{figure}

    \newpage

    Последней характеристикой рассчитаем относительную пропускную способность $Q$ на Рисунке \ref{fig:q}:
    \begin{figure}[!ht]
        \centering
        \includegraphics[scale=2]{f15.png}
        \caption{Расчёт $Q$ СМО}
        \label{fig:q}
    \end{figure}

    В итоге, получаем таблицу со всеми характеристиками СМО на Рисунке \ref{fig:all-chars}:
    \begin{figure}[!ht]
        \centering
        \includegraphics[scale=2]{f16.png}
        \caption{Таблица с характеристиками СМО}
        \label{fig:all-chars}
    \end{figure}

    Код макроса для расчёта характеристик:
    \begin{minted}{javascript}
/** @OnlyCurrentDoc */

function factorial (n) {
  if (n <= 1)
    return 1;
  return factorial(n - 1) * n;
}

function P_0(rho, n, beta) {
  return 1 / (
      Array(n + 1).fill(0).reduce(
        (res, v, k) => res + Math.pow(rho, k) / factorial(k), 0)
       + (
        Math.pow(rho, n) / factorial(n))
      * Array(1000).fill(0).reduce(
        (res, v, l) => res + Math.pow(rho, l + 1) / Array(l + 1).fill(0).reduce((resi, v, i) => resi * ( n + (i + 1) * beta), 1),
      0)
  );
}

function M_Q(rho, n, p_0, beta) {
  return (Math.pow(rho, n) / factorial(n)) * p_0 * Array(1000).fill(0).reduce(
    (res, v, l) => res + (l + 1) * Math.pow(rho, l + 1) / Array(l + 1).fill(0).reduce((resi, v, i) => resi * ( n + (i + 1) * beta), 1),
  0);
}

function M_B(p_n, rho, beta) {
  const n = p_n.length ? p_n.length * p_n[0].length : 0;
  if (!n) return 0;

  return p_n.reduce((res, p_k, k) => res + (k + 1) * p_k[0], 0) + n * Array(1000).fill(0).reduce(
    (res, v, l) => res + p_n[p_n.length - 1][0] * Math.pow(rho, l + 1) / Array(l + 1).fill(0).reduce((resi, v, i) => resi * ( n + (i + 1) * beta), 1),
  0);
}
    \end{minted}

    Все найденные характеристики сходятся с вычислительными значениями, а это значит, что всё выполнено верно.

    \newpage
    \section*{Заключение}
    \addcontentsline{toc}{section}{Заключение}

    Теория массового обслуживания для химчистки --- это, по сути, средство анализа затрат. Для отдела было бы непомерно дорого или свидетельствовало бы о том, что у них не так много клиентов, чтобы никому из их клиентов никогда не приходилось стоять в очереди. В качестве упрощенного примера, для химчистки, чтобы исключить обстоятельства, когда людям приходится стоять в очереди, чтобы постирать одежду. Таким образом, химчистка может использовать информацию, полученную из теории очередей, чтобы настроить свои операционные функции таким образом, чтобы найти баланс между стоимостью обслуживания клиентов и неудобствами для клиентов, вызванными простоями в очереди.

    Работа химчистке отличается достаточно большим потоком информации от клиентов и сотрудников, которая достаточно быстро распределяется. При помощи имитационного моделирования, осуществлен расчёт основных характеристик СМО.

    \newpage
    \section*{Список литературы}
    \addcontentsline{toc}{section}{Список литературы}

    \begin{enumerate}
        \item Головко Николай Иванович. Исследование моделей систем массового обслуживания в информационных сетях : диссертация ... доктора технических наук : 05.13.18 / Головко Николай Иванович; [Место защиты: Ин-т автоматики и процессов управления ДВО РАН].- Владивосток, 2007.- 404 с.;
        \item Климов Г.П. Теория массового обслуживания. / 2-е издание, переработанное. – М.: Издательство Московского университета. – 2011. – 312с.
        \item Mosquera, A., Olarte Pascual, C. y Juaneda Ayensa , E. (2017): Under-standing the customer experience in the age of omni-channel shopping, Icono 14, volumen 15 (2), pp. 235-255. doi: 10.7195/ri14.v15i2.1070;
        \item Ngai, E. W. T., Moon, K. K., Lam, S. S., Chin, E. S. K., \& Tao, S. S. C. (2015). Social media models, technologies, and applications. Industrial Management \& Data Systems, 115(5), 769–802;
        \item Кетокиви  М.А.,  Шредер  Р.Г.  (2004)  Стратегические,  структурные непредвиденные обстоятельства и институциональные объяснения в принятии инновационных методов производства. Дж Опер Манаг 22: 63–89;
        \item Pourkhani, A. \& Abdipour, Khadije \& Baher, B. \& Moslehpour, Massoud. (2019). The impact of social media in business growth and performance: A scien-tometrics analysis. International Journal of Data and Network Science;
        \item Астахова Н.И. Менеджмент : учебник для прикладного бакалавриата / Н. И. Астахова [и др.] ; ответственный редактор Н. И. Астахова, Г. И. Моск-витин. — Москва : Издательство Юрайт, 2020. — 422 с;
    \end{enumerate}

\end{document}