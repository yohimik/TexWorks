\documentclass[12pt]{article}


\usepackage[english, russian]{babel}
\usepackage[T2A]{fontenc}
\usepackage[utf8]{inputenc}
\usepackage[left=2cm,right=2cm, top=1cm,bottom=1.5cm,bindingoffset=0cm]{geometry}

\usepackage{amsmath}

% \usepackage{multirow}
% \usepackage{hhline}

\usepackage{indentfirst}

% \usepackage{enumitem,kantlipsum}

% \usepackage{graphicx}
% \graphicspath{{pictures/}}
% \DeclareGraphicsExtensions{.pdf,.png,.jpg}

\usepackage{tikz}
\usetikzlibrary{patterns}
\usepackage{pgfplots}
\pgfplotsset{compat=1.9}
% \usepgfplotslibrary{fillbetween}

% \usepackage{ulem}

% \usepackage{hyperref}

% \usepackage{circuitikz}

% \usepackage{fp}
% \usepackage{xfp}

% \usepackage{siunitx}
% \sisetup{output-decimal-marker={,}}

% \usepackage{minted}

% \let\oldref\ref
% \renewcommand{\ref}[1]{(\oldref{#1})}

\begin{document}
    \pagestyle{empty}
    \begin{center}
        \textbf{Федеральное государственное автономное образовательное учреждение высшего образования}

        \vspace{5pt}

        {\small
        \textbf{САНКТ-ПЕТЕРБУРГСКИЙ НАЦИОНАЛЬНЫЙ}

        \textbf{ИССЛЕДОВАТЕЛЬСКИЙ УНИВЕРСИТЕТ ИТМО}

        \textbf{ФАКУЛЬТЕТ ПРОГРАММНОЙ ИНЖЕНЕРИИ И КОМПЬЮТЕРНОЙ ТЕХНИКИ}%
        }

        \vspace{140pt}

        {\Large
        \textbf{РЕФЕРАТ ПО ДИСЦИПЛИНЕ}

        \vspace{7pt}

        \textbf{ЭКОНОМИКА}%
        }

        \vspace{10pt}

        {\large
        \textbf{«Чистая монополия»}

        \vspace{5pt}

        \textbf{}%
        }

        \vspace{170pt}

        \begin{tabular}{lll}
            Проверил:                                                                                   & \hspace{70pt} & Выполнил:                                             \\
            ........................                \rule[0.66\baselineskip]{2cm}{0.4pt}                &               & Студент группы P3455                                  \\
            «\rule[0.66\baselineskip]{1cm}{0.4pt}»  \rule[0.66\baselineskip]{2cm}{0.4pt} \the\year г.   &               & Федюкович С. А. \rule[0.66\baselineskip]{2cm}{0.4pt}  \\
            &               &                                                       \\
            Оценка          \hspace{12pt}           \rule[0.66\baselineskip]{2.7cm}{0.4pt}              &               &                                                       \\
        \end{tabular}

        \vspace*{\fill}

        Санкт-Петербург

        \the\year
    \end{center}

    \newpage

    \pagestyle{plain}
    \setcounter{page}{1}

    \section*{Сущность чистой монополии}

    Чистая монополия (от греч. ``моно'' --- один, ``полио'' --- продаю) является противоположностью совершенной конкуренции. Данная рыночная структура состоит из одной фирмы или отрасли, т.е. понятия ``фирма'' и ``отрасль'' совпадают.

    Абсолютная, или чистая монополия возникает в том случае, когда одна фирма становится единственным производителем продукта, у которого нет близких заменителей, или субститутов. Чистая монополия характеризуется целым рядом специфических черт.

    Предприятие-монополист олицетворяет собой целую отрасль, так как последняя представлена всего одним предприятием, единственным поставщиком данного товара. Поэтому то, что свойственно поведению предприятия, характерно и для отрасли. Именно в этом смысле чистая монополия и занимает противоположную позицию относительно чистой конкуренции.

    Продукт монополии уникален в том смысле, что не существует адекватных или близких заменителей монополизированного продукта. Перед покупателем стоит проблема выбора. Он вынужден либо покупать продукт у монополиста, либо обходиться без него. В отличии от предприятий, действующих в условиях чистой конкуренции, предприятие-монополист осуществляет значительный контроль над ценой. При нисходящей кривой спроса монополист может изменять цены на продукт путем манипуляции количеством его предложения.

    В условиях чистой монополии общий уровень средних издержек производства может быть и ниже, и выше, чем в условиях совершенной конкуренции. Ниже он становится, в частности, благодаря использованию эффекта масштабов производства, что не под силу мелким капиталам.

    Однако в любом варианте издержки производства чистой монополии выше средних общих издержек.

    Прежде всего, во всех моделях рынков несовершенной конкуренции средние издержки производства выше минимальных средних издержек, уровень которых может достигаться лишь фирмами, имеющими минимально эффективные размеры масштаба производства (см. Рис. \ref{fig:costs})

    \begin{figure}[h]
        \centering
        \begin{tikzpicture}
            \begin{axis}[
            ticks=none,
            xmin = -1,
            ymin = -1,
            xmax = 10,
            ymax = 10,
            axis lines=middle,
            axis line style={->},
            x label style={at={(axis description cs:1,0.14)},anchor=east},
            xlabel = $Q$,
            y label style={at={(axis description cs:0.16,1)},anchor=north},
            ylabel = {$AC$},
            width=10cm,
            height=10cm,
            ]
            \node[label={200:{0}},circle,fill,inner sep=1pt] at (axis cs:0,0) {};

            \addplot [
                domain=2:9,
                samples=200,
                color=black,
            ]
            {7 - sqrt(- x^2 + 11 * x - 18)};

            \node[label={60:{$AC_{opt}$}},circle,fill,inner sep=1pt] at (axis cs:0,3.5){};
            \node[label={90:{A}},circle,fill,inner sep=1pt] at (axis cs:5.5,3.5) {};
            \addplot[thick, dashed] plot coordinates {(0, 3.5)  (5.5, 3.5)};
            \addplot[thick, dashed] plot coordinates {(5.5, 3.5) (5.5, 0)};
            \node[label={60:{$Q_{opt}$}},circle,fill,inner sep=1pt] at (axis cs:5.5, 0){};

            \node[label={60:{$AC_{mon}$}},circle,fill,inner sep=1pt] at (axis cs:0,5.011) {};
            \node[label={60:{B}},circle,fill,inner sep=1pt] at (axis cs:2.62,5.011) {};
            \addplot[thick, dashed] plot coordinates {(0, 5.011)  (2.62, 5.011)};
            \addplot[thick, dashed] plot coordinates {(2.62, 5.011) (2.62, 0)};
            \node[label={60:{$Q_{mon}$}},circle,fill,inner sep=1pt] at (axis cs:2.62,0) {};
            \end{axis}
        \end{tikzpicture}
        \caption{Издержки производства}
        \label{fig:costs}
    \end{figure}

    На Рис. \ref{fig:costs} точка $A$ фиксирует средние общие издержки фирмы, имеющей оптимальные масштабы производства, а точка $B$ показывает средние общие издержки реальной средней фирмы, которые складываются из суммы издержек всех фирм, функционирующих в данной отрасли, масштабы производства которых отклоняются от оптимального уровня.

    Однако это еще далеко не все. Дело в том, что монополия обязательно ведет к завышению издержек производства, т. е. наличие монополии вызывает специфические расходы, не связанные напрямую с особенностями производства и реализации производимых продуктов или услуг.

    Обычно рассматривают три группы причин, вызывающих специфические расходы монополии:

    \begin{enumerate}
        \item Сохранение монополии --- сюда относятся так называемые легальные издержки, находящиеся в пределах, допускаемых действующим законодательством (затраты на приобретение лицензий, квот, патентов и т.п.) и нелегальные издержки, связанные с применением методов нечестной конкуренции, а также с привлечением внимания представителей политической элиты, осуществляющих лоббирование (от английского \emph{lobby} --- кулуары) в пользу данной фирмы-монополиста (обработка парламентариев, членов правительства и правительственного аппарата, руководителей политических партий, профсоюзов, журналистов, представителей местных властей и т.п.)

        \item Безнаказанность и самоуверенность монополистической власти. Это оборотная сторона отсутствия конкуренции. Затраты, вытекающие из этой ситуации получили название X-эффективность, хотя по смыслу чаще их называют Х-неэффективность. Первоначально этот термин относился к взаимоотношениям управляющих и рабочих, затем его использование было расширено на отношения управляющих и собственников.

        Можно выделить некоторые виды затрат и упущенных возможностей, которые вызывают эффект Х-неэффективности:
        \begin{itemize}
            \item неэффективное управление, вытекающее из пренебрежения достижениями науки и техники, опытом более эффективных фирм, волевых решений управляющих;

            \item уклонение от предпринимательского риска;

            \item принятие на работу по принципу родства, личной преданности, знакомства, протекции, подхалимажа, в счет оплаты за оказанные услуги и т.п., а не по критерию компетентности;

            \item утрата инициативы рабочими и специалистами.
        \end{itemize}


        Те или иные признаки Х-неэффективности присущи любому типу монополизма, включая монополию государства.

        В результате монопольная фирма не способна минимизировать свои издержки производства и максимизировать свой выпуск при заданных ресурсах. Такое явление еще называют ситуацией технической неэффективности.

        \item Можно выделить еще такую причину как искусственно создаваемая монополия, хотя она по содержанию чем-то перекликается с затратами на сохранение монополии.

        В 60-70-х годах прошлого века в США появилась так называемая теория извлечения ренты, или теория политической ренты. Она рассматривает перемещение богатства в форме ренты (трансферт ренты). Главным условием является монопольное положение фирмы на рынке, которое достигается предоставлением государством тех или иных исключительных прав (тарифы, квоты, лицензии, установление порогов цен, государственное финансирование и т. п.). Эти государственные привилегии вводятся под давлением заинтересованных групп. Установление подобных привилегий требует затрат реальных ресурсов, которые не приводят к созданию новых ценностей и являются абсолютно непроизводительными.

        Политическая рента выступает как доход от полученных привилегий за вычетом понесенных затрат. В основе этого дохода лежит монополия, опирающаяся не на естественную редкость, а на искусственно созданную с помощью государства.

        Затраты, связанные с установлением и присвоением политической ренты, также увеличивают издержки монопольной фирмы.
    \end{enumerate}

    Таким образом, издержки производства в модели рынка чистой монополии не только выше минимальных средних общих издержек, но и вообще выше средних общих издержек, т.е.:

    $$ ATC_{\text{чм}} > ATC_{\text{нк}}$$

    $ATC_{\text{чм}}$ --- средние общие издержки чисто монопольной фирмы;

    $ATC_{\text{нк}}$ --- средние общие издержки фирмы на рынках несовершенной конкуренции.

    Это положение проиллюстрировано на Рис. \ref{fig:costs:compare}.

    \begin{figure}[h]
        \centering
        \begin{tikzpicture}
            \begin{axis}[
            ticks=none,
            xmin = -1,
            ymin = -1,
            xmax = 10,
            ymax = 10,
            axis lines=middle,
            axis line style={->},
            x label style={at={(axis description cs:1,0.12)},anchor=east},
            xlabel = $Q$,
            y label style={at={(axis description cs:0.14,1)},anchor=north},
            ylabel = {$AC$},
            width=15cm,
            height=15cm,
            ]
            \node[label={200:{0}},circle,fill,inner sep=1pt] at (axis cs:0,0) {};

            \node[label={60:{$ATC_{\text{нк}}$}}] at (axis cs:0.5,4) {};
            \addplot [
                domain=2:9,
                samples=200,
                color=black,
            ]
            {7 - sqrt(- x^2 + 11 * x - 18)};

            \node[label={60:{$ATC_{\text{чм}}$}}] at (axis cs:0.5,6) {};
            \addplot [
                domain=2:9,
                samples=200,
                color=black,
            ]
            {5 - sqrt(- x^2 + 11 * x - 18)};

            \node[label={60:{$B_1$}},circle,fill,inner sep=1pt] at (axis cs:2.62,5.011) {};
            \node[label={60:{$B$}},circle,fill,inner sep=1pt] at (axis cs:2.62,3.011) {};
            \addplot[thick, dashed] plot coordinates {(2.62, 5.011) (2.62, 0)};
            \node[label={60:{$Q_B$}},circle,fill,inner sep=1pt] at (axis cs:2.62,0) {};

            \node[label={60:{$A_1$}},circle,fill,inner sep=1pt] at (axis cs:5.5,3.5) {};
            \node[label={60:{$A$}},circle,fill,inner sep=1pt] at (axis cs:5.5,1.5) {};
            \addplot[thick, dashed] plot coordinates {(5.5, 3.5) (5.5, 0)};
            \node[label={60:{$Q_A$}},circle,fill,inner sep=1pt] at (axis cs:5.5, 0){};

            \node[label={0:{$C_1$}},circle,fill,inner sep=1pt] at (axis cs:8.38,5.011) {};
            \node[label={0:{$C$}},circle,fill,inner sep=1pt] at (axis cs:8.38,3.011) {};
            \addplot[thick, dashed] plot coordinates {(8.38, 5.011) (8.38, 0)};
            \node[label={60:{$Q_C$}},circle,fill,inner sep=1pt] at (axis cs:8.38,0) {};
            \end{axis}
        \end{tikzpicture}
        \caption{Сравнение издержек производства}
        \label{fig:costs:compare}
    \end{figure}

    \newpage

    \section*{Условия существования}

    Появление и длительное функционирование предприятий-монополистов обусловлено наличием целого ряда экономических, технических, юридических и других барьеров, препятствующих вступлению других товаропроизводителей в отрасль. Однако в долговременном периоде абсолютно непреодолимых барьеров для в вступления в отрасль не существует.

    Исходя из принципа различной степени ограниченности доступа на рынок, монополии могут быть классифицированы как \emph{закрытые}, \emph{естественные} и \emph{открытые}. В силу различных обстоятельств одно предприятие может стать единственным поставщиком продукции на рынке.

    \emph{Закрытая монополия} возникает в результате имеющихся нормативно-законодательных актов, которые либо препятствуют проникновению других предприятий в ту или иную сферу хозяйственной деятельности, либо не допускают возможности использования чужой интеллектуальной собственности. На защите последней стоит, прежде всего, патентное право, институт авторских прав.

    \emph{Естественная монополия} образуется в тех отраслях, в которых долгосрочные издержки достигают минимума только тогда, когда одно предприятие обслуживает весь рынок в целом. В таких отраслях оптимальный масштаб производства товара близок или превосходит тот его объем, на который предъявляется спрос по цене, достаточной для покрытия издержек производства. В данной ситуации разделение выпуска продукции между двумя или большим количеством предприятий приведет к тому, что масштабы производства каждого будут далеки от оптимума, что снизит эффективность их деятельности, а также вызовет потери у потребителей в результате роста цен.

    \emph{Открытая монополия} характеризуется тем, что предприятие на некоторое время становится единственным поставщиком какого-либо продукта не в результате каких-либо мер по защите от конкурентов, как это имеет место в случае \emph{закрытой} или \emph{естественной монополии}, а благодаря новизне предлагаемого товара или услуги. В ситуации \emph{открытой монополии} часто оказываются предприятия, впервые вышедшие на рынок с новой продукцией. Однако это не исключает появления конкурентов, но они могут появиться на рынке гораздо позже.

    Конечно, данная классификация имеет весьма условный характер. Отдельные предприятия при стечении благоприятных для них обстоятельств могут обладать чертами сразу нескольких видов монополий. На больших исторических интервалах времени все без исключения монополии можно считать \emph{открытыми}. Действительно, легальные барьеры, которые защищают \emph{закрытые монополии} от конкурентов, могут быть либо опротестованы, либо отменены, что нередко встречается в антимонопольной практике различных стран. Преимущества в издержках \emph{естественных монополий} могут быть сведены на нет изменениями в технологии или появлением принципиально новых субститутов.

    \newpage

    \section*{Условия максимизации прибыли}

    Традиционно предполагается, что фирма --- абсолютная монополия руководствуется в своем поведении принципом максимизации прибыли:

    $$ \frac{dPR}{dQ} = p(Q) \cdot (1 + \frac{dp(Q)}{dQ} \cdot \frac{Q}{p(Q)}) - MC = p(Q) \cdot ( 1 + E_q^d) - MC =  $$

    $$ = p ( 1 + \frac{1}{\frac{dQ(p)}{dp} \cdot \frac{p}{Q(p)}}) - MC = p(1 + \frac{1}{E_p^d}) - MC = MR - MC = 0 $$

    Таким  образом,  необходимым  условием  максимума  прибыли монополии является равенство предельного дохода и предельных издержек:

    $$ MR = p(1 + \frac{1}{E_p^d}) = MC $$

    Достаточным условием максимума функции прибыли в найденной из необходимого условия экстремальной точке является отрицательность второй производной данной функции:

    $$ \frac{d^2TR}{dQ^2} - \frac{d^2TC}{dQ^2} \leq 0, \text{ или} \frac{d^2TR}{dQ^2} \leq \frac{d^2TC}{dQ^2} \text{, т.е. } MR' \leq MC'$$

    Монополия всегда выбирает объём выпуска, который соответствует эластичному (по цене) участку функции спроса, ведь именно этот сегмент соответствует положительной величине предельного дохода, а монополист никогда не станет увеличивать объём производства, если это не только повлечет за собой дополнительные издержки, но и приведёт к сокращению выручки.

    Достаточное условие будет выполняться при возрастающих предельных  издержек,  где $MC' < 0$, а $MR' > 0$. На убывающих предельных издержек будет не максимум прибыли, а её минимум, то есть максимум убытков. Если функция спроса имеет вид $P = a - bQ$, то функция общей выручки  оказывается квадратичной параболой: $TR = P \cdot Q = aQ - bQ^2$, а предельный доход будет линейной функцией с угловым коэффициентом по модулю в два раза большим, чем у функции спроса: $MR = TR' = a - 2b$Q.

    Один и тот же объем производства монополии может соответствовать различным уровням рыночной цены, если рыночный спрос претерпевает изменения. Это означает   отсутствие функциональной зависимости между объёмом предложения и рыночной ценой в условиях монополии.

    Один и тот же уровень рыночной цены может соответствовать различным объёмам производства монополии, если рыночный спрос претерпевает  изменения.  Это  означает  отсутствие  функциональной зависимости  между  объёмом  предложения и  рыночной  ценой  в условиях монополии.

    Если монополия включает в себя несколько заводов, то для неё встает  задача  оптимального  распределения  максимизирующего  прибыль объёма производства между данными предприятиями:

    $$ \max_{q_i, i = 1,...,n} PR = \max_{q_i, i = 1,...,n} \bigg\{ p Q - \sum_{i = 1}^n TC_i \bigg\},  $$
    где $q_i$ --- выпуск продукта на отдельном заводе, $TC_i$ --- издержки его производства, $i = \{ 1,...,n \}$, $Q = \sum_{i = 1}^n q_i $ --- совокупный выпуск монополии.

    Необходимым  условием  максимума  прибыли  картеля  является равенство нулю частных производных функции прибыли по объемам выпусков  подразделений  данного  монополиста.
    \newpage
    Другими  словами, должна выполняться система из $n$ условий ($i = 1,...,n)$:

    $$ \frac{\partial PR}{\partial q_i} = \frac{\partial TR}{\partial q_i} - \frac{\partial TC}{\partial q_i} = \frac{\partial TR}{\partial Q} \cdot \frac{\partial Q}{\partial q_i} - \frac{\partial \sum_{j = 1}^n TC_j}{\partial q_i} = $$

    $$ = MR \frac{\partial (q_i + \sum_{j = 1, j \neq i}^n q_j)}{\partial q_i} - \sum_{j = 1}^n \frac{\partial TC_j}{ \parital q_i } = MR - MC_i = 0. $$

    Здесь  предполагается  технологическая  обособленность  производственных подразделений монополии:

    $$ \frac{\partial TC_j}{\parital q_i} = 0 \text{ при } j \neq i. $$

    Таким образом, условием максимизации прибыли в данном случае будет равенство предельных издержек производства на каждом из предприятий, входящих в многозаводскую монополию, между собой, а также предельной выручки всей фирмы:

    $$ MR = MC_i, i = \{1, ..., n \} $$
\end{document}