\documentclass[12pt]{article}

\usepackage[english, russian]{babel}
\usepackage[T2A]{fontenc}
\usepackage[utf8]{inputenc}
\usepackage[left=2cm,right=2cm, top=1cm,bottom=1.5cm,bindingoffset=0cm]{geometry}
\usepackage{hyperref}

% \usepackage{multirow}
% \usepackage{hhline}

% \usepackage{indentfirst}

% \usepackage{enumitem,kantlipsum}

% \usepackage{graphicx}
% \graphicspath{{pictures/}}
% \DeclareGraphicsExtensions{.pdf,.png,.jpg}

% \usepackage{tikz}
% \usetikzlibrary{patterns}
% \usepackage{pgfplots}
% \pgfplotsset{compat=1.9}
% \usepgfplotslibrary{fillbetween}

% \usepackage{ulem}

% \usepackage{hyperref}

% \usepackage{circuitikz}

% \usepackage{fp}
% \usepackage{xfp}

% \usepackage{siunitx}
% \sisetup{output-decimal-marker={,}}

% \usepackage{minted}

% \let\oldref\ref
% \renewcommand{\ref}[1]{(\oldref{#1})}

\begin{document}
    \pagestyle{empty}
    \begin{center}
        \textbf{Федеральное государственное автономное образовательное учреждение высшего образования}

        \vspace{5pt}

        {\small
        \textbf{САНКТ-ПЕТЕРБУРГСКИЙ НАЦИОНАЛЬНЫЙ ИССЛЕДОВАТЕЛЬСКИЙ УНИВЕРСИТЕТ ИНФОРМАЦИОННЫХ ТЕХНОЛОГИЙ, МЕХАНИКИ И ОПТИКИ}

        \textbf{ФАКУЛЬТЕТ ПРОГРАММНОЙ ИНЖЕНЕРИИ И КОМПЬЮТЕРНОЙ ТЕХНИКИ}%
        }

        \vspace{140pt}

        {\Large
        \textbf{ДОМАШНЯЯ РАБОТА}

        \vspace{7pt}

        \textbf{РАБОТА}%
        }

        \vspace{10pt}

        {\large
        \textbf{Тестирование программного обеспечения}

        \vspace{5pt}

        \textbf{Высокоуровненое тестирование}%
        }

        \vspace{170pt}

        \begin{tabular}{lll}
            Проверил:                                                                                   & \hspace{70pt} & Выполнил:                                             \\
            Сентерев Ю. А.                \rule[0.66\baselineskip]{2cm}{0.4pt}                &               & Студент группы P3455                                  \\
            «\rule[0.66\baselineskip]{1cm}{0.4pt}»  \rule[0.66\baselineskip]{2cm}{0.4pt} \the\year г.   &               & Федюкович С. А. \rule[0.66\baselineskip]{2cm}{0.4pt}  \\
            &               &                                                       \\
            Оценка          \hspace{12pt}           \rule[0.66\baselineskip]{2.7cm}{0.4pt}              &               &                                                       \\
        \end{tabular}

        \vspace*{\fill}

        Санкт-Петербург

        \the\year
    \end{center}
    \newpage
    \pagestyle{plain}
    \setcounter{page}{1}

    \section*{Высокоуровневое тестирование}
    Высокоуровневое тестирование --- это проверка соответствия программы разумным ожиданиям пользователя, также это следующий шаг к Модульному тестированию.  Хорошо известно, что тестирование модулей не является надежным методом выявления и обнаружения всех ошибок в программном обеспечении. В результате необходимо провести дальнейшие испытания. Эти дополнительные тесты составляют тестирование высокого порядка.



    В большинстве случаев организации используют стороннюю организацию для проверки своего программного обеспечения с помощью Высокоуровневого тестирования.

    \section*{Необходимость Высокоуровневого тестирования}
    Процесс разработки программного обеспечения включает в себя выполнение различных задач или действий или требований, выполняемых в несколько этапов, которые необходимо проверять и подтверждать с помощью различных методов тестирования, охватываемых Высокоуровневым тестированием, таких как:
    \begin{itemize}
        \item Функциональное тестирование
        \item Системное тестирование
        \item Приемочное тестирование
        \item Инсталляционное тестирование
    \end{itemize}


    Функциональное   тестирование --- это тестирование ПО в целях проверки реализуемости функциональных требований, то есть способности ПО в определённых условиях решать задачи, нужные пользователям. В зависимости от цели, функциональное тестирование может проводиться:
    \begin{itemize}
        \item На основе функциональных требований, указанных в спецификации требований. При  этом  для  тестирования  создаются  тестовые  случаи, составление  которых  учитывает  приоритетность  функций  ПО,  которые необходимо покрыть тестами. Таким образом мы можем убедиться в том, что все функции разрабатываемого продукта работают корректно при различных типах входных данных, их комбинаций, количества и т.д.
        \item На основе бизнес-процессов, которые должно обеспечить приложение. В этом  случае,  нас  интересует  не  так  работоспособность  отдельных  функций ПО,  как  корректность  выполняемых  операций,  с  точки  зрения  сценариев использования системы. Таким образом, тестирование в данном случае будет основываться на вариантах использования системы.
    \end{itemize}


    Системное тестирование --- это тестирование программного обеспечения, выполняемое на полной, интегрированной системе, с целью проверки соответствия системы исходным требованиям, как функциональным, так и не функциональным. Выполняя системное тестирование, можно обнаружить следующие типы дефектов:
    \begin{itemize}
        \item Неправильное использование системных ресурсов.
        \item Непредусмотренные комбинации пользовательских данных.
        \item Проблемы с совместимостью окружения.
        \item Непредусмотренные сценарии использования.
        \item Несоответствие с функциональными требованиями.
        \item Плохое удобство использования.
    \end{itemize}

    Системное тестирование выполняется методом «Черного ящика», т.к.проверяемое множество является «внешними» сущностями, которые не требуют взаимодействия с внутренним устройством программы. Также выполнять его рекомендуется в окружении, максимально приближенном к окружению конечного пользователя.

    \vspace{5pt}

    Можно выделить 2 подхода к системному тестированию:
    \begin{itemize}
        \item На базе требований. Тестирование проводится в соответствии с функциональными или не функциональными требованиями, для каждого из которых пишется тестовый случай.
        \item На базе случаев использования. Тестирование происходит в соответствии с вариантами использования продукта, на основе которых создаются пользовательские прецеденты. Для каждого из данных пользовательских прецедентов создаются свои тестовый случай.
    \end{itemize}
    Приемочное тестирование --- вид тестирования, проводимый на этапе сдачи готового продукта (или готовой части продукта) заказчику. Целью приемочного тестирования является определение готовности продукта, что достигается путем прохода тестовых сценариев и случаев, которые построены на основе спецификации требований к разрабатываемому ПО.

    \vspace{5pt}

    Результатом приемочного тестирования может стать:
    \begin{itemize}
        \item Отправка проекта на доработку.
        \item Принятие его заказчиком, в качестве выполненной задачи.
    \end{itemize}
    Это финальный этап тестирования продукта перед его релизом. При этом, он не является сверх-тщательным, всеохватывающим и полным --- тестируется, в основном, только основной функционал.

    \vspace{5pt}

    Приемочное тестирование проводится либо самим заказчиком, либо группой специалистов по тестированию, представляющих интересы заказчика, либо специалистом по тестированию компании --- разработчика, что зависит от предпочтений компании --- заказчика.

    \vspace{5pt}

    Для создания системных тестов нет какой-либо методологии, исключительно опыт специалиста по тестированию.Обычно эти тесты делят на 14 категорий:
    \begin{itemize}
        \item Возможности: Проверяется   полнота   реализации функциональных возможностей, определенных целями.
        \item Предельные объемы данных: Проверяется способность программы обрабатывать предельно большие объемы данных.
        \item Нагрузочное тестирование: Проверяется работоспособность программы при повышенных нагрузках.
        \item Удобство использования: Определяется, насколько удобно конечному  пользователю  работать  с программой.
        \item Безопасность: Предпринимаются  попытки  обойти средства защиты программы.
        \item Производительность: Проверяется соответствие программы требованиям производительности и скорости отклика.
        \item Память: Проверяется  способность  системы эффективно использовать оперативную и постоянную память.
        \item Конфигурация: Проверяется работоспособность системы в рекомендованных конфигурациях.
        \item Совместимость: Определяется  совместимость  новых версий программы с предыдущими.
        \item Установка: Проверяется работоспособность методов  установки программы  на всех поддерживаемых платформах.
        \item Надежность: Определяется соответствие программы специфицированным показателям надежности.
        \item Восстанавливаемость: Определяется, обеспечивает    ли приложение механизмы предоставления  данных  о  событиях, требующий   оказания   технический поддержки.
        \item Документированность: Проверяется точность всей пользовательской документации.
        \item Процедуры: Определяется точность специальных процедур, которые должны соблюдаться в процессе использования   или   обслуживания программы.
    \end{itemize}

    \section*{Выводы}
    В результате выполненного домашнего задания были изучены основные понятия, стратегии и общие практики высокоуровневого тестирования. Были рассмотрены категории системного тестирования. Задание выполнено в полном объеме.
    \newpage
    \section*{Список источников}
    \begin{enumerate}
        \item Гленфорд Майерс, Том Баджетт, Кори Сандлер Искусство тестирования программ, 3-е издание = The Art of Software Testing, 3rd Edition. —М.: «Диалектика», 2015. —272 с. —ISBN 978-5-8459-1796-6
        \item Бейзер Б. Тестирование чёрного ящика. Технологии функционального тестирования программного обеспечения и систем. —СПб.: Питер, 2004. —320 с. —ISBN 5-94723-698-2
        \item Канер Кем, Фолк Джек, Нгуен Енг Кек Тестирование программного обеспечения. Фундаментальные концепции менеджмента бизнес-приложений. —Киев: ДиаСофт, 2001. —544 с. —ISBN 9667393879
        \item \href{https://qalight.com.ua/baza-znaniy/modulnoe-testirovanie/}{Статья “Модульное тестирование”}.
    \end{enumerate}
\end{document}\