\documentclass[12pt]{article}

\usepackage[english, russian]{babel}
\usepackage[TS1, T2A]{fontenc}
\usepackage[utf8]{inputenc}
\usepackage[left=2cm,right=2cm, top=1cm,bottom=1.5cm,bindingoffset=0cm]{geometry}
\setlength{\parindent}{0cm}
\usepackage{hyperref}
\usepackage{tabularx}
\newcolumntype{b}{X}
\newcolumntype{s}{>{\hsize=.8\hsize}X}
\newcolumntype{m}{>{\hsize=.7\hsize}X}
% \usepackage{multirow}
% \usepackage{hhline}
% \usepackage{indentfirst}

% \usepackage{enumitem,kantlipsum}

\usepackage{graphicx}
\graphicspath{{images/}}

\usepackage{listings}
\lstset{
    language=bash,
    basicstyle=\ttfamily
}
% \DeclareGraphicsExtensions{.pdf,.png,.jpg}

% \usepackage{tikz}
% \usetikzlibrary{patterns}
% \usepackage{pgfplots}
% \pgfplotsset{compat=1.9}
% \usepgfplotslibrary{fillbetween}

% \usepackage{ulem}

% \usepackage{hyperref}

% \usepackage{circuitikz}

% \usepackage{fp}
% \usepackage{xfp}

% \usepackage{siunitx}
% \sisetup{output-decimal-marker={,}}

% \usepackage{minted}

% \let\oldref\ref
% \renewcommand{\ref}[1]{(\oldref{#1})}

\begin{document}
    \pagestyle{empty}
    \begin{center}
        \textbf{Федеральное государственное автономное образовательное учреждение высшего образования}

        \vspace{5pt}

        {\small
        \textbf{САНКТ-ПЕТЕРБУРГСКИЙ НАЦИОНАЛЬНЫЙ ИССЛЕДОВАТЕЛЬСКИЙ УНИВЕРСИТЕТ ИНФОРМАЦИОННЫХ ТЕХНОЛОГИЙ, МЕХАНИКИ И ОПТИКИ}

        \textbf{ФАКУЛЬТЕТ ПРОГРАММНОЙ ИНЖЕНЕРИИ И КОМПЬЮТЕРНОЙ ТЕХНИКИ}%
        }

        \vspace{140pt}

        {\Large
        \textbf{ЛАБОРАТОРНАЯ}

        \vspace{7pt}

        \textbf{РАБОТА №3}%
        }

        \vspace{10pt}

        {\large
        \textbf{Тестирование программного обеспечения}

        \vspace{5pt}

        \textbf{}%
        }

        \vspace{170pt}

        \begin{tabular}{lll}
            Проверил:                                                                                   & \hspace{70pt} & Выполнил:                                             \\
            Сентерев Ю. А.                 \rule[0.66\baselineskip]{1.6cm}{0.4pt}                &               & Студент группы P3455                                  \\
            «\rule[0.66\baselineskip]{1cm}{0.4pt}»  \rule[0.66\baselineskip]{2cm}{0.4pt} \the\year г.   &               & Федюкович С. А. \rule[0.66\baselineskip]{2cm}{0.4pt}  \\
            &               &                                                       \\
            Оценка          \hspace{12pt}           \rule[0.66\baselineskip]{2.7cm}{0.4pt}              &               &                                                       \\
        \end{tabular}

        \vspace*{\fill}

        Санкт-Петербург

        \the\year
    \end{center}
    \newpage
    \pagestyle{plain}
    \setcounter{page}{1}
    \section*{Цель работы}

    Целью данной лабораторной работы является изучение методологий и овладение навыками тестирования веб-приложений.

    В ходе выполнения работы будут получены навыки составления тестовых случаев для тестирования веб-приложений, а также навыки работы в составе инспекционной группы с подготовкой итогового отчета о выявленных проблемах.

    \section*{Задачи}

    1. Выбрать веб-приложение для тестирования, обосновать выбор.

    2. Выбрать браузер для тестирования веб-приложения, обосновать выбор.

    3. Составить план тестирования.

    4. Провести тестирование логики и визуализации выбранного веб-приложения. При тестировании   использовать   тесты   относящиеся   к   категории модульного тестирования.

    5. Представить результаты тестирования в виде отчета.

    \section*{Ход Работы}
    1. Для тестирования было выбрано веб приложение OZON \url{https://ozon.ru}. Данное приложение, поскольку оно является одной из самой крупной площадкой для продажи товаров в России на момент написания работы.

    2. Для тестирования был выбран браузер Chromium версии 87.0.4280.141. Данный браузер был выбран, потому что он является основой почти для всех браузеров в настоящее время. Таким образом, данный бразуер является самым популярным.

    3. Поскольку исходный код веб-приложения неизвестен, то для тестирования была выбрана стратегия "черного ящика". План тестирования представлен в виде таблицы:

    \begin{table}[h]
        \centering
        \begin{tabularx}{\textwidth}{m X s X}
            \hline
            Название теста & Тестовый сценарий & Тестовые данные & Ожидаемый результат \\
            \hline
            Попытка входа без регистрации & Открыть окно входа по электронной почте, ввести почту, не зарегистрированную в приложении, нажать кнопку входа  & Электронная почта test-sample@mail.com & Приложение отображает ошибку о невозможности войти в аккаунт \\
            \hline
            Попытка входа/регистрации при некорректном вводе & Открыть окно входа/регистрации, ввести некорректные данные для входа, нажать кнопку входа & Некорректные данные для входа & Приложение отображает ошибку о некорректности введенных данных \\
            \hline
            Регистрация по номеру телефона  & Открыть окно регистрации, ввести корректный номер телефона, ввести код подтверждения, нажать кнопку регистрации & Номер телефона & Приложение успешно регистрирует нового пользователя \\
            \hline
        \end{tabularx}
        \caption{План тестирования. Модуль регистрация}
    \end{table}

    \newpage

    \begin{table}[h]
        \centering
        \begin{tabularx}{\textwidth}{m X s X}
            \hline
            Название теста & Тестовый сценарий & Тестовые данные & Ожидаемый результат \\
            \hline
            Изменение профиля с указанием длинных данных & Открыть настройки пользователя, нажать кнопку изменения личных данных, ввести в поля "Имя" и "Фамилия" текст длиннее более 30 символов & нет & Приложение сохраняет данные или отображает ошибку о превышении максимальной длинны полей \\
            \hline
            Некорректное изменение даты рождения & Открыть настройки пользователя, нажать кнопку изменения личных данных, в поле даты рождения ввести произвольные набор цифр, не являющийся датой & Произвольный набор цифр & Приложение не сохраняет дату рождения и отображает ошибку о некорректности введенных данных \\
            \hline
            Успешное изменение данных профиля & Открыть настройки пользователя, нажать кнопку изменения личных данных, ввести во поля корректные данные профиля & Корректные данные профиля & Приложение сохраняет данные и обновляет их \\
            \hline
        \end{tabularx}
        \caption{План тестирования. Модуль профиль}
    \end{table}

    Тестирование визуальной части будет проводить "на глаз". Все элементы сайта должны быть удобными и должны находиться в одной палитре цветов.

    \newpage

    \begin{table}[h]
        \centering
        \begin{tabularx}{\textwidth}{m X s X}
            \hline
            Название теста & Тестовый сценарий & Тестовые данные & Ожидаемый результат \\
            \hline
            Покупка количества товаров, превышающего лимит & Добавить в корзину любой товар, в корзине установить количество 9999999 & нет & Приложение не даёт установить нужное значение или отображает ошибку   \\
            \hline
            Оплата большого и дорогого заказа & Добавить в корзину большое количество товаров на большую сумму денег, больше 10000000 рублей & нет & Приложение успешно переводит на страницу оплаты   \\
            \hline
            Отмена заказа, ожидающего оплаты & Добавить в корзину любой товар, перейти к оформлению, ввести данные о доставке, перейти на шаг оплаты, перейти к списку заказов и отменить созданный заказ & нет & Приложение успешно отменяет заказ   \\
            \hline
        \end{tabularx}
        \caption{План тестирования. Модуль заказ}
    \end{table}

    4. Тестирование начинается с модуля регистрации.

    Для первого теста вводится не зарегистрированная электронная почта в окно входа. После нажатия на кнопку "Получить код", отображается ошибка:

    \begin{figure}[h]
        \includegraphics[scale=0.22]{test1.png}
        \centering
        \caption{Попытка входа без регистрации}
    \end{figure}

    \newpage

    Для следующего теста вводится некорректный номер телефона в окно входа. После нажатия на кнопку "Получить код", отображается ошибка:

    \begin{figure}[h]
        \includegraphics[scale=0.22]{test2.png}
        \centering
        \caption{Попытка входа/регистрации при некорректном вводе}
    \end{figure}

    В следующем тесте вводится корректный номер телефона в окно входа. После нажатия на кнопку "Получить код" и ввода проверочного кода происходит успешная регистрация:

    \begin{figure}[h]
        \includegraphics[scale=0.22]{test3.png}
        \centering
        \caption{Регистрация по номеру телефона}
    \end{figure}

    \newpage

    В следующем модуле проводится тестирование профиля. В первом тесте проводится попытка изменение имени и фамилии на очень длинные. После нажатия на кнопку сохранить данные не изменяются и не отображается ошибка:

    \begin{figure}[h]
        \includegraphics[scale=0.22]{test4.png}
        \centering
        \caption{Изменение профиля  с  указанием  длинных данных}
    \end{figure}

    В ходе данного теста было установлено, что максимальная длина имени и фамилии это 15 символов.

    \newpage

    В тесте установки некорректной даты приложение не сообщает об ошибке, а округляет дату к ближайшей возможной, включая даже дату рождения в будущем:

    \begin{figure}[h]
        \includegraphics[scale=0.22]{test5.png}
        \centering
        \caption{Некорректное изменение даты рождения}
    \end{figure}

    \begin{figure}[h]
        \includegraphics[scale=0.22]{test6.png}
        \centering
        \caption{Некорректное изменение даты рождения}
    \end{figure}

    \newpage

    В ходе последнего теста данного модуля при вводе корректных данных профиль успешно обновляется:

    \begin{figure}[h]
        \includegraphics[scale=0.22]{test7.png}
        \centering
        \caption{Успешное изменение   данных профиля}
    \end{figure}

    \begin{figure}[h]
        \includegraphics[scale=0.22]{test8.png}
        \centering
        \caption{Успешное изменение   данных профиля}
    \end{figure}

    \newpage

    В первом тесте последнего модуля проверяется возможность ввести слишком большое количество товаров к заказу в корзине, но приложение не даёт ввести количество товаров большее возможного:

    \begin{figure}[h]
        \includegraphics[scale=0.22]{test9.png}
        \centering
        \caption{Покупка  количества товаров, превышающего лимит}
    \end{figure}

    В следующем тесте последнего модуля проверяется возможность сделать очень большой и дорогой заказ. Приложение не выводит никаких ошибок во время оформления такого заказа, доводя прямо до оплаты:

    \begin{figure}[h]
        \includegraphics[scale=0.22]{test10.png}
        \centering
        \caption{Оплата   большого и дорогого заказа}
    \end{figure}

    \newpage

    В последнем тесте последнего модуля проверяется возможность отменить заказ, ожидающий оплаты. После нажатия соответствующей кнопки и указания причины отмены, приложение успешно отменяет заказ:

    \begin{figure}[h]
        \includegraphics[scale=0.22]{test11.png}
        \centering
        \caption{Отмена заказа, ожидающего оплаты}
    \end{figure}



    5. В ходе тестирования было проведено 9 тестов из 3 модулей. 2 теста не прошли проверку, что составляет 22\% ошибки. Ошибочные сценарии описаны в плане тестирования в Таблице 2.

    Графическая составляющая приложения выполнена отлично. Во время тестирования не возникло никаких затруднений в том, чтобы разобраться в навигации. Также добавление большого количества товаров не заняло много времени.

    \section*{Вывод}

    В ходе лабораторной работы я успешно провёл тестирование веб-приложения Ozon согласно плану и подготовил отчет по проведенному тестированию. Выполнил все поставленные задачи. Лабораторную работу считаю выполненной в полном объеме.

    \section*{Используемая литература}

    1. Гленфорд Майерс, Том Баджетт, Кори Сандлер. Искусство тестирования программ, 3-е издание—М.: «Диалектика», 2015

    2. Бейзер Б. Тестирование чёрного ящика. Технологии функционального тестирования программного обеспечения и систем --- СПб.: Питер, 2004

    3. Канер Кем, Фолк Джек, Нгуен Енг Кек. Тестирование программного обеспечения. Фундаментальные концепции менеджмента бизнес-приложений --- Киев: ДиаСофт, 2001

    4. Винниченко И. Автоматизация процессов тестирования. --- СПб, «Питер», 2018

    5. Котляров В. П., Коликова Т. В. Основы тестирования программного обеспечения --- СПб, Бином. Лаборатория знаний, 2006

    6. Полевой В. Как автоматизировать тестирование ПО? --- Cnews, 2019
\end{document}