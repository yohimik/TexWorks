\documentclass[12pt]{article}

\usepackage[english, russian]{babel}
\usepackage[TS1, T2A]{fontenc}
\usepackage[utf8]{inputenc}
\usepackage[left=2cm,right=2cm, top=1cm,bottom=1.5cm,bindingoffset=0cm]{geometry}
\setlength{\parindent}{0cm}
\usepackage{hyperref}
\usepackage{tabularx}
\newcolumntype{b}{X}
\newcolumntype{s}{>{\hsize=.8\hsize}X}
\newcolumntype{m}{>{\hsize=.7\hsize}X}
% \usepackage{multirow}
% \usepackage{hhline}

% \usepackage{indentfirst}

% \usepackage{enumitem,kantlipsum}

\usepackage{graphicx}
\graphicspath{{images/}}

\usepackage{listings}
\lstset{
    language=bash,
    basicstyle=\ttfamily
}
% \DeclareGraphicsExtensions{.pdf,.png,.jpg}

% \usepackage{tikz}
% \usetikzlibrary{patterns}
% \usepackage{pgfplots}
% \pgfplotsset{compat=1.9}
% \usepgfplotslibrary{fillbetween}

% \usepackage{ulem}

% \usepackage{hyperref}

% \usepackage{circuitikz}

% \usepackage{fp}
% \usepackage{xfp}

% \usepackage{siunitx}
% \sisetup{output-decimal-marker={,}}

% \usepackage{minted}

% \let\oldref\ref
% \renewcommand{\ref}[1]{(\oldref{#1})}

\begin{document}
    \pagestyle{empty}
    \begin{center}
        \textbf{Федеральное государственное автономное образовательное учреждение высшего образования}

        \vspace{5pt}

        {\small
        \textbf{САНКТ-ПЕТЕРБУРГСКИЙ НАЦИОНАЛЬНЫЙ ИССЛЕДОВАТЕЛЬСКИЙ УНИВЕРСИТЕТ ИНФОРМАЦИОННЫХ ТЕХНОЛОГИЙ, МЕХАНИКИ И ОПТИКИ}

        \textbf{ФАКУЛЬТЕТ ПРОГРАММНОЙ ИНЖЕНЕРИИ И КОМПЬЮТЕРНОЙ ТЕХНИКИ}%
        }

        \vspace{140pt}

        {\Large
        \textbf{ЛАБОРАТОРНАЯ}

        \vspace{7pt}

        \textbf{РАБОТА №1}%
        }

        \vspace{10pt}

        {\large
        \textbf{Тестирование программного обеспечения}

        \vspace{5pt}

        \textbf{}%
        }

        \vspace{170pt}

        \begin{tabular}{lll}
            Проверил:                                                                                   & \hspace{70pt} & Выполнил:                                             \\
            Сентерев Ю. А.                 \rule[0.66\baselineskip]{1.6cm}{0.4pt}                &               & Студент группы P3455                                  \\
            «\rule[0.66\baselineskip]{1cm}{0.4pt}»  \rule[0.66\baselineskip]{2cm}{0.4pt} \the\year г.   &               & Федюкович С. А. \rule[0.66\baselineskip]{2cm}{0.4pt}  \\
            &               &                                                       \\
            Оценка          \hspace{12pt}           \rule[0.66\baselineskip]{2.7cm}{0.4pt}              &               &                                                       \\
        \end{tabular}

        \vspace*{\fill}

        Санкт-Петербург

        \the\year
    \end{center}
    \newpage
    \pagestyle{plain}
    \setcounter{page}{1}
    \section*{Цель работы}

    Целью данной лабораторной работы является овладение навыками ручного тестирования и составления тестовых случаев.

    В ходе выполнения работы будут получены навыки документирования процедуры и описания методик тестирования программного обеспечения, а также навыки работы в составе инспекционной группы с подготовкой итогового отчета о выявленных проблемах.

    \section*{Задачи}

    1. Выбрать программу для тестирования и обосновать выбор.

    2. Выбрать стратегию тестирования.

    3. Составить план тестирования.

    4. Провести тестирование программы.

    5. На основании результатов тестирования, составить вывод.

    \section*{Ход Работы}
    1. Для тестирования была выбрана программа LibreOffice версии $6.4.6.2 40(Build:2)$, а именно функция $CLI$ для преобразования документов любого типа в $PDF$ файл. Данная программа была выбрана, поскольку планируется её использование в $WEB$ файловой системе, а так как LibreOffice свободно распространяемый офисный пакет с открытым исходным кодом, то в ней возможно наличие ошибок.

    Функция $CLI$ имеет следующий синтаксис:
    \begin{lstlisting}
        libreoffice --convert-to pdf <paths-to-files>
    \end{lstlisting}
    Где paths-to-files --- это набор путей(минимум 1) к файлам, которые необходимо сконвертировать. После конвертации программа создаёт $PDF$ файл в той же папке, где был исходный документ, с тем же именем, что и у исходного файла.

    2. Хоть и LibreOffice имеет открытый исходный код, но в тестировании будет использоваться стратегия "Чёрного ящика".

    \newpage
    3. Для тестирования был подготовлен документ, содержащий ровно одну страницу, текст в нескольких шрифтах, таблицу и график. Изображение документа представлено на Рис. 1.
    \begin{figure}[h]
        \includegraphics[scale=0.62]{document.png}
        \centering
        \caption{Документ для тестирования}
    \end{figure}

    \newpage

    Данный документ представлен в двух наиболее популярных на момент написания работы форматах $ODT$ и $DOCX$.

    Конвертация документа будет оцениваться "на глаз". Все элементы документа должны быть на своих местах. Шрифты и элементы текста(подчеркивание, курсив, жирный и т. д.) должны быть переведены в формат $PDF$ без потерь. Также в каждом тестовом случае программа не должна изменять передаваемые файлы, а так же создавать новые файлы, если в данном тестовом случае это не предусмотрено.

    План тестирования представлен в виде таблицы ниже:

    \begin{table}[h]
        \centering
        \begin{tabularx}{\textwidth}{m X s X}
            \hline
            Название теста & Тестовый сценарий & Тестовые данные & Ожидаемый результат \\
            \hline
            Запуск без указания путей к файлам & Запустить функцию $CLI$ программы LibreOffice, не указав paths-to-files & нет & Программа выдает ошибку, сообщающую отсутствие путей файлов \\
            \hline
            Запуск с указанием файлов, не являющимися документами & Запустить функцию $CLI$ программы LibreOffice, указав 3 путей к файлам, не являющимися документами & 3 любых файла, не являющихся документами & Программа выдает ошибку, сообщающую о невозможности конвертировать файлы \\
            \hline
            Преобразование $ODT$ & Сконвертировать тестовый документ в формате $ODT$ при помощи функции $CLI$ программы LibreOffice в $PDF$ документ & Тестовый документ в формате $ODT$ & Документ в формате $PDF$, сконвертированный без потерь \\
            \hline
            Преобразование $DOCX$ & Сконвертировать тестовый документ в формате $DOCX$ при помощи функции $CLI$ программы LibreOffice в $PDF$ документ & Тестовый документ в формате $DOCX$ & Документ в формате $PDF$, сконвертированный без потерь \\
            \hline
            Преобразование файлов с одинаковым именем & Сконвертировать два тестовых документа, имеющих одинаковые имена & Два тестовых документа с одинаковым именем & Один файл должен пройти конвертацию без потерь, а второй завершиться конвертацией с ошибкой \\
            \hline
            Преобразование документов вместе с не документом & Сконвертировать два тестовых документа, указав ещё одним аргументом файл, не являющийся документом & Два тестовых документа и один файл, не являющийся документом & Документы должен пройти конвертацию, а при попытке конвертировать не документ программа должна выдать ошибку \\
            \hline
        \end{tabularx}
        \caption{План тестирования}
    \end{table}
    \newpage
    4. Для процесса тестирования была создана отдельная папка, в которой и будет происходить запуск тестовой программы. Перед каждый запуском теста данная папка очищается от всех результатов работы предыдущих тестов, исключая подготовленные тестовые документы.

    Для первого теста запустим программу, не передав ни одного аргумента:
    \begin{figure}[h]
        \includegraphics[scale=0.7]{test1.png}
        \centering
        \caption{Запуск без указания путей к файлам}
    \end{figure}

    Программа перестаёт отвечать и не выводит никакого сообщения.

    Для следующего теста запустим программу, указав пути к файлам, не являющимся документами:
    \begin{figure}[h]
        \includegraphics[scale=0.6]{test2.png}
        \centering
        \caption{Запуск  с  указанием файлов, не являющимися документами}
    \end{figure}

    Программа выдаёт ошибку, что файлы не могут быть загружены. Так же file3 не существует.

    В следующем тесте запустим программу для конвертации $ODT$ файла:
    \begin{figure}[h]
        \includegraphics[scale=0.45]{test3.png}
        \centering
        \caption{Преобразование $ODT$}
    \end{figure}

    \newpage
    Программа завершилась успешно. Ниже представлено содержание $PDF$ файла:
    \begin{figure}[h]
        \includegraphics[scale=0.56]{exp.png}
        \centering
        \caption{Преобразованный $ODT$ файл в $PDF$}
    \end{figure}

    \newpage
    Дальше проведём тест преобразования $DOCX$ файла в $PDF$:
    \begin{figure}[h]
        \includegraphics[scale=0.45]{test4.png}
        \centering
        \caption{Преобразование $DOCX$}
    \end{figure}

    Программа завершилась успешно. Ниже представлено содержание $PDF$ файла:
    \begin{figure}[h]
        \includegraphics[scale=0.45]{exp.png}
        \centering
        \caption{Преобразованный $DOCX$ файл в $PDF$}
    \end{figure}

    \newpage
    Следующий тест заключается в конвертации двух файлов с одинаковым именем:
    \begin{figure}[h]
        \includegraphics[scale=0.45]{test5.png}
        \centering
        \caption{Преобразование файлов с одинаковым именем}
    \end{figure}

    Программа завершилась успешно, но перезаписала уже созданный файл. Результирующий файл полностью аналогичен двум предыдущим.

    В последнем тесте проверяется конвертация нескольких рабочих файлов вместе с ошибочным:
    \begin{figure}[h]
        \includegraphics[scale=0.45]{test6.png}
        \centering
        \caption{Преобразование документов вместе с не документом}
    \end{figure}

    Программа завершилась успешно и обработала все документы, исключая не документ.

    В ходе проведения всех тестов изменение исходных документов и создание программой не преднамеренных файлов не было обнаружено. Программа не делает того, чего не должна.

    \newpage
    5. Итоговые результаты тестирования представлены в таблице ниже:

    \begin{table}[h]
        \centering
        \begin{tabularx}{\textwidth}{m X X}
            \hline
            Название теста & Фактический результат & Ожидаемый результат \\
            \hline
            Запуск без указания путей к файлам & Программа не отвечает & Программа выдает ошибку, сообщающую отсутствие путей файлов \\
            \hline
            Запуск с указанием файлов, не являющимися документами & Программа выдает ошибку, сообщающую о невозможности конвертировать файлы & Программа выдает ошибку, сообщающую о невозможности конвертировать файлы \\
            \hline
            Преобразование $ODT$ & Документ в формат $PDF$ сконвертирован без потерь & Документ в формате $PDF$, сконвертированный без потерь \\
            \hline
            Преобразование $DOCX$ & Документ в формат $PDF$ сконвертирован без потерь & Документ в формате $PDF$, сконвертированный без потерь \\
            \hline
            Преобразование файлов с одинаковым именем & Программа перезаписывает один и тот же файл два раза & Один файл должен пройти конвертацию без потерь, а второй завершиться конвертацией с ошибкой \\
            \hline
            Преобразование документов вместе с не документом & Документы успешно прошли конвертацию, а не документ не был сконвертирован с ошибкой & Документы должен пройти конвертацию, а при попытке конвертировать не документ программа должна выдать ошибку \\
            \hline
        \end{tabularx}
        \caption{Результаты тестирования}
    \end{table}
    Из результатов тестирования видно, что функция $CLI$ программы LibreOffice по преобразованию документов в $PDF$ файл содержит ошибки работы. Данные ошибки не значительны и не влияют на работу программы, но их наличие показывает, что во всех программах присутствуют ошибки работы.

    \section*{Вывод}

    В ходе лабораторной работы я провёл ручное тестирование программы и выполнил все поставленные задачи. Лабораторную работу считаю выполненной.

    \section*{Используемая литература}

    1. Гленфорд Майерс, Том Баджетт, Кори Сандлер. Искусство тестирования программ, 3-е издание—М.: «Диалектика», 2015

    2. Бейзер Б. Тестирование чёрного ящика. Технологии функционального тестирования программного обеспечения и систем --- СПб.: Питер, 2004

    3. Канер Кем, Фолк Джек, Нгуен Енг Кек. Тестирование программного обеспечения. Фундаментальные концепции менеджмента бизнес-приложений --- Киев: ДиаСофт, 2001

    4. Винниченко И. Автоматизация процессов тестирования. --- СПб, «Питер», 2018

    5. Котляров В. П., Коликова Т. В. Основы тестирования программного обеспечения --- СПб, Бином. Лаборатория знаний, 2006
\end{document}\