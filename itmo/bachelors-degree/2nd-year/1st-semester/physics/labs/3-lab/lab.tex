%
\documentclass[12pt]{article}

\usepackage{mathtext} 
\usepackage{amsmath}

\usepackage[english, russian]{babel}
\usepackage[TS1]{fontenc}
\usepackage[utf8]{inputenc}
\usepackage{pscyr}
\usepackage[left=2cm,right=2cm, top=1cm,bottom=1.5cm,bindingoffset=0cm]{geometry}

\usepackage{multirow}
\usepackage{hhline}

% \usepackage{indentfirst}

\usepackage{enumitem,kantlipsum}

% \usepackage{graphicx}
% \graphicspath{{pictures/}}
% \DeclareGraphicsExtensions{.pdf,.png,.jpg}

% \usepackage{tikz}
% \usetikzlibrary{patterns}
\usepackage{pgfplots}
\pgfplotsset{compat=1.9}
% \usepgfplotslibrary{fillbetween}

% \usepackage{ulem}

% \usepackage{hyperref}  

% \usepackage{circuitikz}

% \usepackage{fp}
% \usepackage{xfp}

% \usepackage{siunitx}
% \sisetup{output-decimal-marker={,}}

% \usepackage{minted}

\let\oldref\ref
\renewcommand{\ref}[1]{(\oldref{#1})}

\begin{document}
    \pagestyle{empty}
    \begin{center}
        \textbf{Федеральное государственное автономное образовательное учреждение высшего образования}
        
        \vspace{5pt}
        
        {\small
            \textbf{САНКТ-ПЕТЕРБУРГСКИЙ НАЦИОНАЛЬНЫЙ ИССЛЕДОВАТЕЛЬСКИЙ  УНИВЕРСИТЕТ ИНФОРМАЦИОННЫХ ТЕХНОЛОГИЙ, МЕХАНИКИ И ОПТИКИ}

            \textbf{ФАКУЛЬТЕТ  ПРОГРАММНОЙ ИНЖЕНЕРИИ И КОМПЬЮТЕРНОЙ ТЕХНИКИ}%
        }

        \vspace{140pt}

        {\Large            
            \textbf{ОТЧЁТ}

            \vspace{7pt}

            \textbf{ПО ЛАБОРАТОРНОЙ РАБОТЕ №3}%
        }

        \vspace{10pt}

        {\large
            \textbf{«Определение постоянной Ридберга} 

            \vspace{5pt}

            \textbf{для атомного водорода»}%
        }

        \vspace{170pt}
        
        \begin{tabular}{lll}
            Проверил:	 	  							                & \hspace{70pt}	&	Выполнил:							        	\\
            Пшеничнов В.Е.	 \_\_\_\_\_\_\_\_\_\_\_\_\_                 &			    &	Студент группы P3255				        	\\
            «\_\_\_\_\_\_» 	\_\_\_\_\_\_\_\_\_\_\_\_\_\_ \the\year г.	& 			    &	Федюкович С. А. \_\_\_\_\_\_\_\_\_\_\_\_\_\_	\\
			                    							            &			    &									            	\\
                                                                        &			    &										            \\
        \end{tabular}

        \vspace*{\fill}

        Санкт-Петербург

        \the\year
    \end{center}
    \newpage
    \pagestyle{plain}
    \setcounter{page}{1}

    \section*{Цель работы}

    Получение  численного  значения  постоянной  Ридберга  для  атомного водорода из экспериментальных данных и его сравнение с рассчитанной теоретически.

    \section*{Теоретические основы лабораторной работы}

    В 1885г. Бальмер показал на примере спектра испускания атомного водорода, что длины волн четырёх линий, лежащих в видимой части и обозначаемых символами $ H_\alpha $,$ H_\beta $, $ H_\gamma $, $ H_\sigma $, можно точно представить эмпирической формулой:
    \begin{equation}
        \label{eq:1}
        \lambda = B \frac{n^2}{n^2-4},
    \end{equation}
    где вместо $ n $ следует подставить числа $ 3 $, $ 4 $, $ 5 $, и $ 6 $; $ В $ --- эмпирическая константа $ 364,61нм $

    Закономерность, выраженная формулой Бальмера, становится особенно наглядной, если представить эту формулу в том виде, в каком ею пользуются в настоящее время. Для этого следует преобразовать ее так, чтобы она позволяла вычислять не длины волн, а частоты или волновые числа.

    Известно, что частота $ \nu = \frac{c}{\lambda_0}, c^{-1}$ --- число колебаний в 1 сек., где $ c $ --- скорость света в вакууме; $ \lambda_0 $ --- длина волны в вакууме.

    Волновое число --- это число длин волн, укладывающихся в $ 1 м $:
    \begin{equation}
        \label{eq:2}
        \widetilde{\nu} = \frac{1}{\lambda} = \frac{1}{B} \cdot \frac{n^2 - 4}{n^2} = \frac{4}{B} \left( \frac{1}{4} - \frac{1}{n^2} \right);
    \end{equation}
    обозначив $ \frac{4}{B} $ через $ R $, перепишем формулу \ref{eq:2}:
    \begin{equation}
        \label{eq:3}
        \widetilde{\nu} = R \cdot \left( \frac{1}{2^2} - \frac{1}{n^2} \right),
    \end{equation}
    где $ n = 3, 4, 5, ... $.

    Уравнение \ref{eq:3}  представляет  собой формулу   Бальмера   в   обычном   виде. Выражение \ref{eq:3}  показывает,  что  по  мере увеличения  $ n $ разность  между  волновыми числами  соседних  линий  уменьшается  и при $ n \to \infty $ мы   получаем   постоянное значение $ \widetilde{\nu} = \frac{R}{4} $.Таким  образом,  линии должны постепенно сближаться, стремясь к предельному  положению $ \widetilde{\nu} = \frac{R}{4} $.

    Предельное  волновое  число, около которого сгущаются линии при $ n \to \infty $,  называется границей  серии. Для  серии  Бальмера  это  волновое число $ \widetilde{\nu} = 2742000м^{-1}$,    и    ему соответствует  значение  длины  волны $ \lambda_0 =364,61нм $.

    Наряду  с  серией  Бальмера  в спектре   атомного   водорода   был обнаружен ряд других серий. Все эти серии   могут   быть   представлены общей формулой:
    \begin{equation}
        \label{eq:4}
        \widetilde{\nu} = R \cdot \left( \frac{1}{n^2_1} - \frac{1}{n^2_1} \right),
    \end{equation}
    где $ n_1 $ имеет для каждой серии постоянное значение $ n_1 =1,  2,  3,  4, 5,...; $ для серии Бальмера $ n_1 = 2; \, n_2 $ --- ряд целых чисел от $ (n_1 + 1) $ до $ \infty $.

    Формула  \ref{eq:4}   называется обобщенной формулой Бальмера. Она выражает собой один из главных законов физики  --- закон, которому подчиняется процесс изучения атома.

    Теория атома водорода и водородоподобных ионов создана Нильсом Бором. В основе теории лежат постулаты Бора, которым подчиняются любые атомные системы.

    Второй квантовый закон относится к переходам с излучением. Согласно этому закону электромагнитное  излучение,  связанное  с  переходом  атомной  системы  из  стационарного состояния  с  энергией $ E_j $  стационарное  состояние  с  энергией $ Е_l < Е_j $,  является монохроматическим, и его частота определяется соотношением:
    \begin{equation}
        \label{eq:5}
        Е_j - Е_i = h \nu,
    \end{equation}
    где $ h $ --- постоянная Планка.

    Стационарные  состояния $ Е_i $ в  спектроскопии  характеризуют  уровни  энергии,  а  об излучении  говорят  как  о  переходах  между  этими  уровнями  энергии.  Каждому возможному переходу  между  дискретными  уровнями  энергии  соответствует  определенная  спектральная линия,   характеризуемая   в   спектре   значением   частоты   (или   волнового   числа) монохроматического излучения.

    Дискретные уровни энергии атома водорода определяются известной формулой Бора:
    \begin{equation}
        \label{eq:6}
        E_n = - \frac{2 \pi^2 m e^4}{h} \cdot \frac{1}{n^2} = - h c R \frac{1}{n^2},
    \end{equation}
    \begin{equation}
        \label{eq:7}
        R = \frac{2 \pi^2 m e^4}{c h^3} \text{ (СГС) или } R = \frac{m e^4}{8 c h^3 \varepsilon^2_0} \text{ (СИ)},
    \end{equation}
    где  $ n $ --- главное квантовое число; $ m $ --- масса электрона (точнее, приведенная масса протона и электрона).

    Для волновых чисел спектральных линий согласно условию частот \ref{eq:5} получается общая формула:
    \begin{equation}
        \label{eq:8}
            \widetilde{\nu} = \frac{E n_2}{hc} - \frac{E n_1}{hc} = \frac{R}{n^2_1} - \frac{R}{n^2_2} = R \left( \frac{1}{n^2_1} - \frac{1}{n^2_2} \right),
    \end{equation}
    где $ n_1 < n_2 $, а $ R $ определяется  формулой  \ref{eq:7}.  При  переходе  между  определенным  нижним уровнем ($ n_1 $ фиксировано) и последовательными верхними уровнями ($ n_2 $ изменяется от \\ $ (n1+1) $ до $ \infty $)  получаются  спектральные  линии  атома  водорода.  В  спектре  водорода  известны следующие серии: серия Лаймана $ (n_1=1, n_2 \geq 2) $; серия Бальмера $ (n_1=2; n_2 \geq 3) $; серия Пашена $ (n_1=3, n_2 \geq 4) $; серия Брекета $ (n_1=4, n_2 \geq 5) $; серия Пфунта $ (n_1=5, n_2 \geq 6) $; серия Хамфри $ (n_1=6, n_2 \geq 7) $.

    Как  видим,  формула  \ref{eq:8}  совпадает  с  формулой  \ref{eq:4},  полученной  эмпирически,  если $ R $ --- постоянная Ридберга, связанная с универсальными константами формулой \ref{eq:7}.

    Из уравнения \ref{eq:3}, отложив по вертикальной оси значения волновых чисел линий серии Бальмера,  а  по  горизонтальной --- соответственно  значения $ 1/n^2 $,  получаем  прямую,  угловой коэффициент которой дает постоянную $ R $, а точка пересечения прямой с осью ординат дает значение $ R/4 $

    Для  определения  постоянной  Ридберга  нужно знать квантовые  числа  линий  серии  Бальмера  атомного  водорода. Длины волн линий водорода определяются с помощью монохроматора (спектрометра).
    
    Изучаемый   спектр   сравнивается   с   линейчатым спектром,  длины  волн  которого  известны.  По  спектру известного  газа, можно построить градуировочную кривую монохроматора, по которой затем определить длины волн излучения атомного водорода.

    \newpage
    \section*{Ход работы}
    \begin{enumerate}[wide, labelwidth=!, labelindent=0pt]
        \item Зажечь  ртутную  лампу  ДРШ.  Для  этого  включить  тумблер  «сеть»  на  источнике питания  $ ЭПС-III $,  включить  тумблер  «лампа  ДРШ», нажать  кнопку  «пуск»и  удерживать  её нажатой $ 2-3 $ секунды.
        \item Установить ширину  входной  щели  примерно  $ 0,1мм $.
        \item Снять градуировочную кривую монохроматора по спектру ртути и заполнить таблицу \ref{tab:1}.
        \begin{table}[h!]
            \caption{Градуировка барабана монохроматора}
            \label{tab:1}
            \centering
            \begin{tabular}{|c|c|}
                \hline
                Длина волны $\lambda, [нм]$ & Угол поворота $ m $  \\
                \hline                
                690,700 & 2900,000\\ 
 \hline 
671,600 & 2825,000\\ 
 \hline 
623,400 & 2660,000\\ 
 \hline 
612,300 & 2620,000\\ 
 \hline 
607,200 & 2590,000\\ 
 \hline 
579,000 & 2460,000\\ 
 \hline 
576,900 & 2450,000\\ 
 \hline 
546,000 & 2270,000\\ 
 \hline 
491,600 & 1850,000\\ 
 \hline 
435,800 & 1190,000\\ 
 \hline 
434,700 & 1175,000\\ 
 \hline 
433,900 & 1140,000\\ 
 \hline 
407,700 & 700,000\\ 
 \hline 
404,600 & 635,000\\ 
 \hline 
                
            \end{tabular}
        \end{table}
        \begin{figure}[h!]
            \label{graph:1}
            \caption{Градуировочная кривая}
            \centering
            \begin{tikzpicture}
				\begin{axis}[		
				xlabel = {$\lambda, [нм]$},
				ylabel = {$m$},	
				xmin = 400,
				ymin = 400,
				xmax = 700,
				ymax = 3000,	
				axis x line=center,
				axis y line=center,
				width = 500,
                height = 295,
                minor x tick num={4},
                minor y tick num={4},
				grid = both
				]
				    \addplot+[black, smooth,mark size=1pt] coordinates {
                        (690.7,2900) (671.6,2825) (623.4,2660) (612.3,2620) (607.2,2590) (579,2460) (576.9,2450) (546,2270) (491.6,1850) (435.8,1190) (434.7,1175) (433.9,1140) (407.7,700) (404.6,635) 
                    };
				\end{axis}
				\end{tikzpicture}
        \end{figure}
        \item Поставить  перед  монохроматором  водородную лампу, обозначив длины волн линий водорода $ \lambda_1 $, $ \lambda_2 $ и $ \lambda_3 $, снять отсчет их положения  $ m' $ по барабану длин волн. Заполнить таблицу \ref{tab:2}. 
        \begin{table}[h!]
            \caption{Определение длин волн спектра излучения атома водорода}
            \label{tab:2}
            \centering
            \begin{tabular}{|c|c|c|c|c|}
                \hline
                Угол поворота $ m' $    & Длина волны $\lambda, [нм]$   & Волновое число $ \widetilde{v}, м^{-1} $  & Квантовое число $ n $     & $ 1/n^2 $\\
                \hline
                2790,000 & 661,376 & 15119,998 & 3,000 & 0,111\\ 
 \hline 
1800,000 & 487,373 & 20518,177 & 4,000 & 0,063\\ 
 \hline 
1160,000 & 434,357 & 23022,529 & 5,000 & 0,040\\ 
 \hline 
                
            \end{tabular}
        \end{table}
        \item По  построенной  градуировочной  кривой  определить  длины  волн  линий  спектра водорода, рассчитать волновые числа для полученных длин. Результаты записать в таблицу \ref{tab:2}.
        \item Построить  график  зависимости $ \widetilde{v}, [см^{-1}] $, от $ 1/n^2 $,  где $ n $ --- соответствующее  главное квантовое число.
        \begin{figure}[h!]
            \label{graph:2}
            \caption{Градуировочная кривая}
            \centering
            \begin{tikzpicture}
				\begin{axis}[		
				xlabel = {$ 1/n^2 $},
				ylabel = {$ \widetilde{v}, [см^{-1}] $},	
				xmin = 0,
				ymin = 14900,
				xmax = 0.115,
				ymax = 28000,	
				axis x line=center,
				axis y line=center,
				width = 500,
                height = 300,
                minor x tick num={4},
                minor y tick num={4},
				grid = both
				]
				    \addplot[only marks,black,mark size=1pt] coordinates {
                        (0.1111111111111111,15119.997800727593) (0.0625,20518.177239745575) (0.04,23022.52918927808) 
                    };
                    \addplot[black] coordinates {
                        (0,27465.464638390975) (0.2471774992151612, 0)
                    };
				\end{axis}
				\end{tikzpicture}
        \end{figure}

        Уравнение аппроксимирующей прямой: 
        \begin{equation}
            \label{eq:ap}
            \widetilde{v} = -111116,363 \cdot (1/n^2) +27465,465
        \end{equation}
        \item Найти постоянную Ридберга двумя способами: 
        \begin{enumerate}
            \item Из углового коэффициента прямой уравнения \ref{eq:ap} получаем $ R =  111116,363 [cm^{-1}]$
            \item Подставив $ 0 $ в уравнение \ref{eq:ap} получаем:
            \begin{equation*}
                R/4 =  27465,465 [см^{-1}]                 
                ; \, R = 109861,859 [см^{-1}]
            \end{equation*}
            Теоретическое значение:
            \begin{equation*}
                R = 109677,593 [см^{-1}]
            \end{equation*}
        \end{enumerate}
        \item Используя  полученное  значение  постоянной  Ридберга, рассчитать энергию ионизации атома водорода, находящегося в основном состоянии:
        \begin{equation*}
            E_и = hcR =  1,054 \cdot 10^{-15} \cdot 299792458 \cdot 109861,859 / 1,6 = 13,464[эВ]
        \end{equation*}
        Теоретическое значение:
            \begin{equation*}
                E =  13,600[эВ] 
            \end{equation*}
    \end{enumerate}
    \section*{Вывод}
    В ходе выполнения данной работы мной был проведён эксперимент по изучению серии Бальмера, в результате которого я подтвердил свои теоретические знания практическим путём. Также экспериментальным путём были получены значений постоянной Ридберга 
    $ R = 111116,363[см^{-1}]; \,R = 109861,859 [см^{-1}]$
    и энергии ионизации атома водорода
    $ E_И = 13,464[эВ] $ 
    , разница которых с теоретическими значениями незначительна и вызвана погрешностью измерений, что опять же подтверждает верность теорий.  
\end{document}