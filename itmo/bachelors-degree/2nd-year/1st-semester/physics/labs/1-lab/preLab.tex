\documentclass[12pt]{article}

\usepackage{mathtext} 
\usepackage{amsmath}

\usepackage[english, russian]{babel}
\usepackage[TS1]{fontenc}
\usepackage[utf8]{inputenc}
\usepackage{pscyr}
\usepackage[left=2cm,right=2cm, top=1cm,bottom=1.5cm,bindingoffset=0cm]{geometry}

\usepackage{multirow}
\usepackage{hhline}

\usepackage{indentfirst}

\usepackage{enumitem,kantlipsum}

% \usepackage{graphicx}
% \graphicspath{{pictures/}}
% \DeclareGraphicsExtensions{.pdf,.png,.jpg}

% \usepackage{tikz}
% \usetikzlibrary{patterns}
\usepackage{pgfplots}
\pgfplotsset{compat=1.9}
% \usepgfplotslibrary{fillbetween}

% \usepackage{ulem}

% \usepackage{hyperref}  

% \usepackage{circuitikz}

% \usepackage{fp}
% \usepackage{xfp}

% \usepackage{siunitx}
% \sisetup{output-decimal-marker={,}}

% \usepackage{minted}

\let\oldref\ref
\renewcommand{\ref}[1]{(\oldref{#1})}

\begin{document}
    \pagestyle{empty}
    \begin{center}
        \textbf{Федеральное государственное автономное образовательное учреждение высшего образования}
        
        \vspace{5pt}
        
        {\small
            \textbf{САНКТ-ПЕТЕРБУРГСКИЙ НАЦИОНАЛЬНЫЙ ИССЛЕДОВАТЕЛЬСКИЙ  УНИВЕРСИТЕТ ИНФОРМАЦИОННЫХ ТЕХНОЛОГИЙ, МЕХАНИКИ И ОПТИКИ}

            \textbf{ФАКУЛЬТЕТ  ПРОГРАММНОЙ ИНЖЕНЕРИИ И КОМПЬЮТЕРНОЙ ТЕХНИКИ}%
        }

        \vspace{140pt}

        {\Large            
            \textbf{ОТЧЁТ}

            \vspace{7pt}

            \textbf{ПО ЛАБОРАТОРНОЙ РАБОТЕ №1}%
        }

        \vspace{10pt}
        
        {\large
            \textbf{«Определение длины световой волны} 

            \vspace{5pt}

            \textbf{при помощи опыта Юнга»}%
        }

        \vspace{170pt}
        
        \begin{tabular}{lll}
            Проверил:	 	  							                & \hspace{70pt}	&	Выполнил:							        	\\
            Пшеничнов В.Е.	 \_\_\_\_\_\_\_\_\_\_\_\_\_                 &			    &	Студент группы P3255				        	\\
            «\_\_\_\_\_\_» 	\_\_\_\_\_\_\_\_\_\_\_\_\_\_ \the\year г.	& 			    &	Федюкович С. А. \_\_\_\_\_\_\_\_\_\_\_\_\_\_	\\
			                    							            &			    &									            	\\
                                                                        &			    &										            \\
        \end{tabular}

        \vspace*{\fill}

        Санкт-Петербург

        \the\year
    \end{center}
    \newpage
    \pagestyle{plain}
    \setcounter{page}{1}

    \section*{Цель работы}

    Определение длины световой волны по интерференционной картине от двух щелей.

    \section*{Теоретические основы}

    Интерференция света это пространственное распределение энергии вызванное суперпозицией электромагнитных волн видимого диапазона. Условием интерференции волн является их когерентность. Когерентность --- это согласованность в протекании колебательных процессов. Необходимая согласованность заключается, в постоянстве разности фаз волн, приходящих в данную точку пространства. Из-за значительной немонохроматичности обычных источников это условие невыполнимо для волн, испускаемых двумя независимыми источниками. Поэтому обычно для получения когерентных световых волн, при наблюдении двухлучевой интерференции, поступают следующем образом: световой пучок от одного источника, разделяют тем или иным способом на два пучка, «идущие» разными путями в одну и ту же область пространства, где и наблюдается интерференция.

    Различают два основных метода получения интерферирующих пучков: метод деления волнового фронта и метод деления амплитуды. Из-за малости длин волн видимого света и требований пространственной когерентности наблюдение интерференции света методом деления волнового фронта сопряжено с определенными сложностями. Один из первых успешных экспериментов, демонстрирующих двухлучевую интерференцию методом деления волнового фронта, был осуществлен Томасом Юнгом в начале $ XIX $ века. Яркий пучок солнечных лучей падает по нормали на экран $ А $ с малым отверстием $ S $. Прошедший через отверстие свет образует расходящийся пучок, который падает на второй экран $ B $ с двумя малыми отверстиями $ S_1 $ и $ S_2 $, расположенными близко друг к другу. Эти отверстия равноудалены от $ S $ и действуют как вторичные синфазные источники. Исходящие от них волны, перекрываясь, создают интерференционную картину, наблюдаемую на удаленном экране $ С $. Измеряя ширину интерференционных полос, Юнг в 1802 г. определил длины световых волн разных цветов, хотя эти измерения и не были достаточно точными.

    В данной лабораторной работе источником служит лазер, обладающий по сравнению с обычными источниками высокой степенью монохроматичности и большой яркостью. Это позволяет наблюдать значительное количество интерференционных полос. Кроме того, лазерное излучение является пространственно когерентным по всему сечению пучка, поэтому, если ширины пучка хватает, чтобы одновременно осветить оба отверстия $ S_1 $ и $ S_2 $ , то можно обойтись без первого экрана с отверстием $ S $. Для увеличения яркости наблюдаемой интерференционной картины вместо точечных отверстий в качестве $ S_1 $ и $ S_2 $ в данной работе используются узкие длинные параллельные друг другу щели.

    Найдем связь периода интерференционной картины с длиной волны в опыте Юнга. Обозначим: $ d $ --- расстояние между источниками $ S_1 $ и $ S_2 $, $ L $ --- расстояние от источников до плоскости наблюдения интерференционной картины, $ x $ --- расстояние от точки $ P $ до центра $ O' $ интерференционной картины. Обычно интерферирующие лучи идут под малыми углами к оси системы $ OO' $, угол $ \theta $ мал, и для него справедливо соотношение: $ \theta \approx x / L $. В этом случае разность хода $ \Delta = r_2 – r_1 $ можно выразить как:
    \begin{equation}
        \label{eq:d}
        \Delta \approx d\theta \approx dx /dL.
    \end{equation}

    При выполнении условия:
    \begin{equation}
        \label{eq:con}
        \Delta = k\lambda,
    \end{equation}
    где $ k $ --- любое целое число, $ \lambda $ --- длина волны света, в точке $ P $ наблюдается интерференционный максимум. Если же:
    \begin{equation}
        \label{eq:d2}
        \Delta = (k + \frac{1}{2})\lambda,
    \end{equation}
    то в точке $ P $ наблюдается минимум.

    Шириной интерференционной полосы(периодом интерференционной картины) называют расстояние между соседними максимумами или минимумами. В данной лабораторной работе период картины определяется по расстоянию между минимумами, поскольку их положения фиксируются точнее. Сравнивая выражения \ref{eq:d} и \ref{eq:d2}, находим координаты минимумов в плоскости $ PO' $ :
    \begin{equation}
        \label{eq:xk}
        xk = (k + \frac{1}{2})\lambda \cdot \frac{L}{d}
    \end{equation}

    Отсюда для ширины полосы получаем:
    \begin{equation}
        \label{eq:dx}
        \Delta x = x_{k+1} - x_k = \lambda \frac{L}{d}.
    \end{equation}

    Для проверки формулы \ref{eq:dx} и увеличения точности определения длины волны период $ \Delta x $ измеряется при нескольких расстояниях $ L $. Как видно из уравнения \ref{eq:dx}, зависимость $ \Delta x $ от $ L $ является линейной, а коэффициент наклона графика этой зависимости $ K = \lambda / d  $. Построив экспериментальный график $ \Delta x $ от $ L $, можно убедиться в том, что зависимость действительно линейна, а по коэффициенту наклона получившейся прямой и известному значению $ d $ определить длину волны.

    \section*{Ход работы}

    \begin{enumerate}[wide, labelwidth=!, labelindent=0pt]
        \item Расстояние до экрана $ X_э = 18,000 [мм] $. Расстояние между щелями $ 0,100 \pm 0,001 $.
        \item Для щели №39:
        $$ Y_1  = 3,620 [мм]; \, Y_2 = 4,540 [мм]; $$
        $$ Y_1  = 3,660 [мм]; \, Y_2 = 4,500 [мм]; $$
        $$ Y_1  = 3,680 [мм]; \, Y_2 = 4,530 [мм]; $$
        $$ d_{ср} = 0,100 \cdot ((4,540 - 3,620 ) + (4,500 - 3,660 ) + (4,530 - 3,680) ) / 3 \approx 0,087 [мм] .$$
        $$ \Delta_{ср} = 0,100 \cdot 1,300 \cdot \sqrt{\frac{1}{N(N-1)} \cdot ((0,920 - 0,879)^2 +}$$
        $$ \overline{+ (0,840 - 0,870)^2 + (0,850 - 0,870)^2)} \approx 0,003 [мм] $$

        Для щели №40:
        $$ Y_1  = 3,600 [мм]; \, Y_2 = 4,540 [мм]; $$
        $$ Y_1  = 3,580 [мм]; \, Y_2 = 4,520 [мм]; $$
        $$ Y_1  = 3,620 [мм]; \, Y_2 = 4,560 [мм]; $$
        $$ d_{ср} = 0,100 \cdot ((4,540 - 3,600 ) + (4,520 - 3,580 ) + (4,560 - 3,620) ) / 3 \approx 0,094 [мм] .$$
        $$ \Delta_{ср} = 0,100 \cdot 1,300 \cdot \sqrt{\frac{1}{N(N-1)} \cdot ((0,940 - 0,940)^2 +}$$
        $$ \overline{+ (0,940 - 0,940)^2 + (0,940 - 0,940)^2)} = 0,000 [мм] $$

        Таким образом, $ \Delta d_{ср} = 0,001 [мм] $.

        \item Для щели №39:
        $$ X = 20,000; 30,000; 40,000; 50,000; 60,000; 70,000 [см] ;$$
        $$ Y = 18,200; 28,200; 38,200; 48,200; 58,200; 68,200 [см] ;$$
        $$ 18,200 : 1 = 6,000[мм] \, ; k = 4,000 ; \, \partial x = 1,500 [мм]  $$
        $$ 28,200 : 1 = 9,000[мм] \, ; k = 4,000 ; \, \partial x = 2,250 [мм]  $$
        $$ 38,200 : 1 = 12,000[мм] \, ; k = 4,000 ; \, \partial x = 3,000 [мм]  $$
        $$ 48,200 : 1 = 15,000[мм] \, ; k = 4,000 ; \, \partial x = 3,750 [мм]  $$
        $$ 58,200 : 1 = 19,000[мм] \, ; k = 4,000 ; \, \partial x = 4,750 [мм]  $$
        $$ 68,200 : 1 = 22,000[мм] \, ; k = 4,000 ; \, \partial x = 5,500 [мм]  $$

        Для щели №40:
        $$ X = 20,000; 30,000; 40,000; 50,000; 60,000; 70,000 [см] ;$$
        $$ Y = 18,200; 28,200; 38,200; 48,200; 58,200; 68,200 [см] ;$$
        $$ 18,200 : 1 = 7,000[мм] \, ; k = 5,000 ; \, \partial x = 1,400 [мм]  $$
        $$ 28,200 : 1 = 10,000[мм] \, ; k = 5,000 ; \, \partial x = 2,000 [мм]  $$
        $$ 38,200 : 1 = 16,000[мм] \, ; k = 6,000 ; \, \partial x = 2,670 [мм]  $$
        $$ 48,200 : 1 = 19,000[мм] \, ; k = 6,000 ; \, \partial x = 3,170 [мм]  $$
        $$ 58,200 : 1 = 15,000[мм] \, ; k = 4,000 ; \, \partial x = 3,750 [мм]  $$
        $$ 68,200 : 1 = 22,000[мм] \, ; k = 5,000 ; \, \partial x = 4,400 [мм]  $$

        \item Графики зависимости длины штриха от расстояния:
        \begin{figure}[h!]
            \label{graph:1}
            \caption{Зависимость длины штриха от расстояния для щели №39}
            \centering
            \begin{tikzpicture}
				\begin{axis}[		
                    xlabel = {$ X [см] $},
                    ylabel = {$ \partial x [мм] $},	
                    xmin = 15,
                    ymin = 0,
                    xmax = 75,
                    ymax = 6,	
                    axis x line=center,
                    axis y line=center,
                    width = 500,
                    height = 250,
                    minor x tick num={4},
                    minor y tick num={4},
                    grid = both
				]
				    \addplot[only marks,black,mark size=1pt] coordinates {
                        (18.2, 1.5) (28.2, 2.25) (38.2, 3) (48.2, 3.75) (58.2, 4.4) (68.2, 5.5)
                    };                    
                    \addplot[black, domain=15:75] {0.0777*x+0.0427};
                    \node[label={-90:{$A$}},circle,fill,inner sep=1pt] at (axis cs:17,1.37) {};
                    \node[label={-180:{$B$}},circle,fill,inner sep=1pt] at (axis cs:73,5.71) {};
                    
				\end{axis} 
				\end{tikzpicture}
        \end{figure}
        \begin{figure}[h!]
            \label{graph:2}
            \caption{Зависимость длины штриха от расстояния для щели №40}
            \centering
            \begin{tikzpicture}
				\begin{axis}[		
                    xlabel = {$ X [см] $},
                    ylabel = {$ \partial x [мм] $},	
                    xmin = 15,
                    ymin = 0,
                    xmax = 75,
                    ymax = 5,	
                    axis x line=center,
                    axis y line=center,
                    width = 500,
                    height = 250,
                    minor x tick num={4},
                    minor y tick num={4},
                    grid = both
				]
				    \addplot[only marks,black,mark size=1pt] coordinates {
                        (18.2, 1.4) (28.2, 2) (38.2, 2.67) (48.2, 3.17) (58.2, 3.75) (68.2, 4.4)
                    };                    
                    \addplot[black, domain=15:75] {0.0593*x+0.3372};
                    \node[label={0:{$A$}},circle,fill,inner sep=1pt] at (axis cs:15,1.25) {};
                    \node[label={-180:{$B$}},circle,fill,inner sep=1pt] at (axis cs:74,4.75) {};
                    
				\end{axis} 
				\end{tikzpicture}
        \end{figure}

        \item Вычисление коэффициентов наклона:
        $$ K_{39} = \frac{B_y - A_y}{B_x - A_x} = \frac{5,750 -1,250}{72,000 - 17,000} \cdot 0,100 \approx 0,008; \, K_{40} = \frac{4,750 -1,250}{74,000 - 15,000} \cdot 0,100 \approx 0,005 $$

        Длины волн для двух серий эксперимента:
        $$ \lambda_{39} = K_{39} \cdot d_{ср} = 0,008 \cdot 0,087 \approx 705,000 [нм].  $$
        $$ \lambda_{40} = K_{40} \cdot d_{ср} = 0,005 \cdot 0,055 \approx 555,000 [нм].  $$

        \item Погрешность для $ \lambda_{40} $:
        $$ \Delta K_{40} = \frac{K_{40}}{B_y - A_y} \cdot \sqrt{\frac{2}{N-2}\cdot ((1,500 – 1,400)^2 + (2,000 - 2,000)^2  +} $$
        $$ \overline{+ (2,500 – 2,670)^2 + (3,350 – 3,170)^2 + (3,750 – 3,750)^2 + (4,500 – 4,400)^2) } \approx 33,000 [нм] $$

        $$ \Delta \lambda_{40} = \sqrt{\left(\frac{d \lambda \Delta K_{40}}{dK}\right)^2 + \left(\frac{d \lambda \Delta d_{ср} }{d \, d_{ср}}\right)^2} \approx 32,500 [нм] $$
        
    \end{enumerate}

    \section*{Вывод} 

    В ходе работы я определил длину световой волны по картине дифракции на круглом отверстии на основе опыта Юнга, которая составила $ \lambda_{39} \approx 705,000[нм] $, также $ \lambda_{40} \approx 555,000\pm 32,500[нм] $

    Диапазон красного цвета спектра определяют длиной волны $ 620—740  [нм]$, поэтому, полученные значения в результате выполнения лабораторной работы попадает под заданные значения диапазона.

    Погрешность составила около $ 8,5 \% $, что является приемлемой погрешностью. Вызвана она в связи с неточностью измерений маленьких  величин, а так же погрешностью при расчетах.       
\end{document}