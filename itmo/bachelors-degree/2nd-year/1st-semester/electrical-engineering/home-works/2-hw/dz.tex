\documentclass[12pt]{article}
\usepackage{mathtext}
\usepackage{amsmath}
\usepackage{multirow}
\usepackage{hhline}

\usepackage[english, russian]{babel}
\usepackage[TS1, T2A]{fontenc}
\usepackage[utf8]{inputenc}
\usepackage{pscyr}

\usepackage{fp}
%\usepackage{pgfplots}
%\pgfplotsset{compat=1.9}
\usepackage{circuitikz}

\usepackage{siunitx}
\sisetup{output-decimal-marker={,}}

\usepackage[left=2cm,right=2cm, top=1cm,bottom=1.5cm,bindingoffset=0cm]{geometry}

\usepackage{graphicx}
\graphicspath{{pictures/}}
\DeclareGraphicsExtensions{.pdf,.png,.jpg}

\begin{document}
    \pagestyle{empty}
    \begin{center}
        \normalsize
        \textbf{Федеральное государственное автономное образовательное учреждение высшего образования}

        \small
        \medskip
        \textbf{САНКТ-ПЕТЕРБУРГСКИЙ НАЦИОНАЛЬНЫЙ ИССЛЕДОВАТЕЛЬСКИЙ  УНИВЕРСИТЕТ ИНФОРМАЦИОННЫХ ТЕХНОЛОГИЙ, МЕХАНИКИ И ОПТИКИ}

        \medskip
        \textbf{ФАКУЛЬТЕТ ПРОГРАММНОЙ ИНЖЕНЕРИИ И КОМПЬЮТЕРНОЙ ТЕХНИКИ}
        \bigskip\bigskip\bigskip\bigskip\bigskip\bigskip\bigskip\bigskip\bigskip\bigskip\bigskip\bigskip
        \par\medskip\par\smallskip\par\smallskip
        \Large
        \textbf{Основы электротехники}

        \textbf{Домашняя работа №2}

        \large
        \par\bigskip
        \textbf{«Расчет цепей синусоидального тока методом комплексных амплитуд»}
        \par\bigskip\par\bigskip\par\bigskip\par\bigskip\par\bigskip\par\bigskip
        \par\bigskip\par\bigskip\par\bigskip\par\bigskip\par\bigskip\par\bigskip
        \par\bigskip\par\bigskip\par\bigskip\par\bigskip\par\bigskip\par\bigskip
        \normalsize
        \begin{tabular}{lllll}
            \hspace{170pt}	 							& \hspace{80pt}	&	Выполнил:								&\\
            &			&	Студент группы P3255					&\\
            & 			&	Федюкович С. А. \_\_\_\_\_\_\_\_\_\_\_\_\_\_	&\\
            &			&										&\\
            &			&										&\\
        \end{tabular}
        \par\bigskip\par\bigskip\par\bigskip
        \par\bigskip \par\bigskip
        \par\bigskip\par\bigskip\par\bigskip\par\bigskip\par\bigskip\par\bigskip\par\bigskip\par\bigskip

        Санкт-Петербург
        \par\bigskip
        2018
    \end{center}
    \newpage
    \pagestyle{plain}
    \setcounter{page}{1}
    \section*{Задание}
    \begin{center}
        \begin{circuitikz} \draw
        (0,0) to[american current source, l=$e(t)$ , i=$i_1$]
        (0,4) to[generic, l=$R_1$, a=$\rightarrow U_1$]
        (4,4) to[capacitor, l=$C_4$, a=$\rightarrow U_4$,i=$i_3$]
        (8,4) --
        (8,0) to[generic,l=$R_5$,a=$\leftarrow U_5$]
        (4,0) -- (0,0)

        (4,4)to[inductor,*-*, l=$L_3$,a=$U_3 \downarrow$,i=$i_2$] (4,0);

            %	\draw (1,1) -- (1,3) -- (3,3) -- (3,1) -- (1,1);
        \end{circuitikz}
    \end{center}

    \FPset{\rOne}{2}
    \FPset{\lThree}{25}
    \FPset{\cFour}{1000}
    \FPset{\rFive}{2}
    \textbf{Дано:}

    \par\bigskip
    $R_1= R_5=\num\rFive[Ом];$

    $L_3=\num\lThree[мГн]; C_4=\num\cFour[мкФ];$

    $i_2=3,861sin(200t-87,199^{\circ})$

    \par\bigskip
    \textbf{Задания:}

    \begin{enumerate}
        \item Методом комплексных амплитуд рассчитать мгновенные значения ЭДС источника, токов в ветвях и напряжений на элементах.
        \item Построить векторные диаграммы для любого контура и любого узла.
        \item Осуществить проверку, составив баланс мощностей.
    \end{enumerate}

    \section*{Решение}
\end{document}