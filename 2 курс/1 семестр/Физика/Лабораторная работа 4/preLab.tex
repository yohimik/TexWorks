\documentclass[12pt]{article}

\usepackage{mathtext} 
\usepackage{amsmath}

\usepackage[english, russian]{babel}
\usepackage[TS1]{fontenc}
\usepackage[utf8]{inputenc}
\usepackage{pscyr}
\usepackage[left=2cm,right=2cm, top=1cm,bottom=1.5cm,bindingoffset=0cm]{geometry}

\usepackage{multirow}
\usepackage{hhline}

% \usepackage{indentfirst}

\usepackage{enumitem,kantlipsum}

% \usepackage{graphicx}
% \graphicspath{{pictures/}}
% \DeclareGraphicsExtensions{.pdf,.png,.jpg}

% \usepackage{tikz}
% \usetikzlibrary{patterns}
\usepackage{pgfplots}
\pgfplotsset{compat=1.9}
\usepgfplotslibrary{polar}
% \usepgfplotslibrary{fillbetween}

% \usepackage{ulem}

% \usepackage{hyperref}  

% \usepackage{circuitikz}

% \usepackage{fp}
% \usepackage{xfp}

% \usepackage{siunitx}
% \sisetup{output-decimal-marker={,}}

% \usepackage{minted}

\let\oldref\ref
\renewcommand{\ref}[1]{(\oldref{#1})}

\begin{document}
    \pagestyle{empty}
    \begin{center}
        \textbf{Федеральное государственное автономное образовательное учреждение высшего образования}
        
        \vspace{5pt}
        
        {\small
            \textbf{САНКТ-ПЕТЕРБУРГСКИЙ НАЦИОНАЛЬНЫЙ ИССЛЕДОВАТЕЛЬСКИЙ  УНИВЕРСИТЕТ ИНФОРМАЦИОННЫХ ТЕХНОЛОГИЙ, МЕХАНИКИ И ОПТИКИ}

            \textbf{ФАКУЛЬТЕТ  ПРОГРАММНОЙ ИНЖЕНЕРИИ И КОМПЬЮТЕРНОЙ ТЕХНИКИ}%
        }

        \vspace{140pt}

        {\Large            
            \textbf{ОТЧЁТ}

            \vspace{7pt}

            \textbf{ПО ЛАБОРАТОРНОЙ РАБОТЕ №4}%
        }

        \vspace{10pt}

        {\large
            \textbf{«Исследование поляризации лазерного луча»} 

            \vspace{5pt}

            \textbf{}%
        }

        \vspace{170pt}
        
        \begin{tabular}{lll}
            Проверил:	 	  							                & \hspace{70pt}	&	Выполнил:							        	\\
            Пшеничнов В.Е.	 \_\_\_\_\_\_\_\_\_\_\_\_\_                 &			    &	Студент группы P3255				        	\\
            «\_\_\_\_\_\_» 	\_\_\_\_\_\_\_\_\_\_\_\_\_\_ \the\year г.	& 			    &	Федюкович С. А. \_\_\_\_\_\_\_\_\_\_\_\_\_\_	\\
			                    							            &			    &									            	\\
                                                                        &			    &										            \\
        \end{tabular}

        \vspace*{\fill}

        Санкт-Петербург

        \the\year
    \end{center}
    \newpage
    \pagestyle{plain}
    \setcounter{page}{1}

    \section*{Цель работы}

    Исследование характера поляризации лазерного излучения и экспериментальная проверка закона Малюса.

    \section*{Теоретические основы лабораторной работы}

    Поперечные волны обладают особым, присущим только им, свойством, известным под названием поляризация. Под этим понимается пространственное соотношение между направлением распространения светового луча и направлением колебания вектора напряженности электрического $ \vec{E} $  (или магнитного $ \vec{H} $ ) поля. Теория Максвелла для электромагнитной волны утверждает только, что векторы напряженности электрического и магнитного полей лежат в плоскости, перпендикулярной направлению распространения света, но не накладывает никаких ограничений на их поведение в этой плоскости. Друг относительно друга вектора $ \vec{E} $ и $ \vec{H} $ ориентированы взаимно перпендикулярно. Поэтому для описания колебаний в световой волне достаточно указывать один из них. Исторически таким вектором выбран вектор напряженности электрического поля $ \vec{E} $, который также называют световым.

    Если при распространении световой волны направление колебаний электрического вектора $ \vec{E} $ бессистемно, хаотически изменяется с равной амплитудой и, следовательно, любое его направление в плоскости, перпендикулярной распространению волны, равновероятно, то такой свет называют неполяризованным, или естественным. Если колебания электрического вектора фиксированы строго в одном направлении, свет называется линейно- или плоскополяризованным.

    Плоскость, образованная направлением распространения электромагнитной волны и направлением колебаний вектора напряженности электрического поля, называется плоскостью поляризации электромагнитной волны.

    Поляризация света наблюдается при отражении и преломлении света на границе прозрачных изотропных диэлектриков. Если угол падения естественного света на границу раздела двух прозрачных диэлектриков отличен от нуля, то отраженный и преломленный пучки оказываются частично-поляризованными. В отраженном свете преобладают колебания вектора $ \vec{E} $ , перпендикулярные к плоскости падения, а в преломленном свете --- параллельные плоскости падения. Степень поляризации обеих волн (отраженной и преломленной) зависит от угла падения. Соответствующую зависимость в 1815 г. установил шотландец Дэвид Брюстер. Как показали опыты, при некотором значении угла падения светового луча на границу раздела двух сред с показателями преломления $ n_1 $ и $ n_2 $ соответственно, угол между отраженным и преломленным лучом становится равен $ 90^\circ $. При таком условии отраженный луч оказывается полностью поляризован (колебания вектора $ \vec{E} $  в нем перпендикулярны плоскости падения). Прошедший луч поляризован частично и содержит преимущественно параллельную составляющую вектора $ \vec{E} $ . Тогда значение угла, соответствующего полной поляризации отраженного луча, определяется из закона преломления:
    \begin{equation}
        \label{eq:bru}
        \frac{n_2}{n_1} = \frac{\sin{\alpha}}{\sin{\beta}} = \frac{\sin{\alpha}}{\sin{(90^\circ  - \alpha)}} = \frac{\sin{\alpha}}{\cos{\alpha}} = tg{\alpha}
    \end{equation}

    Степень поляризации преломленной волны при угле падения, равном углу Брюстера, достигает максимального значения, однако эта волна остается лишь частично поляризованной. Так как коэффициент отражения света в данном случае значительно меньше единицы (около $ 0,15 $ для границы раздела воздух-стекло), можно использовать преломленный свет, повышая его степень поляризации путем ряда последовательных отражений и преломлений. Это осуществляют с помощью, так называемой стопы, состоящей из нескольких одинаковых и параллельных друг другу пластинок, установленных под углом Брюстера к падающему свету. При достаточно большом числе пластинок проходящий через эту систему свет будет практически полностью линейно-поляризованным. И интенсивность прошедшего через такую стопу света (в отсутствие поглощения) будет равна половине падающего на стопу естественного света.

    Эта идея нашла высокоэффективное использование в лазерах, где торцы разрядной трубки представляют собой плоскопараллельные стеклянные пластинки, расположенные под углом Брюстера к оси трубки. Поэтому излучение, распространяющееся вдоль оси трубки между зеркалами и поляризованное в плоскости падения на пластинки, многократно проходит сквозь них практически беспрепятственно, не испытывая отражения. В результате из лазера выходит луч, поляризованный в этой плоскости, что и показано на рисунке. Другая составляющая излучения, плоскость поляризации которой перпендикулярна плоскости падения, почти полностью удаляется из пучка благодаря отражениям.

    Для получения, обнаружения и анализа плоскополяризованного света используют приспособления, называемые поляризаторами. Поляризаторы могут быть сконструированы на основе рассмотренного отражения и преломления света на границе раздела двух сред, также на основе двойного лучепреломления (призмы Николя), на основе явления дихроизма. Поляризаторы свободно пропускают колебания вектора $ \vec{E} $, параллельные плоскости, которую называют плоскостью пропускания поляризатора. Колебания же, перпендикулярные к этой плоскости, задерживаются полностью или частично. Широкое распространение для получения плоскополяризованного света имеют поляризаторы, действие которых основано на явлении дихроизма --- селективного поглощения света в зависимости от направления колебаний электрического вектора световой волны. Сильным дихроизмом обладают кристаллы турмалина.

    Для получения плоско-поляризованного света применяются также поляроиды --- пленки на которые, как правило, наносятся кристаллики герапатита --- двоякопреломляющего вещества с сильно выраженным дихроизмом в видимой области. Так, при толщине $ \approx 0,1 мм $ такая пленка полностью поглощает лучи с перпендикулярными к плоскости падения колебаниями $ \vec{E} $  в видимой области спектра, являясь в таком тонком слое хорошим поляризатором. Недостаток поляроидов по сравнению с поляризационными призмами --- их недостаточная прозрачность, селективность поглощения при разных длинах волн и небольшая термостойкость.

    Поляризаторы можно  использовать и в качестве анализаторов --- для определения характера и степени поляризации интересующего нас света. Пусть на анализатор падает линейно-поляризованный свет, вектор $ \vec{E_1} $  которого составляет угол $ \varphi $  с плоскостью пропускания $ P $. Анализатор пропускает только ту составляющую вектора $ \vec{E_1} $ , которая параллельна его плоскости пропускания $ P $, т.е. $ E_2 = E_1 \cdot \cos{\varphi} $ . Интенсивность пропорциональна квадрату модуля светового вектора ( $ I \sim E^2 $ ), поэтому интенсивность прошедшего света:
    \begin{equation}
        \label{eq:in}
        I_2 = I_1 \cdot \cos^2{\varphi},
    \end{equation}
    где $ I_1 $ --- интенсивность падающего плоскополяризованного света. Это соотношение было установлено в 1810 г. французским физиком Этьеном Луи Малюсом и носит название закона Малюса.
    \newpage
    \section*{Ход работы}
    \begin{enumerate}[wide, labelwidth=!, labelindent=0pt]
        \item Включить источник (лазер) и вольтметр.
        \item Аккуратно убрать поляроид из хода луча лазера. Записать показания вольтметра (соответствует интенсивности  $ I_0 $).
        \item Вставить поляроид в ход луча, зафиксировав его между направляющими стержнями. Вращая поляроид, следить за изменения показаний вольтметра. Найти таким образом положение максимума пропускания и, начиная с него провести измерения интенсивности прошедшего через поляроид излучения в зависимости от угла поворота поляроида. Таким образом, с шагом в $ 10^\circ $, записывать показания вольтметра пока не будет сделан целый оборот поляроида вокруг светового пучка. Данные измерений занести в таблицу \ref{tab:1}.
        \begin{table}[h!]
            \caption{Экспериментальные данные}
            \label{tab:1}
            \centering
            \begin{tabular}{|c|c|c|c|}
                \hline
                Угол поворота $ \varphi, ^\circ $    &  Интенсивность $ I, мА $ &  $ I/I_{max}, мА $ & $ \cos^2{(\varphi - \varphi_m)} $ \\
                \hline                
                %nodePrint table1
                \multicolumn{4}{|c|}{ %nodePrint data
                }  \\          
                \hline   
            \end{tabular}
        \end{table}
        \item Проанализировав записанные показания вольтметра, найти максимальное $ I_{max} $, соответствующее углу $ \varphi_m $, и разделить каждое из экспериментальных значений $ I $ на $ I_{max} $. Результаты занести в таблицу \ref{tab:1}. 
        \item Построить график зависимости нормированной интенсивности $ I/I_{max} $ от угла $ \varphi $, соединив полученные экспериментальные точки аппроксимирующей кривой. На той же координатной плоскости построить график зависимости $ \cos^2{(\varphi - \varphi_m)} $ от угла поворота поляроида $ \varphi $:
        \begin{figure}[h!]
            \label{graph:1}
            \caption{Зависимость интенсивности и квадрата косинуса от угла поворота }
            \centering
            \begin{tikzpicture}
                \begin{polaraxis}[height = 400, width = 400]
                    \legend{$I/I_{max} ( \varphi)$,$\cos^2 ( \varphi)$}
                    \addplot+[red, mark size = 2pt, smooth] coordinates {
                        %nodePrint graph1
                    }; 
                    \addplot+[blue, mark size = 2pt, smooth] coordinates {
                        %nodePrint graph2
                    };
                \end{polaraxis}
            \end{tikzpicture}
        \end{figure}
        \item Найти коэффициенты пропускания использованного поляроида для параллельной и перпендикулярной ориентации его плоскости пропуская по отношению направлению колебаний вектора $ \vec{E} $ в излучении лазера:
        $$ k_{\parallel } = I_{max} / I_0 = %nodePrint kpar
        $$
        $$ k_{\bot } = I_{min} / I_0 = %nodePrint kper
        $$
    \end{enumerate}

    \section*{Вывод}

    В ходе выполнения данной работы мной был проведён эксперимент по изучению поляризации и проверке закона Малюса, в результате которого я подтвердил свои теоретические знания практическим путём. Также по построенному графику видно, что графики кривых почти совпадают, из чего следует, что закон Малюса верен, а небольшие расхождения вызваны погрешностью измерений.
\end{document}