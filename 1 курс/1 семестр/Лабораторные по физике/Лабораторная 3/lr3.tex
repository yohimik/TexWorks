\documentclass[12pt]{article}
\usepackage{hhline}
\usepackage{graphicx}
\graphicspath{{pictures/}}
\DeclareGraphicsExtensions{.png}
\usepackage{multirow}
\usepackage{amsmath}
\usepackage{mathtext}
\usepackage[T2A]{fontenc}
\usepackage[utf8]{inputenc}
\usepackage{pscyr} 
\usepackage[left=2cm,right=2cm, top=1.5cm,bottom=1cm,bindingoffset=0cm]{geometry}

\begin{document}
\pagestyle{empty}
\begin{center}
\large{\textbf{Университет ИТМО}}
\end{center}
\rule{500pt}{1pt}
\par\bigskip\par\bigskip\par\bigskip\par\bigskip\par\bigskip\par\bigskip\par\bigskip\par\bigskip
\begin{center}
\Large
\textbf{Отчёт по лабораторной работе №3}

\textbf{\textit{«Определение момента инерции крестовины при различном расположении грузов»}}


\end{center}
\par\bigskip\par\bigskip\par\bigskip\par\bigskip\par\bigskip\par\bigskip\par\bigskip\par\bigskip\par\bigskip\par\bigskip\par\bigskip\par\bigskip\par\bigskip\par\bigskip      
\begin{flushright}
\large
Выполнил: Федюкович С. А.
\par\bigskip
Факультет: МТУ “Академия ЛИМТУ”
\par\bigskip
Группа: S3100                       
\par\bigskip\par\bigskip\par\bigskip

\rule{150pt}{0.5pt}
\par\bigskip\par\bigskip\par\bigskip\par\bigskip                                                            
 Проверил: Пшеничников В. Е. 
\par\bigskip \par\bigskip

\rule{150pt}{0.5pt}
\end{flushright}
\par\bigskip\par\bigskip\par\bigskip\par\bigskip\par\bigskip\par\bigskip\par\bigskip\par\bigskip\par\bigskip\par\bigskip     
\begin{center}
\large
Санкт-Петербург
\par\bigskip
2018
\end{center}
\newpage

\section*{Цель работы}
Измерить момент инерции крестовины при заданном расположении грузов на спицах относительно оси вращения.
\section*{Теоретические основы лабораторной работы}
Момент инерции вращающейся системы зависит от распределения массы относительно оси вращения. Эта зависимость имеет вид $I\sim R^2$. В данной работе $R$ --- расстояние от центра груза на спице до оси вращения. Положение груза на первой риске соответсвует $R = 67$ мм, расстояние между рисками $25$ мм.

Основное уравнение динамики вращательного движения в проекции на ось вращения для вращающейся крестовины записывается следующим образом:
\begin{equation}
M_{н}-M_{тр}=I,\varepsilon
\end{equation}
где $M_{н}$ – момент силы натяжения нити, вызывающей вращение; $M_{тр}$ – момент сил трения; $\varepsilon$ – угловое ускорение, $I$ – момент инерции системы.

Вращение крестовины вызвано поступательным движение каретки с шайбами. Это движение описывается следующим  уравнениям динамики: 
\begin{equation}
mg-F_{н}=ma,	
\end{equation}
Здесь $m$ – масса каретки с шайбами, $F_{н}$ – сила натяжения нити. 

Сила натяжения из уравнения $(2)$:
\begin{equation}
F_{н} = mg – ma	
\end{equation}

Считая движение каретки равноускоренным, можно вычислить ускорение по формуле:
\begin{equation}
a = \frac{2h}{t^2}.
\end{equation}

Подстановка выражения $(4)$ в формулу $(3)$ даёт:
\begin{equation}
	F_{н}=m(g-\frac{2h}{t^2 }).	
\end{equation}

Соотвественно момент силы натяжения:
\begin{equation}
	M_{н} = F_{н}r,	
\end{equation}	
Где $r$ – радиус ступицы. 

Выражая радиус ступицы через её диаметр $r=\frac{d}{2}$ и учитывая формулу $(5)$, получаем:
\begin{equation} 
	M_{н}=\frac{md}{2} (g-\frac{2h}{t^2 }).	
\end{equation}

При отсутствии проскальзывания нити, угловое ускорение, с которым вращается система, связано с линейным ускорением через радиус:
	\begin{equation}
	 a=\varepsilon r=\varepsilon \frac{d}{2},
	 \end{equation}
где $d$ – диаметр ступицы, $d = 46,0$ мм.

Объединение формул $(4)$ и $(8)$ даёт расчётную формулу для углового ускорения:
\begin{equation}	
\varepsilon=\frac{4h}{dt^2 }.	
\end{equation}

Из уравнения динамики $(1)$ вращающий момент силы натяжения
\begin{equation}
	M_{н}=M_{тр}+I\varepsilon.
	\end{equation}
 График функции $M_{н}=f(\varepsilon)$ представляет собой прямую линию.

\end{document}