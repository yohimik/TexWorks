\documentclass[12pt]{article}
\usepackage{hhline}
\usepackage{graphicx}
\graphicspath{{pictures/}}
\DeclareGraphicsExtensions{.png}
\usepackage{multirow}
\usepackage{amsmath}
\usepackage{mathtext}
\usepackage[T2A]{fontenc}
\usepackage[utf8]{inputenc}
\usepackage{pscyr} 
\usepackage[left=2cm,right=2cm, top=1.5cm,bottom=1cm,bindingoffset=0cm]{geometry}
\begin{document}
\pagestyle{empty}
\begin{center}
\large{\textbf{Университет ИТМО}}
\end{center}
\rule{500pt}{1pt}
\par\bigskip\par\bigskip\par\bigskip\par\bigskip\par\bigskip\par\bigskip\par\bigskip\par\bigskip
\begin{center}
\Large
\textbf{Лабораторная работа №1}

\textbf{\textit{«Основные понятия линейного программирования»}}


\end{center}
\par\bigskip\par\bigskip\par\bigskip\par\bigskip\par\bigskip\par\bigskip\par\bigskip\par\bigskip\par\bigskip\par\bigskip\par\bigskip\par\bigskip\par\bigskip\par\bigskip      
\begin{flushright}
\large
Выполнил: Федюкович С. А.
\par\bigskip
Факультет: МТУ “Академия ЛИМТУ”
\par\bigskip
Группа: S3100                       
\par\bigskip\par\bigskip\par\bigskip

\rule{150pt}{0.5pt}
\par\bigskip\par\bigskip\par\bigskip\par\bigskip                                                            
 Проверил: ...
\par\bigskip \par\bigskip

\rule{150pt}{0.5pt}
\end{flushright}
\par\bigskip\par\bigskip\par\bigskip\par\bigskip\par\bigskip\par\bigskip\par\bigskip\par\bigskip\par\bigskip\par\bigskip     
\begin{center}
\large
Санкт-Петербург
\par\bigskip
2018
\end{center}
\newpage
\section*{Теоретические основы лабораторной работы}

Линейное программирование – это направление математического
программирования, изучающее методы решения экстремальных задач, которые
характеризуются линейной зависимостью между переменными и линейным критерием.

Необходимым условием постановки задачи линейного программирования являются
ограничения на наличие ресурсов, величину спроса, производственную мощность
предприятия и другие производственные факторы.

Сущность линейного программирования состоит в нахождении точек наибольшего
или наименьшего значения некоторой функции при определенном наборе ограничений,
налагаемых на аргументы и образующих систему ограничений, которая имеет, как
правило, бесконечное множество решений. Каждая совокупность значений переменных
(аргументов функции $F$), которые удовлетворяют системе ограничений, называется
допустимым планом задачи линейного программирования. Функция $F$, максимум или
минимум которой определяется, называется целевой функцией задачи. Допустимый план,
на котором достигается максимум или минимум функции $F$, называется оптимальным
планом задачи.

Система ограничений, определяющая множество планов, диктуется условиями
производства. Задачей линейного программирования (ЗЛП) является выбор из множества
допустимых планов наиболее выгодного (оптимального).

В общей постановке задача линейного программирования выглядит следующим
образом:

Имеются какие-то переменные $х = (х_1,  х_2 , … х_n )$ и функция этих переменных $f(x) =
f (х_1, х_2, … х_n )$, которая носит название целевой функции. Ставится задача: найти
экстремум (максимум или минимум) целевой функции $f(x)$ при условии, что переменные x
принадлежат некоторой области $G$:

\begin{center}
$\begin{cases}
   f(x) \Rightarrow extr \\
  x \in G 
 \end{cases}$
 \end{center}
 
Линейное программирование характеризуется:
\begin{itemize}
\itemфункция $f(x)$ является линейной функцией переменных $х_1 , х_2 , … х_n$
\itemобласть $G$ определяется системой линейных равенств или неравенств.
\end{itemize}
\end{document}