\documentclass[12pt]{article}
\usepackage[russian]{babel}
\usepackage{hhline}
\usepackage{graphicx}
\graphicspath{{pictures/}}
\DeclareGraphicsExtensions{.png}
\usepackage{multirow}
\usepackage{amsmath}
\usepackage{mathtext}
\usepackage[T2A]{fontenc}
\usepackage[utf8]{inputenc}
\usepackage{pscyr} 
\usepackage[left=1.5cm,right=1.5cm, top=1cm,bottom=1cm,bindingoffset=0cm]{geometry}
\begin{document}

\pagestyle{empty}
\begin{center}
\large{\textbf{Университет ИТМО}}
\end{center}
\rule{525pt}{1pt}
\par\bigskip\par\bigskip\par\bigskip\par\bigskip\par\bigskip\par\bigskip\par\bigskip\par\bigskip
\begin{center}
\Large
\textbf{Лабораторная работа №4}

\textbf{\textit{«Транспортная задача. Методы нахождения начального решения транспортной задачи»}}


\end{center}
\par\bigskip\par\bigskip\par\bigskip\par\bigskip\par\bigskip\par\bigskip\par\bigskip\par\bigskip\par\bigskip\par\bigskip\par\bigskip\par\bigskip\par\bigskip\par\bigskip      
\begin{flushright}
\large
Выполнил: Федюкович С. А.
\par\bigskip
Факультет: МТУ “Академия ЛИМТУ”
\par\bigskip
Группа: S3100                       
\par\bigskip\par\bigskip\par\bigskip

\rule{150pt}{0.5pt}
\par\bigskip\par\bigskip\par\bigskip\par\bigskip                                                            
 Проверила: Авксентьева Е. Ю.
\par\bigskip \par\bigskip

\rule{150pt}{0.5pt}
\end{flushright}
\par\bigskip\par\bigskip\par\bigskip\par\bigskip\par\bigskip\par\bigskip\par\bigskip\par\bigskip\par\bigskip\par\bigskip     
\begin{center}
\large
Санкт-Петербург
\par\bigskip
2018
\end{center}
\newpage
\section*{Теоретические основы лабораторной работы}
Частным случаем задачи линейного программирования является транспортная задача, которая в общем виде состоит в определении оптимального плана перевозок некоторого груза из $m$ пунктов отправления
 $А_1 , А_2 , ..., А_m$ в $n$ пунктов назначения  $B_1 , B_2 , ..., B_n$. 

Теорема. Любая транспортная задача, у которой суммарный объем запасов совпадает с
суммарным объемом потребностей, имеет решение.

Методы составления опорного плана транспортной задачи:
\begin{enumerate}
\item Метод северо-западного угла --- заключается в последовательном удовлетворении потребностей каждого $j-го$ потребиля за счет $i-го$  поставщика. Процесс
продолжается до тех пор, пока все потребители не будут удовлетворены.
\item Метод минимальной стоимости --- заключается в том, что из всей таблицы стоимостей выбирается наименьшая,
и в клетку, которая ей соответствует, помещается меньшее из чисел $A_i$, или $B_j$. Затем, из
рассмотрения исключается либо строка, соответствующая поставщику, запасы которого
полностью израсходованы, либо столбец, соответствующий потребителю, потребности
которого полностью удовлетворены, либо и строку и столбец, если израсходованы запасы
поставщика и удовлетворены потребности потребителя. Из оставшейся части таблицы
стоимостей снова выбирают наименьшую стоимость, и процесс распределения запасов
продолжают, пока все запасы не будут распределены, а потребности удовлетворены.
\item Метод аппроксимации Фогеля --- заключается в поиске наибольших разностей между двумя наимененьшими стоимостями перевозок, из которых после формируется опорный план.
\item Метод двойного предпочтения --- заключается в посике наименьших стоимостей в каждом столбце и строке и из их пересечений формируется опорное решение.
\end{enumerate}

\newpage
\section*{Решение заданий}

Составить опорные планы различными методами, сравнить значения суммарной
стоимости перевозок по каждому плану: 

\subsection*{Задача 1}

\begin{table}[h!]
\begin{center}
\begin{tabular}{|c|c|c|c|c|c|}
\hline
           & $B_1$ & $B_2$ & $B_3$ & $B_4$ & $A_i$	\\
\hline
 $A_1$ & 2 & 3 & 2 & 4 & 30	\\
\hline
 $A_2$ & 3 & 2 & 5 & 1 & 40	\\
\hline
 $A_3$ & 4 & 3 & 2 & 6 & 20	\\ 
\hline
 $B_j$ & 20 & 30 & 30 & 10 & 90	\\
\hline
\end{tabular}
\end{center}
\end{table} 
\subsection*{Решение}
Решим методом северо-западного угла:
\begin{table}[h!]
\begin{center}
\begin{tabular}{|c|c|c|c|c|c|}
\hline
           & $B_1$ & $B_2$ & $B_3$ & $B_4$ & $A_i$	\\
\hline
 $A_1$ & $20^2$ & $10^3$  & 2 & 4 & 30	\\
\hline
 $A_2$ & 3 & $20^2$  & $20^5$  & 1 & 40	\\
\hline
 $A_3$ & 4 & 3 & $10^2$  & $10^6$  & 20	\\ 
\hline
 $B_j$ & 20 & 30 & 30 & 10 & 90	\\
\hline
\end{tabular}\\

Стоимость: 290
\end{center}
\end{table} 

Решим методом минимальной стоимости:
\begin{table}[h!]
\begin{center}
\begin{tabular}{|c|c|c|c|c|c|}
\hline
           & $B_1$ & $B_2$ & $B_3$ & $B_4$ & $A_i$	\\
\hline
 $A_1$ & 2 & 3 & $30^2$ & 4 & 30	\\
\hline
 $A_2$ & 3 & $30^2$ & 5 & $10^1$ & 40	\\
\hline
 $A_3$ & $20^4$ & 3 & 2 & 6 & 20	\\ 
\hline
 $B_j$ & 20 & 30 & 30 & 10 & 90	\\
\hline
\end{tabular}\\
Стоимость: 210
\end{center}
\end{table} 
\newpage
\subsection*{Задача 2}
\begin{table}[h!]
\begin{center}
\begin{tabular}{|c|c|c|c|c|c|c|}
\hline
           & $B_1$ & $B_2$ & $B_3$ & $B_4$ & $B_5$ & $A_i$	\\
\hline
 $A_1$ & 2 & 7 & 3 & 6 & 2 & 30	\\
\hline
 $A_2$ & 9 & 4 & 5 & 7 & 3 & 70	\\
\hline
 $A_3$ & 5 & 7 & 6 & 2 & 4 & 50	\\
\hline
 $B_j$ & 10 & 40 & 20 & 60 & 20 & 150	\\
\hline
\end{tabular}
\end{center}
\end{table} 
\subsection*{Решение}
Решим методом северо-западного угла:
\begin{table}[h!]
\begin{center}
\begin{tabular}{|c|c|c|c|c|c|c|}
\hline
           & $B_1$ & $B_2$ & $B_3$ & $B_4$ & $B_5$ & $A_i$	\\
\hline
 $A_1$ & $10^2$ & $20^7$ & 3 & 6 & 2 & 30	\\
\hline
 $A_2$ & 9 & $20^4$ & $20^5$ & $30^7$ & 3 & 70	\\
\hline
 $A_3$ & 5 & 7 & 6 & $30^2$ & $20^4$ & 50	\\
\hline
 $B_j$ & 10 & 40 & 20 & 60 & 20 & 150	\\
\hline
\end{tabular}\\

Стоимость: 690
\end{center}
\end{table} 

Решим методом минимальной стоимости:
\begin{table}[h!]
\begin{center}
\begin{tabular}{|c|c|c|c|c|c|c|}
\hline
           & $B_1$ & $B_2$ & $B_3$ & $B_4$ & $B_5$ & $A_i$	\\
\hline
 $A_1$ & $10^2$ & 7 & 3 & 6 & $20^2$ & 30	\\
\hline
 $A_2$ & 9 & $40^4$ & $20^5$ & $10^7$ & 3 & 70	\\
\hline
 $A_3$ & 5 & 7 & 6 & $50^2$ & 4 & 50	\\
\hline
 $B_j$ & 10 & 40 & 20 & 60 & 20 & 150	\\
\hline
\end{tabular}\\
Стоимость: 390
\end{center}
\end{table} 
\newpage
\subsection*{Задача 3}
\begin{table}[h!]
\begin{center}
\begin{tabular}{|c|c|c|c|c|c|c|}
\hline
           & $B_1$ & $B_2$ & $B_3$ & $B_4$ & $B_5$ & $A_i$	\\
\hline
 $A_1$ & 4 & 2 & 5 & 7 & 6 & 20	\\
\hline
 $A_2$ & 7 & 8 & 3 & 4 & 5 & 110	\\
\hline
 $A_3$ & 2 & 1 & 4 & 3 & 2 & 120	\\
\hline
 $B_j$ & 70 & 40 & 30 & 60 & 50 & 250	\\
\hline
\end{tabular}
\end{center}
\end{table} 
\subsection*{Решение}
Решим методом северо-западного угла:
\begin{table}[h!]
\begin{center}
\begin{tabular}{|c|c|c|c|c|c|c|}
\hline
           & $B_1$ & $B_2$ & $B_3$ & $B_4$ & $B_5$ & $A_i$	\\
\hline
 $A_1$ & $20^4$ & 2 & 5 & 7 & 6 & 20	\\
\hline
 $A_2$ & $50^7$ & $40^8$ & $20^3$ & 4 & 5 & 110	\\
\hline
 $A_3$ & 2 & 1 & $10^4$ & $50^3$ & $60^2$ & 120	\\
\hline
 $B_j$ & 70 & 40 & 30 & 60 & 50 & 250	\\
\hline
\end{tabular}\\

Стоимость: 1120
\end{center}
\end{table} 

Решим методом минимальной стоимости:
\begin{table}[h!]
\begin{center}
\begin{tabular}{|c|c|c|c|c|c|c|}
\hline
           & $B_1$ & $B_2$ & $B_3$ & $B_4$ & $B_5$ & $A_i$	\\
\hline
 $A_1$ & 4 & 2 & 5 & 7 & $20^6$ & 20	\\
\hline
 $A_2$ & 7 & 8 & $30^3$ & $50^4$ & $30^5$ & 110	\\
\hline
 $A_3$ & $70^2$ & $40^1$ & 4 & 3 & $10^2$ & 120	\\
\hline
 $B_j$ & 70 & 40 & 30 & 60 & 50 & 250	\\
\hline
\end{tabular}\\
Стоимость: 760
\end{center}
\end{table} 
\newpage
\subsection*{Задача 4}
\begin{table}[h!]
\begin{center}
\begin{tabular}{|c|c|c|c|c|c|c|}
\hline
           & $B_1$ & $B_2$ & $B_3$ & $B_4$ & $B_5$ & $A_i$	\\
\hline
 $A_1$ & 2 & 8 & 4 & 6 & 3 & 120	\\
\hline
 $A_2$ & 3 & 2 & 5 & 2 & 6 & 30	\\
\hline
 $A_3$ & 6 & 5 & 8 & 7 & 4 & 40	\\
\hline
 $A_4$ & 3 & 4 & 4 & 2 & 1 & 60	\\
\hline
 $B_j$ & 30 & 90 & 80 & 20 & 30 & 250	\\
\hline
\end{tabular}
\end{center}
\end{table} 
\subsection*{Решение}
Решим методом северо-западного угла:
\begin{table}[h!]
\begin{center}
\begin{tabular}{|c|c|c|c|c|c|c|}
\hline
           & $B_1$ & $B_2$ & $B_3$ & $B_4$ & $B_5$ & $A_i$	\\
\hline
 $A_1$ & $30^2$ & $90^8$ & 4 & 6 & 3 & 120	\\
\hline
 $A_2$ & 3 & 2 & $30^5$ & 2 & 6 & 30	\\
\hline
 $A_3$ & 6 & 5 & $40^8$ & 7 & 4 & 40	\\
\hline
 $A_4$ & 3 & 4 & $10^4$ & $20^2$ & $30^1$ & 60	\\
\hline
 $B_j$ & 30 & 90 & 80 & 20 & 30 & 250	\\
\hline
\end{tabular}\\

Стоимость: 1360
\end{center}
\end{table} 

Решим методом минимальной стоимости:
\begin{table}[h!]
\begin{center}
\begin{tabular}{|c|c|c|c|c|c|c|}
\hline
           & $B_1$ & $B_2$ & $B_3$ & $B_4$ & $B_5$ & $A_i$	\\
\hline
 $A_1$ & $30^2$ & $10^8$ & $80^4$ & 6 & 3 & 120	\\
\hline
 $A_2$ & 3 & $10^2$ & 5 & $20^2$ & 6 & 30	\\
\hline
 $A_3$ & 6 & $40^5$ & 8 & 7 & 4 & 40	\\
\hline
 $A_4$ & 3 & $30^4$ & 4 & 2 & $30^1$ & 60	\\
\hline
 $B_j$ & 30 & 90 & 80 & 20 & 30 & 250	\\
\hline
\end{tabular}\\
Стоимость: 870
\end{center}
\end{table} 
\end{document}